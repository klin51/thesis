\phantomsection
\addcontentsline{toc}{chapter}{Bibliography}
\chapter*{Bibliography}

\begin{hangparas}{0.5in}{1}
Aspin, Leigh. ``Tempo, Form, And Counterpoint: Notes On A Graphic Analysis Of Glenn Gould's Two Recordings Of The Fugue In F Sharp Minor From Bach's Well- Tempered Clavier, Book II.'' Glenngould 5.1 (1999): 22-23. RILM Abstracts of Music Literature. Web. 31 Mar. 2015.

Barbour, J. Murray. ``Bach and the Art of Temperament.'' The Musical
Quarterly 33, 1946.

.--- ``Irregular Systems of Temperament.'' Journal of the American Musicological Society 1, no. 3, 1948, pp. 20-27.

.--- Tuning and Temperament, a History Survey. East Lansing: Michigan State College Press, 1951.

Barnes, John. ``Bach's Keyboard Temperament: Internal Evidence From The Well- Tempered Clavier.'' Collected Work: Baroque music. Series: Library of essays on music performance practice, No. 3 Published by: Farnham: Ashgate, 2011. ISBN: 978-0-7546-2882-8; 0-7546-2882-5. Pages: 129-146. (AN: 2011-15070).  n.p.: 2011. RILM Abstracts of Music Literature. Web. 31 Mar. 2015.

Bobbitt, Richard. ``Das Wohltemperierte Clavier: Tuning and Musical Structure,'' English Harpsichord Magazine II (April, 1980), 137-140.

Cook, Nicholas. ``Structure And Performance Timing In Bach's C-Major Prelude (WTC I): An Empirical Study.'' Music Analysis 6.3 (1987): 257-272. RILM Abstracts of Music Literature. Web. 31 Mar. 2015.

Duffin, Ross W. How Equal Temperament Ruined Harmony (and Why You Should Care). New York: W.W. Norton, 2007. Print.

Halperin, Tamar. Bach's Choice of Key: A Determining Factor in the Compositional Strategy Drawn from a Selection of His Instrumental Works. , 2009.  Print.

Haluska, Ján. ``Uncertainty measures and well-tempered systems'' General Systems vol.  31 no. 1, 2002, pp. 73-96.

Jackendoff, Ray. ``Musical Processing And Musical Affect.'' (1991): RILM Abstracts of Music Literature. Web. 8 Apr. 2015

Jorgensen, Owen Henry. ``Forgotten Sounds Of Music.'' Piano Technicians Journal 14.10 (1971): 16-18. RILM Abstracts of Music Literature. Web. 10 Apr. 2015.

.--- The Equal-beating Temperaments: A Handbook for Tuning Harpsichords and Fortepianos, with Tuning Techniques and Tables of Fifteen Historical Temperaments. Raleigh: Sunbury, 1981. Print.

.--- Tuning: Containing the Perfection of Eighteenth-century Temperament, the Lost Art of Nineteenth-century Temperament, and the Science of Equal Temperament, Complete with Instructions for Aural and Electronic Tuning.  East Lansing, MI: Michigan State UP, 1991. Print.

.--- Tuning the Historical Temperaments by Ear: A Manual of Eighty-nine Methods for Tuning Fifty-one Scales on the Harpsichord, Piano, and Other Keyboard Instruments. Marquette: Northern Michigan UP, 1977. Print.

Kellner, Herbert Anton. ``Considering the Tempering Tonality B-Major in Well- Tempered Clavier II'' Bachvol. 30 no. 1, 1999, pp. 10-25.

.--- A Mathematical Approach Reconstituting J. S. Bach's Keyboard Temperament, Bach X (1979), 2-9.

Ledbetter, David. Bach's Well-tempered Clavier: The 48 Preludes and Fugues. New Haven: Yale UP, 2002. Print.

Lehman, Bradley. ``Bach's extraordinary temperament: our Rosetta Stone -- 1,'' Early Music Vol. 33, No. 1, February 2005, pp. 3-23.

.--- ``Bach's extraordinary temperament: our Rosetta Stone - 2, ''Early Music Vol. 33, No. 2, May 2005, pp. 211-231.

Lindley, Mark, and Ibo, Ortgies. ``Bach-Style Keyboard Tuning.'' Early Music 34.4 (2006): 613-623. RILM Abstracts of Music Literature. Web. 10 Apr. 2015.

Lindley, Mark. ``J.S. Bach's Tunings.'' The Musical Times And Singing Class Circular 126.1714 (1985): 721-26. RILM Abstracts of Music Literature.  Web. 31 Mar. 2015.

.--- ``Valuable Nuances Of Tuning For Part 1 Of J.S. Bach's Das Wohl Temperirte Clavier.'' (2011): RILM Abstracts of Music Literature. Web. 10 Apr. 2015.

.--- ``Temperaments,'' ``Equal Temperament,'' ``Just Intonation,'' ``Meantone,'' ``Pythagorean Intonation,'' ``Well-Tempered Clavier,'' The New Grove Dictionary of Music and Musicians, ed. Stanley Sadie. London: Macmillan, 1980.

O'Donnell, John. ``Bach's Temperament, Occam's Razor, And The Neidhardt Factor.'' Early Music 34.4 (2006): 625-633. RILM Abstracts of Music Literature.  Web. 10 Apr. 2015.

Pirro, André. The Aesthetics of Johann Sebastian Bach. New York: n.p., 1938.  Print. Potgieter, Zelda. ``Analyses of selected works from the Well-tempered clavier of J.S.  Bach: A synthesis of existing approaches.'' (1999). RILM Abstracts of Music Literature. Web. 31 Mar. 2015.

Rasch, Rudolf A. ``Does 'Well-tempered' mean 'Equal-tempered'?'', in Williams (ed.), Bach, Händel, Scarlatti tercentenary Essays, Cambridge University Press, Cambridge, 1985, pp. 293-310.

Reinhard, Johnny. Bach and Tuning. N.p.: n.p., 2009. Print.

Steblin, Rita. A History of Key Characteristics in the Eighteenth and
Early Nineteenth Centuries. Rochester, NY: U of Rochester, 2002. Print.

Tukey, John W. Exploratory Data Analysis. Reading, MA: Addison-Wesley
Publishing Company, 1977. Print.

Werckmeister, Andreas. Musicalische Temperatur (Quedlinburg 1691).  Rudolf Rasch, ed. Tuning and Temperament Library, Vol. I. Utrecht: The Diapason Press, 1983.

.--- Musiktheoretische Schriften: Harmonologia musica / Musikalische Paradoxal- Discourse Faksimile der Ausgaben Frankfurt/Leipzig 1702 und Quedlinburg 1707. Introductory essay (German and English) to the facsimile reprint, Laaber-Verlag, 2002.

Wolff, Christoph. Johann Sebastian Bach: The Learned Musician. New York: W.W.  Norton, 2000. Print.
\end{hangparas}
