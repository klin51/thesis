    
    
    

    \hypertarget{Methods}{\chapter{Methods}\label{Methods}}
    \section{The Analytical Framework and Principles of
Analysis}\label{the-analytical-framework-and-principles-of-analysis}

    This section contains the in-depth explanation behind the logos of all
the analysis of the data that has been generated for this dissertation,
providing interpretation and discussion of the motivation behind the
organizational structure of the analytical model progressing from
general to local, as well as the hybrid analytical approach of employing
both computationally driven statistical methods along with theoretical
musical ones from the approach of traditional score analysis. It is also
in this section that I will clearly define the constraints of the thesis
and the null and alternative hypotheses, as well as provide a
comprehensive outline and layout of the content of the three analysis
chapters to follow.

The following are the main principles that will guide the analytical
methods and approach for this dissertation.

    \subsection{Constraints}\label{constraints}

The purpose of these following constraints is to limit and clearly
define the scope of the body of data, as well as ensure for consistency
and comparability across the elements that we will be analyzing.
Consistency across the data is important, because disparity that is
unaccounted for within the data itself can be dangerous in that it can
easily result in unwanted artifacts in the analysis and conclusions.
Feasibility of the scope is also an issue, as a corpus too large in this
instance may prove unwieldy for detailed, localized analysis. The trick
here is the find the balance between a data set that is large enough to
be able to provide substantial enough data points for analysis, but
uniform enough within its elements so that comparisons are not
confounded by inherent incongruities. For these reasons, I have elected
to limit the data to only on the \textbf{minor mode fugues from the
first book of the WTC}. The reasons behind these constraints are as
follows:

\subsubsection{Focus on the First Volume of the Well-Tempered
Clavier}\label{focus-on-the-first-volume-of-the-well-tempered-clavier}

The two volumes of the \emph{Well-Tempered Clavier} were written about
twenty years apart (1722 and 1742 respectively), and while their
organizational form of providing a prelude and fugue set for each key,
progressing chromatically through the 24 major and minor keys, remains
the same, the two collections vary quite drastically in terms of Bach's
stylistic as well as pedagogical approach. Notably, Book I was
envisioned as more of a cohesive unit, in part for pedagogical purposes,
while Book II was compiled with far less a systematic approach,
comprised of pieces from diverse origins assembled from across different
time periods of Bach's life (Ledbetter, 2002, p. 137). For this reason,
I will elect to focus on only the first volume for the sake of
consistency and internal cohesion.

\subsubsection{Focus on Minor Mode}\label{focus-on-minor-mode}

Because a sizable portion of my investigation relies upon intervallic
analysis and comparisons, the difference between the modal scheme of the
major and minor modes can easily present confounding artifacts, as I am
essentially comparing between two very different populations (e.g. the
semitone placement within major and minor mode have inherently different
harmonic functions and voice leading implications). Comparison within
mode, on the other hand, ensures for more uniformity between elements
being compared, and is thus the far better option. In this dissertation,
I have chosen to select the minor mode pieces within the WTC I at the
exclusion of major mode. There is no principle dictating that one could
not have addressed the thesis through analysis of the major mode
compositions, but preference was given to minor mode on account of the
greater amount of idiosyncratic material in terms of dissonant intervals
and chromaticism employed in these compositions.

\subsubsection{Focus on Fugues, Exclusion of
Preludes}\label{focus-on-fugues-exclusion-of-preludes}

The clarity and unambiguity of thematic definition and hierarchy built
into the genre of the Baroque fugue makes it an ideal candidate for
analysis dealing with establishing musical significance on the basis
that it safeguards against the pitfalls of subconscious subjectivism
(this principle will be expounded upon more in detail in chapter 6 that
deals with fugal and thematic analysis), as well as provides a platform
of balanced comparison across compositions. While the exact definition
of the term \emph{fuga} may be elusive due to the complexity behind the
tradition of its development (Ledbetter, 2002, p.73; Mann, 1965, p.5),
what can definitively be observed about the Baroque fugue is the
consistency of treatment of the subject matter as unequivocally
thematically central, and its adherence to polyphonic imitative texture,
two features that allow for consistency within analysis. To expound upon
the latter point of texture, this is not to say that fugues do not
differ in textural setting, which they indeed do, as one only needs to
look at the vast differences between the choral-like texture of the
5-voice B-flat minor fugue, compared against the relative starkness of
the 2-part texture of the E-minor fugue. Rather, the Baroque fugue's
engagement of clear vocal separation and imitation technique ensures
clear delineation between voices, allowing us to reliably track the
vertical and horizontal structures within a piece.

On the other hand, the high variance in the range of compositional
styles and techniques represented in the preludes (Ledbetter, 2002,
p.51) renders these compositions considerably more challenging in terms
of unambiguously establishing thematic certainty and hierarchy, as well
vocal tracking, making comparison across the often very disparate
compositions difficult. For all these reasons I will choose to focus
upon the fugue for analysis.

\subsubsection{Focus on Intervallic
Analysis}\label{focus-on-intervallic-analysis}

The main unit of focus for analysis in this dissertation will be
intervals, as this is the smallest element and musical building block
that temperament directly interacts with, and their various
distributions to characterize how temperament interacts with the musical
and compositional substrate. I will be analyzing intervals on both the
horizontal and vertical domains, looking at distributions as a function
of various musical and temperamental elements, such as keys, key groups,
and interval size (cents), and in varying degrees of scope and musical
settings, all of which we will cover in depth later in this section
dealing with the model and structure of analysis.

\subsubsection{Interest in the Pythagorean Minor
Fugues}\label{interest-in-the-pythagorean-minor-fugues}

This final constraint is not a hard one, but rather partially something
that we will see emerge from the structure as we proceed through the
analysis. This dissertation leans towards the analysis of the
Pythagorean minor keys in particular, as within the minor mode fugues,
it is this group of keys that contains subjects that are more centered
on featuring singular intervals as thematic material, as well as contain
a higher level of idiosyncratic dissonant intervals, and extensive
chromaticism within their subjects, making the interval-based
temperamental analysis of this dissertation more readily applicable
(both of these principles will become increasingly more elucidated as
the analysis unfolds). Additionally, the intervals that will be the most
telling in terms of how temperament was used as an intentional coloring
device (as opposed to an artifact of avoidance) are precisely a) those
intervals idiosyncratic to the new tuning system that were ``unlocked'',
so to speak, through the advent of well-temperament and the complete,
unequal distribution of the comma, and b) dissonant intervals,
especially their extreme counterparts that are dramatically tempered,
and thus viewed as ``least pure'' (e.g. the Pythagorean minor third at
294 cents, 22 cents narrower than its pure, 316 cent counterpart), both
of which are manifested far more in the Pythagorean keys.

One of the principles behind selection---or avoidance---of
certain types of intervals is the very basic notion of the preservation
of consonance in music, and one can regard Barnes' aforementioned study,
as well as the historical predilection towards certain contrapuntal
gestures that draw attention to specific intervals, as a testament to
this principle of maximizing consonance. To wax slightly philosophical
for a moment though, the ultimate effect and import behind consonance
cannot be appraised without the presence of dissonance; in essence, the
notion of consonance is a perceptual one, and it is governed by its
relationship with dissonance, which determines the way these consonances
are presented and ultimately received. Thus, the maximization of
consonance depends not only on how much raw consonance (i.e. presence of
pure over Pythagorean or tempered consonant intervals) is present in the
system, but also how these consonances relate to the dissonances
present. Practically, this can manifest itself in two different ways: a)
the strategic placement of dissonance in a system can make the advent of
a pure consonant interval even more powerful, and b) non-pure consonant
intervals can be perceived as more consonant through preceding
dissonance, an exercise in the power of relativity, if you will.
Furthermore, it is important to consider that at times, consonance may
not be the ultimate function that we wish to maximize if a certain
musical affect is to be achieved; rather, perhaps more germane a matter
to be discussed is the relationship between the dissonance and
consonance in a given system, and the organization of how each is
approached and resolved. Indeed, at the heart of this relationship
between temperament and compositional style is the interaction between
dissonance and consonance---and herein lies the importance of adherence
towards the original temperament in which a compositional was conceived
under; to represent the composition with a different type of temperament
would inevitably bring about an alteration of the intervals employed,
which would in turn disrupt the balance between dissonance and
consonance that is so integral to the affect that a particular piece
carries.

Furthermore, the Pythagorean minor keys (along with their relative
Pythagorean major keys) occupy a special office, in that they are the
keys that were in a sense ``unlocked'' through the system of
well-temperament, being as these were the meantone wolf keys that were
previously inaccessible. These keys differ from the traditional meantone
diatonic keys in that they contain the intervals of the pure fifth (as
opposed to narrower, tempered fifths) and the Pythagorean third (a
narrower minor third---the semiditone---for minor keys, and wider major
third for major keys), as opposed to meantone derived thirds that are
close to just. At a preliminary glance, Bach's treatment of these
Pythagorean minor preludes and fugues in Book I differ from those of
other keys in that he not only employs a~\emph{stile antico}~approach to
some of these compositions (Ledbetter (2002, p.86) notes that the only
two fugues written in this tradition from Book I are set in Pythagorean
minor keys, C-sharp minor and B-flat minor), but also within the
framework of his~prima practica~treatment of counterpoint, he often
features and draws attention to~highly expressive seconda practica~type
dissonances and chromaticism (minor 9th leap in the B-flat minor fugue
subject, augmented 4th leap in the G-sharp minor fugue subject,
diminished fourth leap in the C-sharp minor fugue subject, chromatic
fourth leap in the F minor fugue subject) that are not found in the
preludes and fugues of the more diatonic keys. Because this dissertation
is interested in analyzing how temperament is employed as an expressive
force, as well as Bach's compositional sensitivity to the musical
elements of this new tuning system, these remote Pythagorean keys are of
exceptional interest.

\subsubsection{Usage of Werckmeister III for
Analysis}\label{usage-of-werckmeister-iii-for-analysis}

The specific tuning system of well-temperament that I will be tuning the
scores to for analysis will be Werckmeister III. The usage of
Werckmeister III as the main tuning system in this dissertation's
analysis is not an attempt to make a statement that Bach necessarily
adhered strictly to Werckmeister III as his personal tuning system, but
rather is motivated by the fact that it was the tuning system that was
one of the most well-documented at the time, thus lending itself very
readily to codification and computation. Furthermore, it is important to
stress that Werckmeister III is a type of well-temperament, and because
the systems of well-temperament are all designed and set up from the
same approach of tuning, the placement of tempered intervals are all
very similar across different individual styles, so any other reasonable
system of well-temperament (e.g. Kirnberger) should lend itself to the
same conclusions through testing and analysis.

    \subsection{Hypotheses}\label{hypotheses}

Defining the terms of the thesis in the form of a clear null and
alternative hypothesis, and providing what we would expect the resultant
data to look like under each model is the most straightforward way to
guide the framework of the analysis, and to decide upon a clear
conclusion from the results. The expectations and predictions outlined
here are more general, as they are essentially only a beginning point;
as we progress through the analytical chapters and build upon our
knowledge of the relationship between temperament and the compositions,
our expectations will become more defined and elucidated. While our hope
is that through the progression of the dissertation's analysis we will
be able to amass enough evidence to reject the null hypothesis, we
cannot be consciously guided by this motivation, and must remain as
initially impartial as possible. If we do not find enough evidence to
support our alternative hypothesis we must accept the null hypothesis
for this dissertation; this does not necessarily mean that the
alternative hypothesis is incorrect, but that we cannot demonstrate its
validity through the current framework at this time. It could be that
the framework is not detailed enough to register the fine-grained
temperamental differences that we are looking for, or that the
structures that temperament affects are too complex to reliably parse
them out through the computational model we are relying upon.
Nonetheless, if it comes to this point we will discuss the implications
of the data, to what extent we can make claims about temperament and the
Bach's compositions, and where to possibly progress from there.

\subsubsection{The Null Hypothesis:}\label{the-null-hypothesis}

\begin{quote}
There is no significant, musical connection between the system of
well-temperament and Bach's compositional approach given key. In other
words, temperament has no intrinsic bearing upon the artistic integrity
of a given piece, and is not a significant dimension of musical
expressivity, but rather a historical artifact that exerts no musical
power over the aesthetic outcome of a composition.
\end{quote}

\textbf{Reasons to accept the null hypothesis (from macro to local):}

\begin{itemize}
\tightlist
\item
  There are no notable patterns in the structure of a fugue's overall
  key temperament graph (KDE plot) that are preserved across
  transposition.

  \begin{itemize}
  \tightlist
  \item
    No temperamental traits from original keys or key groups are
    detectable after transposition; i.e., from the graphs of all 12
    fugues that are transposed to the same key, the original fugue
    pitched in that key is unidentifiable. Furthermore, there is no
    cohesion of key groups preserved, but rather all the graphs appear
    scrambled and random.
  \end{itemize}
\item
  There are no incidents in which Bach systematically favors the usage
  of a more dissonant interval over a purer version of the interval
  (principle of avoidance check).
\item
  There are no quantifiable connections between the temperamental scheme
  of a key and key group, and the frequency of the usage of particular
  intervals that are associated with it via temperament (i.e., perfect
  fifths and Pythagorean keys; thirds and sixths and meantone keys).

  \begin{itemize}
  \tightlist
  \item
    Distributions mapping these particular untempered interval
    frequencies as a function of key will look random and uniform.
  \item
    Comparisons between frequency of usage of specific tempered versions
    of intervals in the actual fugue compared to expected values for the
    particular tempered interval given key will show no observable,
    systematic patterns (expected value check)\\
  \end{itemize}
\item
  There are no connections between the temperamental scheme of a key and
  the prominent intervals featured in its fugal subject and themes.
\item
  Distributions mapping certain types of idiosyncratic intervals, or
  intervals within a specific, thematic/motivic setting (e.g. opening
  subject fifths, minor ninths, slow opening subject semitones) will
  show no correlation between the employment of these intervals and
  key/key area; distributions of these intervals as a function of cents
  will show no clear predilection towards certain tempered versions, but
  instead will be random/uniform.
\item
  We can demonstrate that certain themes with specific temperamental
  configurations that we believe to be important to a fugue are still
  retained---or perhaps even greater optimized---after
  transposing piece to other keys (transposition check).
\end{itemize}

\textbf{Implications of accepting the null hypothesis:}

\begin{itemize}
\tightlist
\item
  Keys are not inherently distinct, and transposition will not affect a
  composition's artistic integrity.
\item
  Equal temperament, or any other circular temperament that does not
  contain unacceptable wolf intervals, is an equally viable choice to
  well-temperament in terms of the ideal presentation of these pieces,
  as the micro tuning differences, while physically and acoustically
  present, and perhaps even aurally detectable, are not perceptually
  significant on an artistic level.
\item
  Pursuit of avenues of temperamental analysis are limited in their
  value to teach us more about the artistic significance of compositions
\end{itemize}

\subsubsection{The Alternative
Hypothesis:}\label{the-alternative-hypothesis}

\begin{quote}
There exists a significant, musical connection between the system of
well-temperament and Bach's compositional approach given key.
Temperament is an important and irreplaceable dimension of musical
expressivity, and plays a significant role in controlling the artistic
integrity of a given piece. It is not merely a dismissible historical
artifact, but rather holds a key role in shaping functional and thematic
compositional factors that influence the aesthetic outcome of a
composition.
\end{quote}

\textbf{Reasons to reject the null hypothesis:}

\begin{itemize}
\tightlist
\item
  There are clear patterns in the structure of a fugue's overall key
  temperament graph (KDE plot) that are preserved across transposition.

  \begin{itemize}
  \tightlist
  \item
    Temperamental traits from original keys or key groups are detectable
    after transposition; i.e. from the graphs of all 12 fugues that are
    transposed to the same key, there are elements from the original
    fugue pitched in that key that are still detectable, and we will
    witness patterns of harmonically close keys and members of the same
    key group clustering together (i.e. one's ability to identify the
    original key fugue, and fugues originally pitched in harmonically
    close keys are above chance).
  \end{itemize}
\item
  There are incidents in which Bach systematically favors the usage of a
  more dissonant interval over a purer version of the interval
  (principle of avoidance check).
\item
  There are quantifiable and observable connections between the
  temperament scheme of a key and key group, and the frequency of the
  usage of particular intervals that are associated with it via
  temperament (i.e. perfect fifths and Pythagorean keys).

  \begin{itemize}
  \tightlist
  \item
    Distributions mapping these particular untempered interval
    frequencies as a function of key will show systematic and smooth
    correlations between frequency and certain keys.
  \item
    Comparisons between frequency of usage of specific tempered versions
    of intervals in the actual fugue compared to expected values for the
    particular tempered interval given key will show consistently more
    extreme values in the case of the actual fugue.
  \end{itemize}
\item
  There are observable connections between the temperamental scheme of a
  key and the prominent intervals featured in its fugal subject and
  themes.
\item
  Distributions mapping certain types of idiosyncratic intervals, or
  intervals within a specific, thematic/motivic setting (e.g. opening
  subject fifths, minor ninths, slow opening subject semitones) will
  show clear correlation between the employment of these intervals and
  key/key area; distributions of these intervals as a function of cents
  will show strong and consistent predilections towards certain tempered
  versions.
\item
  We can demonstrate that certain themes with specific temperamental
  configurations that we believe to be important to a fugue are weakened
  or completely lost after transposing piece to other keys
  (transposition check); even stronger yet is if we can demonstrate that
  the home key is the optimal solution compared against all other
  transpositions.
\end{itemize}

\textbf{Implications of rejecting the null hypothesis:}

\begin{itemize}
\tightlist
\item
  Keys are inherently special and carry characteristics and traits
  unique to themselves and their key groups, which would be disrupted
  through transposition.
\item
  Realizing compositions written under well-temperament in equal
  temperament, or any other circular temperaments will undoubtedly
  result in the loss of a portion of the artistic design that the
  composer intended for the piece.
\item
  The development of a better understanding of temperament and its
  creative forces, as well embarking upon temperament related analyses
  of pieces are valuable for fostering a deeper understanding of the
  underlying aesthetic structure and integrity of a piece, similar to
  the importance of promoting sensitivity to other domains of musical
  expressivity such as dynamics, motivic and thematic development, and
  harmonic structure.
\end{itemize}

    \subsection{Establishing Agnosticism through Exploratory Data
Analysis}\label{establishing-agnosticism-through-exploratory-data-analysis}

The principle of initial agnosticism is important to establishing
objectivism in our inquiry. Through the hybrid analytical approach of
computational and theoretical methods, as well as the structural
organization from general analysis to increasingly focused and localized
analysis, the idea is to establish a nonpartisan as possible starting
point, and have the structure emerge through a series of tiered
questions, which will then inform the next set of questions and course
of data exploration until we get to a final level of analysis detailed
enough to adequately answer the questions of the thesis. The reason the
computational model combined with the tiered organizational (general to
local) approach is so powerful is that this setup allows us to be data
driven, and thus establish a maximum level of agnosticism as we progress
to increasingly more detailed and parameterized stages of analysis.

\subsubsection{Hybrid Computation and Theoretical
Model}\label{hybrid-computation-and-theoretical-model}

The analysis will be a hybrid approach, applying a computational
framework (which I will explain in detail in the second portion of this
chapter) that works in conjunction with traditional theoretical score
analysis. The strength behind the dual model lies in the way the two
frameworks contribute to each other symbiotically, filling in each
other's shortcomings: in short, the computational framework establishes
statistical significance, and the theoretical framework establishes
musical significance.

The computational data in this chapter will be presented primarily in
the form of dataframes (to be discussed in the following section on the
computation framework), and plots charting distributions of musical
elements both within individual scores and across the entire corpus,
with particular focus upon distributions of horizontal and vertical
intervals (tempered and untempered), with various parameters set upon
them to observe how they interact with and possibly relate to different
musical domains. The computational model allows us to establish
statistical significance through its ability to parse through large data
sets, providing us that important, initial disinterested bird's eye view
on the general corpus sans parameterization to look for overarching
trends that would be impossible to achieve through traditional, manual
searches. Additionally, the power in the computational model lies in its
extremely speedy and efficient way to mine the data for structure
through various levels of parameterization, computations that generate
outputs in milliseconds that would otherwise take us days to perform
manually, which in turn guide and inform the next level of data for
analysis. This creates a process that is data driven, as well as
extremely prolific and generative, all of which equate to more robust
statistical power. Lastly, the two important tests that we will rely
upon to verify the validity of our statements of
significance---transposition and comparison against expected
values---are extremely cumbersome and, in many cases, flat out
unfeasible to perform by hand. The computational framework enables us to
utilize these tests, as well as produce more accurate outputs that
safeguard against accidental mishandling of the data and personal bias.

However, a pure computational model with no theoretical backing has its
dire pitfalls, as musical structures are extremely complex and
hierarchical, and the first thing we quickly encounter is how difficult
it is for the computational framework to parse out musically significant
elements and perform simple hierarchical tasks, such as separating the
main musical line from secondary, embellishing material (passing tones,
neighbor tones, etc.), something that human perceivers do with great
ease and accuracy. Computers are very good at performing literal tasks,
but when it comes to the complex language of music, formulating,
defining, and codifying the tasks into a set of understandable rules and
statements for computation is surprisingly difficult. Without
theoretical analysis in tandem guiding our computational framework, and
constantly checking if our analysis is musically sensible, our results
could easily become riddled with artifacts that do not accurately
reflect anything that is going on musically in the compositions
analyzed. Therefore, while a great deal of the data that we look at will
consist of distributions of intervals and other musical elements, with
the focus on analyzing these distributions from a statistical angle, at
all times the discussion of the implications of the data will remain
grounded in the spirit of traditional score analysis, as at its core
this is a dissertation about musical significance and import, and these
distributions, regardless of their statistical allure, are sterile and
meaningless without the relevant musical context and underlying
framework.

With the different scopes of analysis (which will be discussed in the
section immediately to follow) will be varying levels of incorporation
of statistical and musical discourse; the initial data dealing with
global trends will be more statistically driven, as the establishment of
statistical significance and agnosticism is of heightened importance in
these early, general sections. As the analysis progresses, and we
proceed to add more and more detailed and focused musical parameters,
there will be a gradual shift towards more theoretical and musical
discourses that bears a closer resemblance to traditional harmonic and
formal analysis looking at contrapuntal, structural, motivic, and
thematic elements, and the engagement of more qualitative musical,
emotional language and abstract concepts that the musician is generally
more accustomed to.

Lastly, although this dissertation is not primarily focused on the
developmental process of the computational framework and its
applications, the efforts to develop the framework have been quite
extensive, and through the example of the analysis of this dissertation,
I hope I can establish what this model of hybrid analysis can achieve,
not only in terms of an optimal mode of temperamental analysis, but also
to the extent of more general forms of musical analysis as well. As the
dissertation unfolds, I will continue to make evident how these
computational tools aid traditional score analysis, and how the both can
be combined to complement one another and be used together to address
musical questions on a larger and more accurate scope, as well as to
quantify and identify trends on a macro level that would be beyond the
capabilities of traditional theoretical analysis. In the end, it is
important to remember that traditional theoretical analysis should
not---and certainly cannot---be replaced by means of
computation, but instead of adopting one modality over the other, the
both can work together in tandem to achieve musical analyses of greater
breadth and depth than any of these modalities can accomplish
separately.

\subsubsection{Tiered General-to-Local Structural
Organization:}\label{tiered-general-to-local-structural-organization}

The analysis is presented in three chapters, with each chapter exploring
the relationship between temperament and composition on a different
level of scope and parameterization, starting from general and moving
towards increasingly more localized and specific, with the goal of
achieving a more holistic view and precise characterization of
temperament's influence on composition. Each chapter focuses on a
different level of parameters that looks at a progressively more
limited, specific subset of the musical population, which is directly
informed and set up by the analysis and results of the chapter preceding
it.

The reason behind this systematic progression from broad to narrow is
again motivated by upholding the principle of agnosticism, and allowing
the results from the data on the current level to guide the kinds of
questions and analyses that will be performed on the next level.
Additionally, each level of scope examines different facets of
temperament's interaction with the musical substrate to ultimately
provide a more comprehensive picture that would otherwise be incomplete
if we were only zooming in on one limited level of the terrain. Similar
to the logos behind the hybrid approach of computational and theoretical
analysis, analysis at both the general and various local levels also
creates a necessary balance of statistical significance with musical
significance. The tiered method allows the analysis to be data driven,
and gives us confidence that the parameterization that we choose at any
given level is not merely constraining the data out of subjective
reasons, but informed by structures that we have already observed on the
larger level. For this reason, although this method of proceeding
through the data is slower and more cumbersome, it ensures that, even on
the most detailed level of analysis, the musical significance in the
claims that we will be making will be backed up by statistical
significance built up through the groundwork of the previous chapters.

Indeed, while the end goal of this thesis is to ultimately reach the
level of localized, detailed score analysis, as it is on this level that
we will be able directly address the musically focused, goal questions
of the dissertation, focusing initially and directly on localized
incidents without knowledge of the entire larger terrain can lead to
problems of being able to confidently characterize the whole by what
could seem as an arbitrary part. Showing examples from individual
incidents of intervals found within certain compositional structures,
and making inferences off of that singular instance may be compelling,
and surely has its rightful place, but the purpose of taking the initial
general surveys is to hedge against counterarguments that raise the
question of statistical significance, stating that no correlation or
causation can necessarily be inferred from any isolated incident. It is
thus important that this dissertation examines the work as a whole if it
wants to make the statement about correlation or causation, and affirm
that the usage of certain intervals was an intentional, systematic, and
purposeful decision that was indeed informed by temperament, instead of
due to some other musical element, or a mere fortunate coincidence.

\subsubsection{Transposition as a Verification
Tool}\label{transposition-as-a-verification-tool}

Throughout all the levels of analysis, I will be applying transposition
as a check for the validity of significance of trends that are detected,
as well as statements about temperament and any sort of special
interaction with key. The ideology behind transposition is that if we
really believe that there are unique causalities and correlations
between temperament and a musical trend being observed, these trends
should be disrupted through transposing. If these connections are
preserved or even strengthened through transposition, then we have
reason to reevaluate our claims.

\subsubsection{Averaged/Expected Value Graphs as a Verification
Tool}\label{averagedexpected-value-graphs-as-a-verification-tool}

For each of the keys, I will be computing averaged values graphs for all
the tempered intervals contained therein, derived through transposing
all the 12 available minor fugues to each key, normalizing to control
for length, and summing up the fugues to arrive at a set of expected
values of tempered intervals within each key (process will be explained
more in detail when we encounter it in the first analysis
chapter).\footnote{NB: Graphs of averages can only be computed in the cases of untempered interval graphs, or graphs of compositions either in, or transposed, to the same key. A tempered graph of averages for pieces belonging to different keys would not make sense as the resultant graph would effectively remove the element of tonality. No piece in Bach's era would look like this, because to average across all the keys implies that we have converged on a piece that is equally in all 12 keys, that is to say, converging upon a piece possessing no tonal center.}
The resulting graphs would represent the closest thing to our
expectation of what the keys would ideally look like under the construct
of the well-tempered system, devoid of added compositional
idiosyncrasies. These graphs are crucial in that they allow us to
appraise the significance of the frequency values of tempered versions
of intervals in actual fugues, in that we are able to evaluate whether
or not these values are anomalous compared against a norm.

    \subsection{Outline of Analysis
Chapters}\label{outline-of-analysis-chapters}

Having established the logic behind the organization of the analysis,
the following now provides an outline of the three analysis chapters,
summarizing briefly the types of analysis covered in each, as well as
their scope, significance, and goals. The outline in this section
functions as more of a broad overview, as at the beginning of each
individual chapter will contain an introductory section that will
provide a more detailed outline of the chapter's analysis, as well as a
more in-depth explanation of the goals and questions that chapter seeks
to answer.

\subsubsection{Analysis I: Global Analysis (First tier;
non-parameterized, general
analysis)}\label{analysis-i-global-analysis-first-tier-non-parameterized-general-analysis}

This chapter will explore the relationship between temperament and the
twelve minor fugues of the WTC I on the most general, macro level. This
will be achieved through looking at the distributions of all tempered
intervals for each minor fugue on both the horizontal and vertical
domain (i.e. melodic intervals, harmonic intervals) separately, with no
parameterizations---in other words, these graphs will account for
every single interval present in the WTC I minor fugues. This is the
maximum level in terms of general analysis, as there is no
parameterization, and therefore no chance for the introduction of
personal bias. Any trends that are detected between key and their
resulting temperament graphs in this section will inform subsequent
searches, as well as types of musical parameters to be set for the next
section.

These general distributions will be analyzed in two different
configurations:

\begin{enumerate}
\def\labelenumi{\arabic{enumi}.}
\tightlist
\item
  Temperament "fingerprint" plots:

  \begin{itemize}
  \tightlist
  \item
    These graphs will present tempered intervals as a function of cents
    for composition-independent key visualization.
  \item
    These temperament plots are nicknamed "fingerprint" plots because
    they seek to characterize the unique and different configuration of
    each of these keys given well-temperament.
  \item
    The purpose of this analysis section is to observe the structure of
    well-temperament given key, and to identify the way that keys relate
    to each other, any key groupings that the tuning system creates, and
    what kinds of tempered intervals correspond with which keys and key
    groups.
  \end{itemize}
\item
  KDE plots:

  \begin{itemize}
  \tightlist
  \item
    These graphs will present tempered intervals as a function of purity
    (measured in cents from just) as an easier/alternative method to
    visualize and compare keys in terms of the spectrum of pure and
    tempered intervals.
  \item
    Transposition plots applied to these KDE plots as a first bid at
    determining whether or not we can detect correlations between
    temperament and composition on a broad scale.
  \end{itemize}
\end{enumerate}

\textbf{Goals:}

\begin{itemize}
\tightlist
\item
  Characterize the keys
\item
  Characterize broad temperament/key structures
\item
  Identify any correlations between temperament and composition
\item
  Determine if correlative effects are stronger on one particular
  directional domain (vertical or horizontal) to better guide which
  domain, as well as intervals and contrapuntal elements, to focus more
  upon in future chapters
\item
  Identify possible important traits of keys/intervals to analyze in the
  next chapter
\end{itemize}

\subsubsection{Analysis II: Intermediate Analysis (Second tier;
functional, bridge
analysis)}\label{analysis-ii-intermediate-analysis-second-tier-functional-bridge-analysis}

This chapter will continue to look at the relationship between
temperament and composition, but based on the result from the previous
chapter will start to place some parameters on the types of musical
elements and intervals examined. The chapter is essentially a bridge
between the very general/statistical first chapter, and the
local/musically driven third chapter.

\begin{enumerate}
\def\labelenumi{\arabic{enumi}.}
\tightlist
\item
  Non-temperamental plots (duration and texture)

  \begin{itemize}
  \tightlist
  \item
    Examine note duration as a function of key
  \item
    Examine vocal thickness as a function of key
  \end{itemize}
\item
  Untempered intervals plots (perfect, imperfect intervals)

  \begin{itemize}
  \tightlist
  \item
    Examine distribution of untempered horizontal and vertical perfect
    and imperfect intervals as a function of key
  \end{itemize}
\item
  Tempered interval plots (perfect, imperfect intervals)

  \begin{itemize}
  \tightlist
  \item
    Examine distribution of tempered horizontal and vertical perfect and
    imperfect intervals as a function of key
  \end{itemize}
\item
  Scale degree plots (fifths, thirds, semitones)

  \begin{itemize}
  \tightlist
  \item
    More detailed characterization of keys based on their scale
    structures given temperament
  \item
    Examine frequency of semitone usages at certain scale degrees and
    its relationship to voice-leading
  \end{itemize}
\end{enumerate}

\textbf{Goals:}

\begin{itemize}
\tightlist
\item
  Identify correlations between certain general compositional elements
  and intervals and key to better characterize keys and key groups, and
  examine if well-temperament structures are sensitive to the historical
  tuning structures and compostional traits of meantone and Pythagorean.
\item
  Finer degree of key characterization through scale degree plots.
\item
  Identify correlations between key and functional harmonic structures
  (i.e. leading tone).
\item
  Formulate predictions based on scale degree graphs and their
  temperamental key schemes to determine certain intervals to be used in
  thematic contexts in the next chapter.
\end{itemize}

\subsubsection{Analysis III: Fugal and Local Analysis (Third tier;
musical/fugal/thematic
analysis)}\label{analysis-iii-fugal-and-local-analysis-third-tier-musicalfugalthematic-analysis}

This final chapter is centered on thematic analysis, and will focus on
how temperament interacts with specific, musical structures within the
fugues, particularly the Pythagorean fugues. We will focus on intervals
and interval structures within a thematic context---especially the
fugal subject---and analyze how these tempered intervals give rise
to the character of the subjects and the structure of their fugues, and
how this relates to our key expectations. The musical structures that we
will be analyzing include motivic structure and symbolism, vertical
structure, form, large scale thematic and textural architecture, and
approach to chromaticism and its implication upon dissonance/consonance
balance.

\begin{enumerate}
\def\labelenumi{\arabic{enumi}.}
\tightlist
\item
  Fugal/local analysis

  \begin{itemize}
  \tightlist
  \item
    Intervals/concepts to be examined:

    \begin{enumerate}
    \def\labelenumii{\arabic{enumii}.}
    \tightlist
    \item
      Fifths
    \item
      Semitones
    \item
      Ninths
    \item
      Enharmonic equivalency in intervals (diminished sevenths,
      diminished fourths, chromatic fourths)
    \end{enumerate}
  \end{itemize}
\item
  Summary of the Pythagorean fugues/keys

  \begin{itemize}
  \tightlist
  \item
    Re-synthesize the information based on key/fugue for clearer
    visualization of conclusions
  \end{itemize}
\end{enumerate}

\textbf{Goals:}

\begin{itemize}
\tightlist
\item
  Determine if temperament plays a significant role on a piece's
  thematic and musical structures through establishing clear connections
  between key and tempered intervals at the thematic/subject level.
\item
  Determine whether or not temperament is an important contributing
  dimension to a piece's aesthetic integrity and expressivity based on
  the how it controls the design of the fugal subject and the resulting
  fugal structure, and how much these elements complement a piece's key
  given our prior predictions/characterizations of key and key group
  from the rest of the dissertation.
\end{itemize}

    \section{The Computational Framework}\label{the-computational-framework}

    \subsection{\texorpdfstring{Joint \texttt{music21} and \texttt{pandas}
Framework for Score
Analysis}{Joint music21 and pandas Framework for Score Analysis}}\label{joint-music21-and-pandas-framework-for-score-analysis}

The workhorse for essentially the entire dissertation relies primarily
upon Python's computer-aided analytical toolbox \texttt{music21}, used
in conjunction with \texttt{pandas} Dataframes for storing intervals and
their respective musical and temperament-related information, and the
graphing utilities of plotting libraries \texttt{matplotlib} and
\texttt{seaborn} for visualization of the data and distributions.
\texttt{music21} was created by Michael Cuthbert and his research lab at
MIT in 2010 (Cuthbert, 2010), and is rapidly expanding, fast becoming
the lingua franca for computer-aided musicology. Detailed
\href{https://web.mit.edu/music21/}{documentation} on the toolbox can be
accessed on MIT's website.

Although \texttt{music21} already contains a rich store of various built
in search classes and functions that allow users to perform score
searches, namely the base class \texttt{music21.search.base}, which
contains functions that search scores based on rhythmic and melodic
material, as well as methods that make various musical transforms and
extraction of musical information (e.g. note name, pitch, intervallic
information) easily accessible, the current framework relies
predominantly upon Python's dictionary containers to store information.
While this method is perfectly valid and effective, it is more labor
intensive and can become tricky if more complex searches that are
constrained by multiple parameters are presented, as the process of
extracting information from nested dictionaries can become complicated.
Furthermore, new searches, or modifying a search, would require the
creation of a new dictionary for each individual instance, which would
eventually start to slow down the analysis process as searches become
more elaborate, or if one needed to toggle between searches of various
parameters or scopes of broadness.

One of the intermediate goals of this dissertation is to create a more
streamlined framework within \texttt{music21} for parsing musical scores
that simplifies and reduces the amount of coding needed to perform score
searches, as well as provide a more intuitive way of viewing the data in
chart form, and to then quickly be able to transform this data into
plots for visualization. I achieved this through creating Python
functions that perform interval searches using the score parsing
capabilities of \texttt{music21}, which output the data in the form of
\texttt{pandas} Dataframe containers. Finally, visualization is made
easy through the usage of \texttt{seaborn}, a visualization library that
is essentially a \texttt{matplotlib} wrapper.

The following sections will provide a brief but comprehensive
walkthrough of the Python modules that I have written that contain
within them all the variables, methods, and functions that I have used
to generate the intervallic searches, subject searches, \texttt{pandas}
Dataframes, and graphs that comprise the bulk of my dissertation's data
and analysis. I will start by explaining the first module,
\texttt{bach.py}, which contains all the functions responsible for
setting up and constructing the various modes of score
searches---vertical, horizontal, transposition, and
subject---each of which I will walk through the reasoning behind. I
will then proceed to break down the second module, \texttt{bach\_df.py},
which reads pickled (Python's \texttt{pickle} module for serializing and
de-serializing objects for storage) Dataframes into any Python
environment and assigns them to variables and lists ready to be analyzed
and put into graphs. In this section I will provide an overview to the
structure of and musical information contained in these \texttt{pandas}
Dataframes, and the logic behind the lists that I have organized them
into, as well as provide intuition to the operations that can be
performed on these dataframes to limit and define search parameters.
Finally, I will touch upon how \texttt{seaborn} is utilized in this
dissertation for its plotting and data visualization capabilities.

The information contained in this Methods chapter is focused on
providing an overview and discussion of the logic and intuition behind
the Python code written for the dissertation, and less so upon its
technicality. Examples of portions of code and functions will be
provided at an \emph{ad hoc} basis in the body of the dissertation, but
the full text of the .py modules and the Python source code is supplied
in the Appendix for further reference.

    \subsection{\texorpdfstring{The \texttt{bach.py} Search
Module}{The bach.py Search Module}}\label{the-bach.py-search-module}

    \begin{Verbatim}[commandchars=\\\{\}]
{\color{incolor}In [{\color{incolor}2}]:} \PY{k+kn}{from} \PY{n+nn}{bach} \PY{k+kn}{import} \PY{o}{*}
\end{Verbatim}

    All of the basic setup and functions written for the analysis portion of
this dissertation are contained within the Python module,
\texttt{bach.py}. Importing it into a Python environment will also
initiate an import of Python packages \texttt{music21}, \texttt{numpy},
\texttt{pandas}, plotting library \texttt{pyplot} from
\texttt{matplotlib}, and \texttt{seaborn}, as well as convert and load
the Well-Tempered Clavier scores in the current environment, all of
which are necessary for the performance of the search functions and
analyses. The structure of the module (which I will be covering in
respective order) is as follows:

\begin{enumerate}
\def\labelenumi{\arabic{enumi}.}
\tightlist
\item
  Import Python packages necessary to run module's functions.
\item
  Convert Humdrum \texttt{**kern} files of all fugues from WTC book I
  and II into \texttt{music21} scores (\texttt{music21.stream.Score}
  objects), and organize into lists based on books, and Major/minor
  mode.
\item
  Define intermediate functions necessary for the set up to run search
  functions.
\item
  Define function for subject search.
\item
  Define functions for vertical and horizontal searches, along with
  transpositions.
\item
  Define fucntions for searching based on metrical hierarchy (horizontal
  search only), as this cannot be achieved through manipulating the
  Dataframes as it involves calculating intervallic distances between
  non-adjacent notes.
\end{enumerate}

\subsubsection{\texorpdfstring{Converting Well-Tempered Clavier Fugues
into \texttt{music21} Score
Objects}{Converting Well-Tempered Clavier Fugues into music21 Score Objects}}\label{converting-well-tempered-clavier-fugues-into-music21-score-objects}

Streams are \texttt{music21}'s main storage container for musical
elements (class \texttt{music21.stream}), which contain musical scores
as one of its subclasses (\texttt{music21.stream.Score}). Any musical
score that is converted into a \texttt{music21} Score object can be
parsed for its musical elements such as note name, pitch and temporal
information, location within the score (beat, offset, measure number,
part), and can also be subject to musical transformations such as
transpositions and re-tuning. \texttt{music21} has built into it a core
corpus of thousands of musical scores, which can be called readily into
a Python environment with the simple command
\texttt{music21.corpus.parse(\textquotesingle{}filename\textquotesingle{})};
however, the Well-Tempered Clavier is not currently contained in the
core corpus. They are, however, available in Humdrum's
\href{http://kern.ccarh.org/}{score database} in the form of
\texttt{**kern} files, which are converted into \texttt{music21} Score
objects through the command \texttt{music21.converter.parse()}, and the
source of all my fugue score files for analysis in this dissertation.

The fugue scores are presented in full-score, or choral format, with
each part on a separate staff, and the staffs organized by vocal
tessitura, instead of by the traditional keyboard grand staff. The
different voices are each assigned their own \texttt{music21} part ID,
indicated by spine number (lower numbers corresponding to lower vocal
parts), the title inherited from the scores' Humdrum origins. This makes
breaking the score down into its individual parts, and isolating parts
from the score's entirety, a straightforward task, especially useful for
the horizontal search functions.

Here is an example of how easily \texttt{music21} can call up the c
minor fugue from WTC I, and display the first 6 measures.


    \begin{Verbatim}[commandchars=\\\{\}]
{\color{incolor}In [{\color{incolor}95}]:} \PY{n}{c\PYZus{}1}\PY{o}{.}\PY{n}{measures}\PY{p}{(}\PY{l+m+mi}{1}\PY{p}{,}\PY{l+m+mi}{6}\PY{p}{)}\PY{o}{.}\PY{n}{show}\PY{p}{(}\PY{p}{)}
\end{Verbatim}

\begin{Example}[H]
\vspace{1.5em}
    \centering
    \adjustimage{max size={0.9\linewidth}{0.9\paperheight}}{methods_files/methods_13_0.png}
    \caption{ C minor Fugue Opening (mm. 1-6). }
\end{Example}    
    And here we can isolate and examine the mezzo voice alone:


    \begin{Verbatim}[commandchars=\\\{\}]
{\color{incolor}In [{\color{incolor}35}]:} \PY{n}{c\PYZus{}1}\PY{o}{.}\PY{n}{parts}\PY{p}{[}\PY{l+m+mi}{1}\PY{p}{]}\PY{o}{.}\PY{n}{measures}\PY{p}{(}\PY{l+m+mi}{1}\PY{p}{,}\PY{l+m+mi}{3}\PY{p}{)}\PY{o}{.}\PY{n}{show}\PY{p}{(}\PY{p}{)}
\end{Verbatim}

\begin{Example}[H]
\vspace{1.5em}
    \centering
    \adjustimage{max size={0.9\linewidth}{0.9\paperheight}}{methods_files/methods_16_0.png}
    \caption{ C minor Fugue Mezzo Voice (mm. 1-3). }
\end{Example}    
    The module loads all 48 fugues (24 from WTC I and 24 from II) and saves
them as variables corresponding to their key (capitalized for major
mode, lower-case for minor mode) and book they are contained in (e.g.
\texttt{C\_1} for the C major fugue from book I, \texttt{bfl\_2} for the
b-flat minor fugue from book II. To avoid using special characters in
these variables names, sharps are represented by 'sh', and flats by
'fl'.) These are then stored in lists that are organized based on book,
major/minor mode, and fugues utilizing pure/Pythagorean thirds. These
lists divide up the data into groups of fugues that share similar and
characteristics, which are necessary for control in terms of isolating
the data to avoid confounding with unwanted variables, as well as
analyzing for relevant relationships and correlations between certain
musical factors. The lists provided are as follows:

\begin{itemize}
\tightlist
\item
  \texttt{wtc\_1}: Python list containing 24 score objects of fugues
  from book I in chromatic order starting with C major
\item
  \texttt{wtc\_2}: Python list containing 24 score objects of fugues
  from book II in chromatic order starting with C major.
\item
  \texttt{wtc\_1\_M}: Python list containing 12 score objects of fugues
  from book I in major mode
\item
  \texttt{wtc\_1\_m}: Python list containing 12 score objects of fugues
  from book I in minor mode
\item
  \texttt{wtc\_1\_m\_pure}: Python list containing 3 score objects of
  the minor fugues from book I that contain pure thirds (e minor, a
  minor, d minor)
\item
  \texttt{wtc\_1\_m\_pythag}: Python list containing 3 score objects of
  the minor fugues from book I that contain Pythagorean thirds (f minor,
  b-flat minor, d-sharp minor)
\end{itemize}

Adding the string \texttt{\textquotesingle{}\_names\textquotesingle{}}
after any of these list names will return a list of the names of the
fugues contained in each in the form of a string; this is mainly for use
in titling charts and plots.

\subsubsection{Intermediate Functions}\label{intermediate-functions}

After all 48 fugues are converted into \texttt{music21} scores,
preliminary functions are provided--most of which are short and serve
a technical purpose--some to be used to extract specific types of
musical information from the score as one of the fields in the search
functions, and others to be used for temperament visualization in the
\texttt{seaborn} plots. Most of the fields in the vertical and
horizontal Dataframes return information that is already directly
extracted by one line of code, generally through \texttt{music21}'s
built in function and attributes, such as note name, pitch, or
durational information; the functions supplied in this section serve the
purpose of extracting information that requires more lines or some form
of simplification. The following is a partial list of the functions with
more direct musical implication, with a brief description of these
functions:

\begin{itemize}
\tightlist
\item
  \texttt{circle\_sort(df)}: takes input argument of list of objects
  (written with lists of scores or dataframes in mind, but could be
  anything from numerical values or strings) that are originally
  organized by ascending chromatic step covering all chromatic pitches
  (c, c\#, d ... b), and returns list reorganized according the circle
  of fifths (c, g, d ... f).
\item
  \texttt{cents\_from\_just(n1,\ n2)}: takes input argument of two
  notes, calculates how far the interval created by those two notes are
  from its just counterpart, returns difference value in cents
\item
  \texttt{bound\_trans(value,max\_new=1,min\_new=.2,max\_old=22,min\_old=0)}:
  linear boundary transformation function using minimum-maximum
  normalization equation to map an interval's absolute cents from just
  value (0 to 22) onto a scale from 0 to 1. This will be used to assign
  the alpha (transparency) values of bars in \texttt{seaborn} bar charts
  to visualize the effects of temperament. Values closer to just will
  show up as lighter bars, and values farther from just will yield
  darker bars.
\item
  \texttt{flat\_sharp(n)}: (to be used in conjunction with
  \texttt{cents\_from\_just()} function) - takes input argument of value
  of cents from just, and returns whether or not the value is just,
  flatter than just, or sharper than just
\item
  \texttt{get\_spine(n)}: takes input argument of musical element (note,
  rest, chord), and returns numerical value corresponding to which
  voice/part the particular note belongs to. Lowest voice = 0, and
  higher numbers correspond to voices of higher tessitura. If musical
  element is not contained within a specific part (as is the case with
  closing measures for certain fugues, in which Bach takes creative
  liberty to flesh out the voices by adding extra notes), part yiels the
  value of -9, an arbitrarily assigned value to make computations
  simpler.
\item
  \texttt{get\_tie(n)}: takes input argument of note object and returns
  information as to whether or not the note is part of a tied group of
  notes or not. If it is part of a tied group, the tie type is given
  (start, stop, continuing). This is primarily used in the vertical
  searches, as when the score is verticalized through \texttt{music21}'s
  \texttt{chordify()} function, sustained notes are broken up if notes
  in other voices are moving at a faster rate; \texttt{get\_tie()} can
  provide information as to whether the note in question is sustained
  from a previous beat, or presented at that particular offset of time.
\end{itemize}

\subsubsection{Subject Search Functions}\label{subject-search-functions}

\paragraph{The Importance of the Subject Search and Isolation of Fugal
Elements}\label{the-importance-of-the-subject-search-and-isolation-of-fugal-elements}

The subject search is an important and integral part of my analysis, as
the subject is the primary motivic building block of the fugue. In some
ways, the subject can be seen as the kernel of the fugue, as in this one
line is succinctly contained the \emph{Geist} of the piece, and from it
other musical material unfolds. Characteristic of works of the Baroque
era, indeed even extending beyond the musical and into the general
artistic sphere, the sense of polyphonic architecture and organized
structure that arises from a strict adherence to thematic development
and unity of the fugal genre makes a well-executed fugue a magnificent
organism to behold, and Bach arguably brings the art form to the
pinnacle of aesthetic expression. It is a beautiful exercise in
duality---the simultaneous independence and dependence of both
horizontal and vertical domains in regards to one another, a thematic
schema so obsessive and persistent, yet freely evocative in its
underlying emotion---that makes the fugue such an ideal candidate
for temperament analysis, as this built-in set of compositional rules
allow for a straightforward method of parsing for structure and thematic
hierarchy, and the isolation of salient musical events.

Thus, an analysis that sets apart and focuses on fugal elements, the
most prominent of these being the subject material, can aid greatly in
addressing the questions of the effects temperament, and tempered
intervals, exert upon the overall musical and motivic structure of a
piece, as it offers a concrete way of filtering and parsing out salient
material that we have reasonable motivation to believe was selected by
deliberate choice, and assigning a hierarchy of importance to intervals
in a score that is filled with them. On this vein, mining not only for
strict subject statements, but other fugal elements such as motivic
material based from partial subject fragments, recurring episodic
motifs, and countersubjects and countersubject motivic material, can be
of notable value as this will grant access to a finer graduation of
levels of motivic hierarchical structure.

I have only provided this first level of searching for motivic
importance through only parsing for subject material, through the means
of the \texttt{get\_subjects()} subject search. However, because the
purpose of the subject search is primarily determining and setting aside
certain intervals and passages of greater motivic importance, what I am
selecting as "subject material" is somewhat looser than just exact
iterations of subject statements. In general, I have included subject
material and subject "family" material--partial subject statements as
well in the search, as well as, in certain instances, motivic material
that utilizes subject fragments, the main deciding factor being the
prominence and frequency of such musical material, and its overall
weight of presence in a given piece. In this dissertation, all such
material is classified under "subject material", and not assigned any
further ranking of importance. For future elaborations and work
furthering the endeavors of this project, great benefit could be gained
from drawing these finer grained distinctions between subject,
countersubject, subject fragments, and episodic material, but the coding
involved for such a division is time intensive, and for the scope of
this project, just the subject material is sufficient to start with,
which is what I shall be working with.

\paragraph{\texorpdfstring{\texttt{get\_subjects()} and
\texttt{in\_subject()} Subject Search
Functions}{get\_subjects() and in\_subject() Subject Search Functions}}\label{get_subjects-and-in_subject-subject-search-functions}

The subject search function \texttt{get\_subjects()} is essentially as
its name suggests: it searches a given fugue score for subject material
and returns a list of these subjects in stream segments. The subjects
are extracted from a fugue's score using either the note name algorithm
or rhythm algorithm in \texttt{music21}'s stream searcher function
\texttt{music21.search.StreamSearcher()}. The subject, in the form of a
search list (a stream, or a list of notes), along with the score the
search is to be performed upon, is provided to the function, and
depending on which algorithm is applied, the stream searcher function
will filter out tones and musical elements in the score that either
match the notes, or the rhythmic pattern of the subject search list, and
return the matches as streams themselves.

The stream searcher is very effective if either rhythmic pattern, or
pitch patterns matches up directly with the given search list, but
because the algorithm looks for exact matches, it does not yield results
even if variation is very slight, and cannot detect for musical
transformations such as pitch transposition, or rhythmic
augmentation/diminution. Notes and elements within the search list can
be omitted from the search algorithm by assigning it to a Wildcard (or
Wildcard duration for a rhythmic search), meaning that the algorithm
skips over the specific properties of that note when running the search
match.

For my fugue subject searches, I mainly relied upon the rhythm algorithm
to search for subject material, as rhythmic pattern is the more fixed
and reliable variable across the fugue subjects, as a simple modulation
would break a search relying upon note name. Wildcards were also
frequently applied to the beginning and ending notes of a subject search
list, as the beginnings and endings of a subject statement frequently
undergo variations in order to integrate subject matter smoothly into
the rest of the piece. Certain fugues only required one subject search
list to successfully gather all statements of the subject, but some of
the more complex fugues--and for the matter, double/triple
fugues--required multiple subject search lists to account for rhythmic
variation across subject statements, and rhythmic transformations such
as diminution and augmentation, and partial subject statements.

The \texttt{get\_subjects()} function takes in an input argument of a
fugue score variable, and yields a list of that particular fugue's
subjects in the form of analyzable \texttt{music21} streams. At this
moment, only fugues from book I are searchable using the
\texttt{get\_subjects()} function, since my analysis focuses on fugues
from book I, and the subjects from book II have not been encoded yet.
The following is a list of subject statements from the C minor fugue in
Book I, obtained through the \texttt{get\_subjects()} function. Because
the function returns only the notes within these subjects, the metrical
representation of the subject statements may look a little funny through
the .show() function, but the pitch and temporal values are all
accurate.

    \begin{Verbatim}[commandchars=\\\{\}]
{\color{incolor}In [{\color{incolor}3}]:} \PY{c+c1}{\PYZsh{}all subject statements with starting measure numbers for the c }
        \PY{c+c1}{\PYZsh{}minor fugue, Book I }
        \PY{p}{[}\PY{p}{(}\PY{n+nb}{print}\PY{p}{(}\PY{n}{i}\PY{p}{[}\PY{l+m+mi}{0}\PY{p}{]}\PY{o}{.}\PY{n}{measureNumber}\PY{p}{)}\PY{p}{,}\PY{n}{i}\PY{o}{.}\PY{n}{show}\PY{p}{(}\PY{p}{)}\PY{p}{)} \PY{k}{for} \PY{n}{i} \PY{o+ow}{in} \PYZbs{}
         \PY{n}{flatten\PYZus{}list}\PY{p}{(}\PY{n}{get\PYZus{}subjects}\PY{p}{(}\PY{n}{c\PYZus{}1}\PY{p}{)}\PY{p}{,}\PY{p}{[}\PY{p}{]}\PY{p}{)}\PY{p}{]}\PY{p}{;}
\end{Verbatim}

    \begin{Verbatim}[commandchars=\\\{\}]
3
    \end{Verbatim}

    \begin{center}
    \adjustimage{max size={0.9\linewidth}{0.9\paperheight}}{methods_files/methods_18_1.png}
    \end{center}
    
    \begin{Verbatim}[commandchars=\\\{\}]
11
    \end{Verbatim}

    \begin{center}
    \adjustimage{max size={0.9\linewidth}{0.9\paperheight}}{methods_files/methods_18_3.png}
    \end{center}
    
    \begin{Verbatim}[commandchars=\\\{\}]
20
    \end{Verbatim}

    \begin{center}
    \adjustimage{max size={0.9\linewidth}{0.9\paperheight}}{methods_files/methods_18_5.png}
    \end{center}
    
    \begin{Verbatim}[commandchars=\\\{\}]
29
    \end{Verbatim}

    \begin{center}
    \adjustimage{max size={0.9\linewidth}{0.9\paperheight}}{methods_files/methods_18_7.png}
    \end{center}
    
    \begin{Verbatim}[commandchars=\\\{\}]
1
    \end{Verbatim}

    \begin{center}
    \adjustimage{max size={0.9\linewidth}{0.9\paperheight}}{methods_files/methods_18_9.png}
    \end{center}
    
    \begin{Verbatim}[commandchars=\\\{\}]
15
    \end{Verbatim}

    \begin{center}
    \adjustimage{max size={0.9\linewidth}{0.9\paperheight}}{methods_files/methods_18_11.png}
    \end{center}
    
    \begin{Verbatim}[commandchars=\\\{\}]
7
    \end{Verbatim}

    \begin{center}
    \adjustimage{max size={0.9\linewidth}{0.9\paperheight}}{methods_files/methods_18_13.png}
    \end{center}
    
    \begin{Verbatim}[commandchars=\\\{\}]
26
    \end{Verbatim}

    \begin{center}
    \adjustimage{max size={0.9\linewidth}{0.9\paperheight}}{methods_files/methods_18_15.png}
    \end{center}
    
    Similar to the \texttt{get\_subjects()} function, the
\texttt{in\_subject()} function also performs a subject search, but
instead of yielding a list of subject streams, it gives information as
to whether or not a particular note or interval within a fugue belongs
to the fugue's subject material. For the original \texttt{in\_subject()}
function, it takes as input argument of a fugue score, and an additional
argument of notes, and returns information as to whether or not those
particular notes belong to the subject material from that given fugue,
and if so, information as to their exact placement (as measured through
\texttt{music21}'s note offset attribute) within the subject in the
fugue. Modified versions of the \texttt{in\_subject()} function are used
within the vertical and horizontal search functions to make subject
searches more streamlined, but their function is virtually the same as
their parent function.

\subsubsection{\texorpdfstring{The Vertical Interval Search Function
\texttt{vert\_full()}}{The Vertical Interval Search Function vert\_full()}}\label{the-vertical-interval-search-function-vert_full}


    \begin{Verbatim}[commandchars=\\\{\}]
{\color{incolor}In [{\color{incolor}86}]:} \PY{c+c1}{\PYZsh{} A harmonic fifth on the vertical dimension}
         \PY{n}{chord}\PY{o}{.}\PY{n}{Chord}\PY{p}{(}\PY{p}{[}\PY{n}{note}\PY{o}{.}\PY{n}{Note}\PY{p}{(}\PY{l+s+s1}{\PYZsq{}}\PY{l+s+s1}{C4}\PY{l+s+s1}{\PYZsq{}}\PY{p}{)}\PY{p}{,}\PY{n}{note}\PY{o}{.}\PY{n}{Note}\PY{p}{(}\PY{l+s+s1}{\PYZsq{}}\PY{l+s+s1}{G4}\PY{l+s+s1}{\PYZsq{}}\PY{p}{)}\PY{p}{]}\PY{p}{)}\PY{o}{.}\PY{n}{show}\PY{p}{(}\PY{p}{)}
\end{Verbatim}

\begin{Example}[H]
\vspace{1.5em}
    \centering
    \adjustimage{max size={0.9\linewidth}{0.9\paperheight}}{methods_files/methods_21_0.png}
    \caption{ A harmonic fifth on the vertical dimension. }
\end{Example}    
    The two most important functions in \texttt{bach.py} are the vertical
search function, and the horizontal search function, as the main bulk of
analysis for the dissertation relies upon these searches to find
intervals and their respective musical attributes and information. While
the composite musical experience is undoubtedly a seamless interaction
between the vertical and horizontal domains, especially in the case of a
fugue, where the fabric of musical vertical and horizontal structure
coexists in a dance of simultaneous independence and co-dependence, for
the purpose of simplifying and producing a more direct path from input
variable to resulting effect, it is necessary to break apart the two
domains for individual analysis first before moving on to discuss a more
complex interaction between the two structures.

It is important to note that, because of the intricate contrapuntal
nature of the fugue, in many instances it is nonsensical to completely
attribute the larger scale effect of an interval to just a single
isolated domain--take a suspension for example: while the existence of
a consonant vertical interval may physically live in an event
characterized by a singular moment, its sum effect is much larger than
that moment in time itself, but rather contextualized and given meaning
through the dissonant interval that just preceded it. Indeed, the two
dimensions, although being separate, are inextricably interlocked; as
Scriabin mused, "Melody is unfurled harmony" (Bowers, 1969), and the
effect of all of music is this unfolding of consonance and dissonance
over the domain of time, which is the interaction of the vertical
dimension across the horizontal. Hence, no single structure in its
isolated form can be completely responsible for an interval's effect in
its entirety, and just because an interval exists physically in one
domain does not preclude its influence on the other. Nevertheless, still
much can be gained and the effect approximated from separating the two
domains for analytical purposes.

The vertical search, \texttt{vert\_full()} analyzes and catalogues into
dataframes intervals presented harmonically in the vertical dimension;
it takes in the input argument of a single fugue, and returns every
instance of a vertical interval within the piece, along with relevant
musical information; the stucture of these dataframes, and the
information they contain will be discussed in greater detail shortly in
the proceeding section on the second Python module for this
dissertation, \texttt{bach\_df.py}. The following is a very brief
walkthrough of the construction of the vertical search function; the
horizontal search very much follows the same structure, but the vertical
one has extra added steps of converting the score into chords, and break
each chord down into matrices that house every combination of pairwise
notes contained in the chord.

\begin{enumerate}
\def\labelenumi{\arabic{enumi}.}
\tightlist
\item
  Selected fugue score is provided as the input variable; file is copied
  via deepcopy to insure against modification of the original file. File
  is analyzed for key with \texttt{music21}'s Krumhansl-Schmuckler's key
  finding algorithm to be logged in one of the dataframe's fields.
\item
  Score is retuned to Werckmeister III well-temperament using Scala
  archive of scales. Scores in the \texttt{music21} corpus are
  automatically tuned to equal temperament, but retuning to different
  temperaments and scales in \texttt{music21} is a simple process using
  \href{http://www.huygens-fokker.org/docs/scalesdir.txt}{Scala's scale
  archive} and \texttt{music21}'s tuning function (e.g. select
  Werckmeister III scale and assign to variable \texttt{t} -
  \texttt{t=scale.ScalaScale(str(\textquotesingle{}werck3\textquotesingle{}))};
  retune fugue score \texttt{f} to Werckmeister III \texttt{t.tune(f)}).
  Scala's archive currently contains more than 4,200 scales, including
  popular well and meantone temperaments such as Kirnberger
  ('kirnberger'), Pietro Aaron's 1/4 comma meantone ('meanquar'), and
  Salinas' 1/6 comma meantone ('meansixth') .
\item
  Fugue's subjects are obtained using function \texttt{get\_subjects()},
  and ID, score offset (placement within the general score), and subject
  offset (placement within the subject statement) of all notes found
  stored into a \texttt{numpy} matrix. This will be used as a database
  for the search function \texttt{in\_subject()}, to determine whether
  or not a given note in a fugue comes from subject material through
  matching ID and score offset.
\item
  Score is broken down into individual parts, and a new score created
  for every pairwise combination of parts (e.g. for a three voice fugue
  with Tenor, Mezzo, and Soprano voices, three scores are created,
  Tenor-Mezzo, Mezzo-Soprano, Tenor-Soprano) and each score's two parts
  melded through "verticalizing" the score using the \texttt{chordify()}
  function.
\item
  Through the process of item 4 above, every pairwise vertical interval
  in the piece is represented in the form of two-note chords contained
  in these chordified, two-voice subscores. Through a series of loops,
  each two-note chord, or interval, is analyzed for several musical
  qualities (e.g. cents of the interval, the interval represented in
  terms of semitones, duration of interval - this will all be detailed
  and expounded upon in the next section covering the dataframes), and
  stored into a \texttt{numpy} matrix.
\item
  Matrix is converted into a multi-columned \texttt{pandas} Dataframe
  container which is the output of the function, and essentially
  characterize the entire fugue in terms of vertical intervallic
  information. The intervals are stored as row entries, and musical
  information attached to the interval housed in the columns.
\end{enumerate}

One thing to note for the way the vertical search function was written
is that there is a possibly simpler, more straightforward way to
structure the function, that is, steps 4 and 5 can be simplified by
directly chordifying the entire score and then taking pairwise
combinations of notes at the chordal level, instead of at the score
level. This was actually the initial way I had approached coding the
function, and I still left that segment of code in the format of the
original set up (rendering the syntax in those portions a bit
unnecessarily awkward). Although both methods of coding achieve
basically the same results in terms of intervals collected, because
\texttt{chordify()} breaks apart a chord by the shortest note value
contained within it, and longer values are tied over from chord to
chord, and the accuracy of the representation of durations of notes in
the vertical domain is greater confounded as voices increase. Although
it is still difficult to get an precise measurement of durations on the
vertical domain, through breaking apart the score into subscores of two
voices first, and tracking tied values, a more accurate representation
can be achieved.

\subsubsection{\texorpdfstring{The Horizontal Interval Search
\texttt{hz\_full()}}{The Horizontal Interval Search hz\_full()}}\label{the-horizontal-interval-search-hz_full}


    \begin{Verbatim}[commandchars=\\\{\}]
{\color{incolor}In [{\color{incolor}203}]:} \PY{c+c1}{\PYZsh{} A melodic fifth on the horizontal dimension}
          \PY{n}{st}\PY{o}{=}\PY{n}{stream}\PY{o}{.}\PY{n}{Stream}\PY{p}{(}\PY{p}{)}\PY{p}{;} \PY{n}{st}\PY{o}{.}\PY{n}{append}\PY{p}{(}\PY{p}{[}\PY{n}{note}\PY{o}{.}\PY{n}{Note}\PY{p}{(}\PY{l+s+s1}{\PYZsq{}}\PY{l+s+s1}{C}\PY{l+s+s1}{\PYZsq{}}\PY{p}{)}\PY{p}{,}\PY{n}{note}\PY{o}{.}\PY{n}{Note}\PY{p}{(}\PY{l+s+s1}{\PYZsq{}}\PY{l+s+s1}{G}\PY{l+s+s1}{\PYZsq{}}\PY{p}{)}\PY{p}{]}\PY{p}{)}
          \PY{n}{st}\PY{o}{.}\PY{n}{show}\PY{p}{(}\PY{p}{)}
\end{Verbatim}

\begin{Example}[H]
\vspace{1.5em}
    \centering
    \adjustimage{max size={0.9\linewidth}{0.9\paperheight}}{methods_files/methods_24_0.png}
    \caption{ A melodic fifth on the horizontal dimension. }
\end{Example}    
    While the vertical search function \texttt{vert\_full()} looks at
intervals present harmonically in the vertical dimension through
chordifying the score, the horizontal search function,
\texttt{hz\_full()}, analyzes intervals presented melodically on the
horizontal dimension through breaking down the fugal score into its
individual melodic parts, and categorizing every adjacent melodic
interval presented in each part. The set up of the search pretty much
follows the same structure as the vertical search function, except in
the case of the horizontal search, the need to chordify the score is
obviously rendered unnecessary. Instead, in place of step 4 in the
vertical section, the score is broken down into a list of its individual
parts, and each adjacent melodic interval pair ({[}note(0),note(1){]},
{[}note(1),note(2){]}...{[}note(n-1),note(n){]}) in every part is
analyzed and documented in the dataframe.

Out of the two dimensions, vertical and horizontal, the process of
coding the function for the vertical was undoubtedly more complex and
involved, and consequently yields results that are more general than the
horizontal searches, and may be more susceptible to containing unwanted
artifacts. Because the intervals in the horizontal domain are obtained
through a more intuitive process of breaking down the score into its
individual parts, and looking at pairwise intervals, these isolated
horizontal intervals are presumably closer to our own auditory
experience and mental representation of the intervals. Contrasting this,
most moments in the vertical domain consist of several simultaneous
intervals present, and the perception of any given interval in a chord
is greatly dependent on the chord's function within a larger harmonic
context (e.g. the upper minor third present in a major triad is going to
have a considerably different effect than the same minor third outlined
by the lower half of a minor triad), and any relation or effect (or lack
of relation, for the matter) observed in the vertical domain must take
in consideration these factors.

This is not to say that the vertical domain is any less important, or
influential, than the horizontal domain, but it may very well be that
because intervallic structure in the vertical domain is more integrated,
and may require additional parameterization before more specific
analyses can be explored on that dimension. So while a large portion of
the more detailed analysis in this dissertation may focus on intervals
in the horizontal domain (another motivating reason being the underlying
linear nature of Bach's writing), the vertical analysis is still
integral in that it remains a key part of our resulting musical
experience.

\subsubsection{\texorpdfstring{Transposition Functions
\texttt{vert\_t3()} and
\texttt{hz\_t()}}{Transposition Functions vert\_t3() and hz\_t()}}\label{transposition-functions-vert_t3-and-hz_t}

The last type of function included in the \texttt{bach.py} module, and
one of the main tests for validity of the correlations and relationships
that are found within the scores using the vertical and horizontal
searches. These transposition functions, one for the vertical and one
for the horizontal, are essentially the same function as their parent
counterparts; the only difference is that the transposition functions
take in one extra input argument of a key, transposes the entire score
to that given key (granted the modes match), and proceeds with running
the interval searches and analysis on the newly transposed fugue.


    \begin{Verbatim}[commandchars=\\\{\}]
{\color{incolor}In [{\color{incolor}219}]:} \PY{c+c1}{\PYZsh{}the c minor fugue (Book 1, mezzo voice) tranposed to a minor}
          \PY{n}{c\PYZus{}1}\PY{o}{.}\PY{n}{parts}\PY{p}{[}\PY{l+m+mi}{1}\PY{p}{]}\PY{o}{.}\PY{n}{measures}\PY{p}{(}\PY{l+m+mi}{1}\PY{p}{,}\PY{l+m+mi}{3}\PY{p}{)}\PY{o}{.}\PY{n}{transpose}\PY{p}{(}\PY{l+s+s1}{\PYZsq{}}\PY{l+s+s1}{m\PYZhy{}3}\PY{l+s+s1}{\PYZsq{}}\PY{p}{)}\PY{o}{.}\PY{n}{show}\PY{p}{(}\PY{p}{)}
\end{Verbatim}

\begin{Example}[H]
\vspace{1.5em}
    \centering
    \adjustimage{max size={0.9\linewidth}{0.9\paperheight}}{methods_files/methods_27_0.png}
    \caption{ C minor fugue transposed to a minor (mm. 1-3). }
\end{Example}    
    The power behind the function of transposition is that it is a built-in
system to test the validity of correlations; it checks whether or not a
trend or distribution observed for a specific fugues is truly unique to
the specific key's tempered configuration. If transposing the fugue to
another key yields identical or similar distributions, the correlative
power between specific key and observed effect is greatly undermined.
The detection of a correlative distribution in a certain key may make it
tempting for one to draw a conclusion in terms of correlation between
key and effect, but if these trends and distributions are preserved
across transposition, we have far weaker a reason to believe that a
certain effect is a function of temperament and key, but instead
possibly due to a separate musical element that is not sensitive to
transposition.

    \subsection{\texorpdfstring{The \texttt{bach\_df.py} Data
Module}{The bach\_df.py Data Module}}\label{the-bach_df.py-data-module}

    \begin{Verbatim}[commandchars=\\\{\}]
{\color{incolor}In [{\color{incolor}2}]:} \PY{k+kn}{from} \PY{n+nn}{bach\PYZus{}df} \PY{k+kn}{import} \PY{o}{*}
\end{Verbatim}

    The \texttt{bach\_df} data module is the second of the analytical
modules written for my dissertation; the first, \texttt{bach.py} serves
as a search module, and the second gathers and organizes all the created
data in their final \texttt{pandas} Dataframe format, and imports them
into any Python environment ready for analysis. This section will go
over the storage and extraction process ("pickle"-ing) of the dataframe
objects created by the vertical and horizontal searches, provide a
walkthrough of the structure of the dataframes themselves, as well as
the organization and logic behind the various lists that are organized
within, and finally a brief tutorial on how these dataframes can be
searched within and manipulate to return values constrained by specific
parameters, and how the data can be visualized in \texttt{seaborn}.

\subsubsection{\texorpdfstring{Python \texttt{pickle} Files to Store
Dataframe
Objects}{Python pickle Files to Store Dataframe Objects}}\label{python-pickle-files-to-store-dataframe-objects}

There is an intermediate module that is not provided in my dissertation,
which is a module that physically runs all of the vertical and
horizontal search functions on all the fugues and stores the resulting
Dataframes into .pkl files (Python's built-in \texttt{pickle} module for
serializing and de-serializing objects, making it possible to save and
store objects in .pkl format, and quickly read them back into a Python
environment). The reason the module is not provided is that it is very
simple--just a few lines of code basically running the vertical and
horizontal search functions found in \texttt{bach.py}, and executing
\texttt{pandas}' \texttt{to\_pickle()} function to store the Dataframe
objects into pickle files--but depending on an individual's computer's
processing speed and capabilities, this process of running the vertical
and horizontal searches for each fugue and their transpositions may take
a good deal of time, so for the sake of this dissertation I have elected
to pre-run everything--a horizontal and vertical dataframe for every
fugue, and all its transpositions--and provide the already created
pickle files in a zipped folder entitled "back\_pickle\_df". Once this
file is downloaded and unzipped, the \texttt{bach\_df.py} module calls
\texttt{pandas}' \texttt{read\_pickle()} function to read the files back
into accessible Dataframes.

\subsubsection{\texorpdfstring{\texttt{bach\_df.py} to Read and Import
Dataframes}{bach\_df.py to Read and Import Dataframes}}\label{bach_df.py-to-read-and-import-dataframes}

The folder "bach\_pickle\_df" contains 624 dataframes stored as .pkl
files corresponding to all the fugues from Book I and all their
respective transpositions into every key. Each of the 24 fugues from
Book I is represented by 26 different dataframes: one for the vertical
search in its original key, one for the horizontal search in its
original key, 12 for the transposition of the fugue into each different
key (vertical), and 12 more for the transpositions for the horizontal
search. The \texttt{bach\_df.py} module reconverts all of these .pkl
files into their original Dataframe format an assigns them each to a
unique variable name, codified by their fugue name variable (e.g.
\texttt{c\_1}, \texttt{Afl\_1}) and specification as to vertical or
horizontal search, signified by "vt" for vertical, "hz" for horizontal
(e.g. \texttt{c\_1\_hz} for the dataframe containing all horizontal
intervals for the c minor fugue, \texttt{Afl\_1\_vt} for the dataframe
containing all vertical intervals for the A-flat major fugue).
Transposition fugues are indicated by a string of text between the fugue
name and the type (hz, vt) expressed by the string 'to\_\{key\}' (e.g.
\texttt{c\_1\_to\_fsh\_hz} for the horizontal search dataframe for the c
minor fugue transposed to f-sharp minor).

The following is a quick example of the horizontal dataframe of the C
minor fugue (just the first four columns shown for the purpose of this
example), and then the horizontal dataframe of the C minor fugue
transposed to F-sharp minor. Note how the intervallic information
remains the same, as expected from a transposition, but what does change
are the note names (to be expected with a transposition regardless of
tempering system), and furthermore, the values contained in the cents,
and cents from just columns, as to be expected with a change in key in
well-temperament. If the fugue scores had not been tempered, we would
see identical values in these particular cent columns in both
dataframes, as by definition the cent value of equal-tempered intervals
are transposition invariant, and remain identical across every key.

    \begin{Verbatim}[commandchars=\\\{\}]
{\color{incolor}In [{\color{incolor}226}]:} \PY{n}{c\PYZus{}1\PYZus{}hz}\PY{o}{.}\PY{n}{head}\PY{p}{(}\PY{p}{)}
\end{Verbatim}
\begin{Verbatim}[commandchars=\\\{\}]
{\color{outcolor}Out[{\color{outcolor}226}]:} 
\end{Verbatim}
\begin{singlespace}
\begin{table}[H]
\centering
\tiny
\begin{tabular}{|lllrrrllll|}
\hline
\textbf{{}} & \textbf{file} & \textbf{     key} & \textbf{ cents} & \textbf{ cents from just} & \textbf{ semitones} & \textbf{name} & \textbf{direction} & \textbf{n1 name} & \textbf{n2 name }\\
\hline
0 &  c\_1 &  C minor &    108 &               -4 &          1 &   m2 &     False &       G &      F\# \\
1 &  c\_1 &  C minor &    108 &               -4 &          1 &   m2 &      True &      F\# &       G \\
2 &  c\_1 &  C minor &    696 &               -6 &          7 &   P5 &     False &       G &       C \\
3 &  c\_1 &  C minor &    294 &              -22 &          3 &   m3 &      True &       C &      E- \\
4 &  c\_1 &  C minor &    402 &               16 &          4 &   M3 &      True &      E- &       G \\
\hline
\end{tabular}
\caption{C minor fugue dataframe. (Selected Nine Columns) }
\end{table}
\normalsize
\end{singlespace}
    \begin{Verbatim}[commandchars=\\\{\}]
{\color{incolor}In [{\color{incolor}226}]:} \PY{n}{c\PYZus{}1\PYZus{}to\PYZus{}fsh\PYZus{}hz}\PY{o}{.}\PY{n}{head}\PY{p}{(}\PY{p}{)}
\end{Verbatim}
\begin{Verbatim}[commandchars=\\\{\}]
{\color{outcolor}Out[{\color{outcolor}226}]:} 
\end{Verbatim}
\begin{singlespace}
\begin{table}[H]
\centering
\tiny
\begin{tabular}{|llllrrrllll|}
\hline
\textbf{{}} & \textbf{file} & \textbf{original key} & \textbf{      key} & \textbf{ cents} & \textbf{ cents from just} & \textbf{ semitones} & \textbf{name} & \textbf{direction} & \textbf{n1 name} & \textbf{n2 name }\\
\hline
0 &  c\_1 &      C minor &  F\# minor &     90 &              -22 &          1 &   m2 &     False &      C\# &      B\# \\
1 &  c\_1 &      C minor &  F\# minor &     90 &              -22 &          1 &   m2 &      True &      B\# &      C\# \\
2 &  c\_1 &      C minor &  F\# minor &    702 &                0 &          7 &   P5 &     False &      C\# &      F\# \\
3 &  c\_1 &      C minor &  F\# minor &    300 &              -16 &          3 &   m3 &      True &      F\# &       A \\
4 &  c\_1 &      C minor &  F\# minor &    402 &               16 &          4 &   M3 &      True &       A &      C\# \\
\hline
\end{tabular}
\caption{C minor fugue transposed to F\# minor dataframe. (Selected Ten Columns) }
\end{table}
\normalsize
\end{singlespace}


    \subsubsection{Dataframe Lists and Concatenated
Dataframes}\label{dataframe-lists-and-concatenated-dataframes}

The \texttt{bach\_df.py} data module not only imports and assigns
variable names to each of the 624 dataframes, but also organizes them
into various Python lists for easier analysis. The module also
concatenates all of the major-key fugues into one large dataframe, as
well as all the minor fugues into another one; this concatenation
simplifies the process of performing a search across all fugues, as only
one search needs to be run on the concatenated dataframe, instead of on
each dataframe on the individual level. The lists created from the
module are as follows, and as stated above, all fugue dataframes
contained within \texttt{bach\_df.py} are from Book I:

\begin{itemize}
\tightlist
\item
  \texttt{M\_vt}: list of 12 major fugue vertical dataframes in
  chromatic order
\item
  \texttt{m\_vt}: list of 12 minor fugues vertical dataframes in
  chromatic order
\item
  \texttt{M\_vt\_circle}: list of 12 major fugues vertical dataframes
  ordered by the circle of fifths
\item
  \texttt{m\_vt\_circle}: list of 12 minor fugues vertical dataframes
  ordered by the circle of fifths
\item
  \texttt{M\_vt\_concat}: Dataframe of the concatenation of all 12 major
  fugue vertical dataframes
\item
  \texttt{m\_vt\_concat}: Dataframe of the concatenation of all 12 minor
  fugue vertical dataframes 
\item
  \texttt{M\_hz}: list of 12 major fugue horizontal dataframes in
  chromatic order
\item
  \texttt{m\_hz}: list of 12 minor fugues horizontal dataframes in
  chromatic order
\item
  \texttt{M\_hz\_circle}: list of 12 major fugues horizontal dataframes
  ordered by the circle of fifths
\item
  \texttt{m\_hz\_circle}: list of 12 minor fugues horizontal dataframes
  ordered by the circle of fifths
\item
  \texttt{M\_hz\_concat}: Dataframe of the concatenation of all 12 major
  fugue horizontal dataframes
\item
  \texttt{m\_hz\_concat}: Dataframe of the concatenation of all 12 minor
  fugue horizontal dataframes
\end{itemize}

Transposed fugues are also housed in lists that are organized in the
same fashion (by mode, and by interval type). There are two different
ways that these transposed fugues are grouped and organized, the first
is by the fugue itself, denoted by a capital "T" after the fugue's name
and dataframe type (e.g. \texttt{C\_1\_hz\_T} for a list of all the
transpositions for the C major fugue - horizontal intervals). Keys are
organized in chromatic order, and like any Python lists, individual
elements from the list can be called via Python indexing, for example,
if we want to select the D major transposition of the C major fugue, we
can simple enter: \texttt{C\_1\_hz\_T{[}2{]}}, and that should yield the
same dataframe as the variable \texttt{C\_1\_to\_D\_hz} (Python indexing
starts at 0 instead of 1).

The second method organizes the transposed fugues into lists by way of
key, that is to say, each major and minor key has its own list
containing all the fugues transposed to that particular key. The syntax
of these lists is 'fg\_1\_to\_\{key\}\_\{type\}'. For example,
\texttt{fg\_1\_to\_C\_hz} would yield a list of all 12 major fugues
transposed to C major. As is with the previous lists, individual fugues
can also be called from this list using indexing.

As a note, all scalar variables and list variables in both
\texttt{bach.py} and \texttt{bach\_df.py} modules (and all Python
variables, for that matter) are case sensitive, with capitalized letters
denoting major mode, and lowercase for minor mode. For all the dataframe
lists, and most list items provided in this dissertation, the default
key organization is chromatic, but passing the list through the function
\texttt{circle\_sort()} will reorder the list according to the circle of
fifths.

\subsubsection{Working and Searching within
Dataframes}\label{working-and-searching-within-dataframes}

In this section I will go through some techniques that will be used to
conduct searches within and navigate these provided fugue Dataframes;
before I begin though I will offer a brief description as to what a
Dataframe is, for the readers that are unfamilar with this
\texttt{pandas} container: A Dataframe is \texttt{pandas}'
two-dimensional labeled data structure that houses information in the
form of a table consisting of rows and columns. Rows correspond to
instances of the variable of interest (in this case, intervals), and
each column vector houses a different attribute of the row variable
(duration of interval, cents contained in the interval, direction of
interval, etc.). Dataframes are powerful in their versatility, in that
virtually any type of Python object, be it integers, strings,
\texttt{music21} objects, lists, \texttt{numpy} matricies, or even other
Dataframes can be housed as individual cell entries. They are also a
powerful organizational tool, as Dataframes can be manipulated through
the selection of existing rows and columns, either through direct
indexing, or passing in a conditional statement (or series of
conditional statements) to be satisfied, or additional new rows and
columns created based on a certain configuration of existing values. The
translation between Dataframe and \texttt{seaborn} plots is also
straightforward and convenient, as \texttt{seaborn} was essentially
written with the Dataframe in mind.

Let us now then pull up a Dataframe as an example to examine its anatomy
- for this instance let us again take a look at the c minor fugue from
Book I. The first and last five entries of the dataframe are shown here:


\begin{table}[H]
   \centering
   \tiny
   \setlength{\tabcolsep}{1pt}
   \iftoggle{pretty}
   {\setstackgap{L}{1.1\normalbaselineskip}}
   {\setstackgap{L}{0.6\normalbaselineskip}}
\begin{tabular}{|lcccccccccccccccc|}
%\begin{tabular}{|lllrrrrllllrrrrrl|}
    \hline
    \textbf {{} } & \textbf{file } & \textbf{key     } & \textbf{cents } & \textbf{\Longunderstack{cents from \\ just} } & \textbf{semitones } & \textbf{\Longunderstack{generic \\ interval} } & \textbf{\Longunderstack{simple \\ name} } & \textbf{name } & \textbf{\Longunderstack{directed \\ name} } & \textbf{direction } & \textbf{\Longunderstack{total \\ semitones} } & \textbf{\Longunderstack{n1 \\ duration} } & \textbf{\Longunderstack{n2 \\ duration} } & \textbf{offset } & \textbf{beat } & \textbf{\Longunderstack{n1 \\ name}  }\\[1.1em]
   \hline
   0   & c\_1 & C minor & 108   & -4                                & 1         & 2                                & m2                          & m2   & m-2                           & False     & 1                               & 0.25                        & 0.25                        & 0.50   & 1.50 & G                        \\
   1   & c\_1 & C minor & 108   & -4                                & 1         & 2                                & m2                          & m2   & m2                            & True      & 1                               & 0.25                        & 0.50                        & 8.75   & 1.75 & F\#                      \\
   2   & c\_1 & C minor & 696   & -6                                & 7         & 5                                & P5                          & P5   & P-5                           & False     & 7                               & 0.50                        & 0.50                        & 9.00   & 2.00 & G                        \\
   3   & c\_1 & C minor & 294   & -22                               & 3         & 3                                & m3                          & m3   & m3                            & True      & 3                               & 0.50                        & 0.50                        & 9.50   & 2.50 & C                        \\
   4   & c\_1 & C minor & 402   & 16                                & 4         & 3                                & M3                          & M3   & M3                            & True      & 4                               & 0.50                        & 0.25                        & 10.00  & 3.00 & E-                       \\
   742 & c\_1 & C minor & 0     & 0                                 & 0         & 1                                & P1                          & P8   & P8                            & True      & 12                              & 1.00                        & 0.50                        & 110.00 & 3.00 & E-                       \\
   743 & c\_1 & C minor & 102   & -10                               & 1         & 2                                & m2                          & m2   & m-2                           & False     & 1                               & 0.50                        & 0.50                        & 111.50 & 4.50 & E-                       \\
   744 & c\_1 & C minor & 192   & -12                               & 2         & 2                                & M2                          & M2   & M-2                           & False     & 2                               & 0.50                        & 0.50                        & 112.00 & 1.00 & D                        \\
   745 & c\_1 & C minor & 696   & -6                                & 7         & 5                                & P5                          & P5   & P5                            & True      & 7                               & 0.50                        & 0.50                        & 112.50 & 1.50 & C                        \\
   746 & c\_1 & C minor & 0     & 0                                 & 0         & 1                                & P1                          & P8   & P-8                           & False     & 12                              & 0.50                        & 0.50                        & 113.00 & 2.00 & G                        \\
   \hline
\end{tabular}
\caption{C minor Fugue Horizontal Intervals Dataframe (first and last five entries) Part 1}
\end{table}
 
\vspace{1em}
\vspace{1em}
\begin{table}[H]
   \centering
\tiny
   \setlength{\tabcolsep}{1pt}
   \iftoggle{pretty}
   {\setstackgap{L}{1.1\normalbaselineskip}}
   {\setstackgap{L}{0.6\normalbaselineskip}}
%\begin{tabular}{|lrrrrlrlrlrrrrrll|}
\begin{tabular}{|lcccccccccccccccc|}
  \hline
  \textbf{{}  } & \textbf{\Longunderstack{n2 \\ name} } & \textbf{\Longunderstack{n1 \\ octave} } & \textbf{\Longunderstack{n2 \\ octave} } & \textbf{measure } & \textbf{part } & \textbf{subject } & \textbf{\Longunderstack{subject \\ offset} } & \textbf{\Longunderstack{subject \\ 1} } & \textbf{\Longunderstack{subject \\ 1 offset} } & \textbf{\Longunderstack{subject \\ 2}} & \textbf{\Longunderstack{subject 2 \\ offset} } & \textbf{count } & \textbf{\Longunderstack{norm \\ count} } & \textbf{\Longunderstack{sc \\ degree 1} } & \textbf{\Longunderstack{sc \\ degree 2} } & \textbf{\Longunderstack{lower \\ note}} \\[1.1em]
   \hline
   0   & F\#                     & 5                         & 5                         & 3       & 2    & True    & 0.00                           & True                      & 0.00                             & False                    & 0.0                              & 1     & 0.001339                   & 7.0                         & 6.0                         & F\#                        \\
   1   & G                       & 5                         & 5                         & 3       & 2    & True    & 0.25                           & True                      & 0.25                             & False                    & 0.0                              & 1     & 0.001339                   & 6.0                         & 7.0                         & F\#                        \\
   2   & C                       & 5                         & 5                         & 3       & 2    & True    & 0.50                           & True                      & 0.50                             & False                    & 0.0                              & 1     & 0.001339                   & 7.0                         & 0.0                         & C                          \\
   3   & E-                      & 5                         & 5                         & 3       & 2    & True    & 1.00                           & True                      & 1.00                             & False                    & 0.0                              & 1     & 0.001339                   & 0.0                         & 3.0                         & C                          \\
   4   & G                       & 5                         & 5                         & 3       & 2    & True    & 1.50                           & True                      & 1.50                             & False                    & 0.0                              & 1     & 0.001339                   & 3.0                         & 7.0                         & E-                         \\
   742 & E-                      & 2                         & 3                         & 28      & 0    & False   & 0.00                           & False                     & 0.00                             & False                    & 0.0                              & 1     & 0.001339                   & 3.0                         & 3.0                         & E-                         \\
   743 & D                       & 3                         & 3                         & 28      & 0    & False   & 0.00                           & False                     & 0.00                             & False                    & 0.0                              & 1     & 0.001339                   & 3.0                         & 2.0                         & D                          \\
   744 & C                       & 3                         & 3                         & 29      & 0    & False   & 0.00                           & False                     & 0.00                             & False                    & 0.0                              & 1     & 0.001339                   & 2.0                         & 0.0                         & C                          \\
   745 & G                       & 3                         & 3                         & 29      & 0    & False   & 0.00                           & False                     & 0.00                             & False                    & 0.0                              & 1     & 0.001339                   & 0.0                         & 7.0                         & C                          \\
   746 & G                       & 3                         & 2                         & 29      & 0    & False   & 0.00                           & False                     & 0.00                             & False                    & 0.0                              & 1     & 0.001339                   & 7.0                         & 7.0                         & G                          \\
   \hline
\end{tabular}
\caption{C minor Fugue Horizontal Intervals Dataframe (first and last five entries) Part 2}
\end{table}
 
\vspace{1em}
\vspace{1em}
\begin{table}[H]
   \centering
\tiny
   \setlength{\tabcolsep}{1pt}
   \iftoggle{pretty}
   {\setstackgap{L}{1.1\normalbaselineskip}}
   {\setstackgap{L}{0.6\normalbaselineskip}}
\begin{tabular}{|lccccccccccccccccc|}
%\begin{tabular}{|lrrrrrrlrlrrrrrrrr|}
   \hline
   {}  & \textbf{\Longunderstack{upper \\ note} } & \textbf{\Longunderstack{sc \\ degree L} } & \textbf{\Longunderstack{sc \\ degree U} } & \textbf{\Longunderstack{av dur} } & \textbf{\Longunderstack{av dur \\ norm} } & \textbf{\Longunderstack{abs \\ cfj} } & \textbf{\Longunderstack{int \\ class} } & \textbf{\Longunderstack{ic \\ base} } & \textbf{\Longunderstack{ic \\ base sc} } & \textbf{\Longunderstack{ic \\ nb} } & \textbf{\Longunderstack{ic nb \\ sc} } & \textbf{\Longunderstack{n1 \\ PC} } & \textbf{\Longunderstack{n2 PC} } & \textbf{\Longunderstack{L PC} } & \textbf{\Longunderstack{U PC} } & \textbf{\Longunderstack{icb \\ PC} } & \textbf{\Longunderstack{icnb \\ PC} }\\[1.1em]
   \hline
   0   & G                          & 6.0                         & 7.0                         & 0.250               & 0.000854                    & 4.0                     & 1.0                       & F\#                     & 6.0                        & G                     & 7.0                      & 7.0                   & 6.0                & 6.0               & 7.0               & 6.0                    & 7.0 \\
   1   & G                          & 6.0                         & 7.0                         & 0.375               & 0.001280                    & 4.0                     & 1.0                       & F\#                     & 6.0                        & G                     & 7.0                      & 6.0                   & 7.0                & 6.0               & 7.0               & 6.0                    & 7.0 \\
   2   & G                          & 0.0                         & 7.0                         & 0.500               & 0.001707                    & 6.0                     & 5.0                       & G                       & 7.0                        & C                     & 0.0                      & 7.0                   & 0.0                & 0.0               & 7.0               & 7.0                    & 0.0 \\
   3   & E-                         & 0.0                         & 3.0                         & 0.500               & 0.001707                    & 22.0                    & 3.0                       & C                       & 0.0                        & E-                    & 3.0                      & 0.0                   & 3.0                & 0.0               & 3.0               & 0.0                    & 3.0 \\
   4   & G                          & 3.0                         & 7.0                         & 0.375               & 0.001280                    & 16.0                    & 4.0                       & E-                      & 3.0                        & G                     & 7.0                      & 3.0                   & 7.0                & 3.0               & 7.0               & 3.0                    & 7.0 \\
   742 & E-                         & 3.0                         & 3.0                         & 0.750               & 0.002561                    & 0.0                     & 0.0                       & E-                      & 3.0                        & E-                    & 3.0                      & 3.0                   & 3.0                & 3.0               & 3.0               & 3.0                    & 3.0 \\
   743 & E-                         & 2.0                         & 3.0                         & 0.500               & 0.001707                    & 10.0                    & 1.0                       & D                       & 2.0                        & E-                    & 3.0                      & 3.0                   & 2.0                & 2.0               & 3.0               & 2.0                    & 3.0 \\
   744 & D                          & 0.0                         & 2.0                         & 0.500               & 0.001707                    & 12.0                    & 2.0                       & C                       & 0.0                        & D                     & 2.0                      & 2.0                   & 0.0                & 0.0               & 2.0               & 0.0                    & 2.0 \\
   745 & G                          & 0.0                         & 7.0                         & 0.500               & 0.001707                    & 6.0                     & 5.0                       & G                       & 7.0                        & C                     & 0.0                      & 0.0                   & 7.0                & 0.0               & 7.0               & 7.0                    & 0.0 \\
   746 & G                          & 7.0                         & 7.0                         & 0.500               & 0.001707                    & 0.0                     & 0.0                       & G                       & 7.0                        & G                     & 7.0                      & 7.0                   & 7.0                & 7.0               & 7.0               & 7.0                    & 7.0 \\
   \hline
\end{tabular}
\caption{C minor Fugue Horizontal Intervals Dataframe (first and last five entries) Part 3}
\end{table}
    Once displayed, the layout and structure of the Dataframe is quite
intuitive; the index numbers of the rows are marked in bold in the
leftmost, unnamed column, and the first row marks the instance of the
first horizontal interval in the piece contained in the highest voice
(recalling the \texttt{bach.py} section, horizontal searches are
organized by analyzing parts one by one, starting from the highest voice
and proceeding downward), and each following row proceeds through each
successive interval, through every voice, until the last interval
presented in the lowest voice is documented in the final row (in this
case, row 746). The column vectors of the dataframe house different
attributes of these intervals; there are 49 columns for the horizontal
search Dataframes (50 for the transposition ones accounting for an extra
column indicating the original key). The structure of the vertical
dataframe is very similar, with only slight modification of the column
categories.

\paragraph{Conditional Selection for Searching Through
Dataframes}\label{conditional-selection-for-searching-through-dataframes}

Now that we have somewhat familiarized ourselves with the structure of
the Dataframe, still using our c minor horizontal dataframe, the
following is an example of how these dataframes can be easily
parameterized with conditional statements to extract specific
information for analysis. This particular dataframe contains 746
entries, meaning that across the fugue's total 3 voices, there are
exactly 746 horizontal adjacent intervals contained in the entire piece.
But say we only wanted to focus on intervals of a minor second, and
filter out all other intervals, we could easily parameterize this
dataframe with a conditional that instructs it to return rows containing
only intervals that have the value of 'm2' in the 'name' column.

    \begin{Verbatim}[commandchars=\\\{\}]
{\color{incolor}In [{\color{incolor}373}]:} \PY{n}{c\PYZus{}1\PYZus{}hz}\PY{p}{[}\PY{n}{c\PYZus{}1\PYZus{}hz}\PY{p}{[}\PY{l+s+s1}{\PYZsq{}}\PY{l+s+s1}{name}\PY{l+s+s1}{\PYZsq{}}\PY{p}{]}\PY{o}{==}\PY{l+s+s1}{\PYZsq{}}\PY{l+s+s1}{m2}\PY{l+s+s1}{\PYZsq{}}\PY{p}{]}\PY{o}{.}\PY{n}{head}\PY{p}{(}\PY{p}{)}
\end{Verbatim}
\begin{Verbatim}[commandchars=\\\{\}]
{\color{outcolor}Out[{\color{outcolor}373}]:} 
\end{Verbatim}
\begin{singlespace}
\begin{table}[H]
\centering
\tiny
\begin{tabular}{|lllrrrrllll|}
\hline
\textbf{{}} & \textbf{file} & \textbf{     key} & \textbf{ cents} & \textbf{ cents from just} & \textbf{ semitones} & \textbf{ generic interval} & \textbf{simple name} & \textbf{name} & \textbf{directed name} & \textbf{direction }\\
\hline
0  &  c\_1 &  C minor &    108 &               -4 &          1 &                 2 &          m2 &   m2 &           m-2 &     False \\
1  &  c\_1 &  C minor &    108 &               -4 &          1 &                 2 &          m2 &   m2 &            m2 &      True \\
5  &  c\_1 &  C minor &    108 &               -4 &          1 &                 2 &          m2 &   m2 &           m-2 &     False \\
6  &  c\_1 &  C minor &    108 &               -4 &          1 &                 2 &          m2 &   m2 &            m2 &      True \\
10 &  c\_1 &  C minor &    108 &               -4 &          1 &                 2 &          m2 &   m2 &           m-2 &     False \\
\hline
\end{tabular}
\caption[C minor Fugue Horizontal Intervals Dataframe, Minor Seconds Only. ]{C Minor Fugue Horizontal Intervals Dataframe, Minor Seconds Only. (First Ten Columns)}
\end{table}
\normalsize
\end{singlespace}

    \begin{Verbatim}[commandchars=\\\{\}]
{\color{incolor}In [{\color{incolor}375}]:} \PY{n+nb}{len}\PY{p}{(}\PY{n}{c\PYZus{}1\PYZus{}hz}\PY{p}{[}\PY{n}{c\PYZus{}1\PYZus{}hz}\PY{p}{[}\PY{l+s+s1}{\PYZsq{}}\PY{l+s+s1}{name}\PY{l+s+s1}{\PYZsq{}}\PY{p}{]}\PY{o}{==}\PY{l+s+s1}{\PYZsq{}}\PY{l+s+s1}{m2}\PY{l+s+s1}{\PYZsq{}}\PY{p}{]}\PY{p}{)}
\end{Verbatim}
\begin{Verbatim}[commandchars=\\\{\}]
{\color{outcolor}Out[{\color{outcolor}375}]:} 250
\end{Verbatim}
    Displayed above is only the first four entries of this new dataframe (in
reality, this is rather big dataframe that contains 250 entries, since
minor seconds are a very commonplace interval on the horizontal domain
of most any composition in the Western tonal practice), but already we
can see that, under the name column, only minor seconds are present,
whereas our original dataframe included other intervals. But let's say
that we wanted to get more specific, and look at only descending minor
seconds, ignoring the ascending ones. Using the logical \texttt{\&}, we
can easily add another search parameter and limit our dataframe further
by restricting the 'direction' column to return only values with the
string 'False'.

    \begin{Verbatim}[commandchars=\\\{\}]
{\color{incolor}In [{\color{incolor}415}]:} \PY{n}{c\PYZus{}1\PYZus{}hz}\PY{p}{[}\PY{p}{(}\PY{n}{c\PYZus{}1\PYZus{}hz}\PY{p}{[}\PY{l+s+s1}{\PYZsq{}}\PY{l+s+s1}{name}\PY{l+s+s1}{\PYZsq{}}\PY{p}{]}\PY{o}{==}\PY{l+s+s1}{\PYZsq{}}\PY{l+s+s1}{m2}\PY{l+s+s1}{\PYZsq{}}\PY{p}{)}\PY{o}{\PYZam{}}\PY{p}{(}\PY{n}{c\PYZus{}1\PYZus{}hz}\PY{p}{[}\PY{l+s+s1}{\PYZsq{}}\PY{l+s+s1}{direction}\PY{l+s+s1}{\PYZsq{}}\PY{p}{]}\PY{o}{==}\PY{l+s+s1}{\PYZsq{}}\PY{l+s+s1}{False}\PY{l+s+s1}{\PYZsq{}}\PY{p}{)}\PY{p}{]}\PY{o}{.}\PY{n}{head}\PY{p}{(}\PY{p}{)}
\end{Verbatim}
\begin{Verbatim}[commandchars=\\\{\}]
{\color{outcolor}Out[{\color{outcolor}415}]:} 
\end{Verbatim}
\begin{singlespace}
\begin{table}[H]
\centering
\tiny
\begin{tabular}{|lllrrrrllll|}
\hline
\textbf{{}} & \textbf{file} & \textbf{     key} & \textbf{ cents} & \textbf{ cents from just} & \textbf{ semitones} & \textbf{ generic interval} & \textbf{simple name} & \textbf{name} & \textbf{directed name} & \textbf{direction }\\
\hline
0  &  c\_1 &  C minor &    108 &               -4 &          1 &                 2 &          m2 &   m2 &           m-2 &     False \\
5  &  c\_1 &  C minor &    108 &               -4 &          1 &                 2 &          m2 &   m2 &           m-2 &     False \\
10 &  c\_1 &  C minor &    108 &               -4 &          1 &                 2 &          m2 &   m2 &           m-2 &     False \\
16 &  c\_1 &  C minor &    102 &              -10 &          1 &                 2 &          m2 &   m2 &           m-2 &     False \\
20 &  c\_1 &  C minor &    102 &              -10 &          1 &                 2 &          m2 &   m2 &           m-2 &     False \\
\hline
\end{tabular}
\caption[C minor Fugue Horizontal Intervals Dataframe, Descending Minor Seconds Only. ]{C Minor Fugue Horizontal Intervals Dataframe, Descending Minor Seconds Only. (First Ten Columns)}
\end{table}
\normalsize
\end{singlespace}

    \begin{Verbatim}[commandchars=\\\{\}]
{\color{incolor}In [{\color{incolor}416}]:} \PY{n+nb}{len}\PY{p}{(}\PY{n}{c\PYZus{}1\PYZus{}hz}\PY{p}{[}\PY{p}{(}\PY{n}{c\PYZus{}1\PYZus{}hz}\PY{p}{[}\PY{l+s+s1}{\PYZsq{}}\PY{l+s+s1}{name}\PY{l+s+s1}{\PYZsq{}}\PY{p}{]}\PY{o}{==}\PY{l+s+s1}{\PYZsq{}}\PY{l+s+s1}{m2}\PY{l+s+s1}{\PYZsq{}}\PY{p}{)}\PY{o}{\PYZam{}}\PY{p}{(}\PY{n}{c\PYZus{}1\PYZus{}hz}\PY{p}{[}\PY{l+s+s1}{\PYZsq{}}\PY{l+s+s1}{direction}\PY{l+s+s1}{\PYZsq{}}\PY{p}{]}\PY{o}{==}\PY{l+s+s1}{\PYZsq{}}\PY{l+s+s1}{False}\PY{l+s+s1}{\PYZsq{}}\PY{p}{)}\PY{p}{]}\PY{p}{)}
\end{Verbatim}
\begin{Verbatim}[commandchars=\\\{\}]
{\color{outcolor}Out[{\color{outcolor}416}]:} 118
\end{Verbatim}
    We now see that this new dataframe contains only descending minor
seconds, with 118 entries (just about half of the previous 250 entries).
If we look in the 'directed name' column, we can also verify that the
intervals in that vector are indeed all descending minor seconds (m-2).
As a note here, these two separate steps in this particular case could
have been combined and simplified into one conditional statement;
setting the 'directed name' column to only return values of 'm-2' would
have yielded the same result as our code which took two separate
conditions. While this instance has a simplified solution, not all
conditional statements can be neatly combined in this manner, as we will
see in the next additional parameter we will introduce to limit our
dataset. But first, let us visualize the distribution of the various
sizes (in cents) of these minor seconds using a \texttt{seaborn} bar
chart:




\begin{figure}[H]
\vspace{1.5em}
    \centering
    \adjustimage{max size={0.6\linewidth}{0.9\paperheight}}{methods_files/methods_58_0.png}
    \caption{ Descending Minor Seconds in c minor fugue in cents. }
\end{figure}    From this simple bar chart, with cents on the x-axis, and the count on
the y-axis, we can see that trend of the minor seconds in this piece
gently tend towards the wider interval measurements, which, in the case
of the minor second, are closer to pure (indicated by bars of increased
transparency in this chart). Suppose at this point we would like to look
at descending minor seconds, but only such seconds contained in the
subject material. We would further constrain the existing dataframe by
introducing a new parameter selecting for only 'True' values in the
'subject' column.

    \begin{Verbatim}[commandchars=\\\{\}]
{\color{incolor}In [{\color{incolor}418}]:} \PY{n}{c\PYZus{}1\PYZus{}hz}\PY{p}{[}\PY{p}{(}\PY{n}{c\PYZus{}1\PYZus{}hz}\PY{p}{[}\PY{l+s+s1}{\PYZsq{}}\PY{l+s+s1}{name}\PY{l+s+s1}{\PYZsq{}}\PY{p}{]}\PY{o}{==}\PY{l+s+s1}{\PYZsq{}}\PY{l+s+s1}{m2}\PY{l+s+s1}{\PYZsq{}}\PY{p}{)}\PY{o}{\PYZam{}}\PY{p}{(}\PY{n}{c\PYZus{}1\PYZus{}hz}\PY{p}{[}\PY{l+s+s1}{\PYZsq{}}\PY{l+s+s1}{direction}\PY{l+s+s1}{\PYZsq{}}\PY{p}{]}\PY{o}{==}\PY{l+s+s1}{\PYZsq{}}\PY{l+s+s1}{False}\PY{l+s+s1}{\PYZsq{}}\PY{p}{)}\PYZbs{}
                 \PY{o}{\PYZam{}}\PY{p}{(}\PY{n}{c\PYZus{}1\PYZus{}hz}\PY{p}{[}\PY{l+s+s1}{\PYZsq{}}\PY{l+s+s1}{subject}\PY{l+s+s1}{\PYZsq{}}\PY{p}{]}\PY{o}{==}\PY{l+s+s1}{\PYZsq{}}\PY{l+s+s1}{True}\PY{l+s+s1}{\PYZsq{}}\PY{p}{)}\PY{p}{]}\PY{o}{.}\PY{n}{head}\PY{p}{(}\PY{l+m+mi}{8}\PY{p}{)}
\end{Verbatim}
\begin{Verbatim}[commandchars=\\\{\}]
{\color{outcolor}Out[{\color{outcolor}418}]:} 
\end{Verbatim}
\begin{singlespace}
\begin{table}[H]
\centering
\tiny
\begin{tabular}{|lllrrrlllr|}
\hline
\textbf{{}} & \textbf{file} & \textbf{     key} & \textbf{ cents} & \textbf{ cents from just} & \textbf{ semitones} & \textbf{name} & \textbf{direction} & \textbf{subject} & \textbf{ subject offset }\\
\hline
0  &  c\_1 &  C minor &    108 &               -4 &          1 &   m2 &     False &    True &            0.00 \\
5  &  c\_1 &  C minor &    108 &               -4 &          1 &   m2 &     False &    True &            2.00 \\
10 &  c\_1 &  C minor &    108 &               -4 &          1 &   m2 &     False &    True &            4.00 \\
16 &  c\_1 &  C minor &    102 &              -10 &          1 &   m2 &     False &    True &            6.00 \\
72 &  c\_1 &  C minor &    102 &              -10 &          1 &   m2 &     False &    True &            0.00 \\
77 &  c\_1 &  C minor &    102 &              -10 &          1 &   m2 &     False &    True &            2.00 \\
82 &  c\_1 &  C minor &    102 &              -10 &          1 &   m2 &     False &    True &            4.00 \\
90 &  c\_1 &  C minor &     96 &              -16 &          1 &   m2 &     False &    True &            7.25 \\
\hline
\end{tabular}
\caption[C minor Fugue Horizontal Intervals Dataframe, Subject Descending Minor Seconds Only. ]{C Minor Fugue Horizontal Intervals Dataframe, Subject Descending Minor Seconds Only. (First Nine Columns)}
\end{table}
\normalsize
\end{singlespace}



    \begin{Verbatim}[commandchars=\\\{\}]
{\color{incolor}In [{\color{incolor}419}]:} \PY{n+nb}{len}\PY{p}{(}\PY{n}{c\PYZus{}1\PYZus{}hz}\PY{p}{[}\PY{p}{(}\PY{n}{c\PYZus{}1\PYZus{}hz}\PY{p}{[}\PY{l+s+s1}{\PYZsq{}}\PY{l+s+s1}{name}\PY{l+s+s1}{\PYZsq{}}\PY{p}{]}\PY{o}{==}\PY{l+s+s1}{\PYZsq{}}\PY{l+s+s1}{m2}\PY{l+s+s1}{\PYZsq{}}\PY{p}{)}\PY{o}{\PYZam{}}\PY{p}{(}\PY{n}{c\PYZus{}1\PYZus{}hz}\PY{p}{[}\PY{l+s+s1}{\PYZsq{}}\PY{l+s+s1}{direction}\PY{l+s+s1}{\PYZsq{}}\PY{p}{]}\PY{o}{==}\PY{l+s+s1}{\PYZsq{}}\PY{l+s+s1}{False}\PY{l+s+s1}{\PYZsq{}}\PY{p}{)}\PYZbs{}
                     \PY{o}{\PYZam{}}\PY{p}{(}\PY{n}{c\PYZus{}1\PYZus{}hz}\PY{p}{[}\PY{l+s+s1}{\PYZsq{}}\PY{l+s+s1}{subject}\PY{l+s+s1}{\PYZsq{}}\PY{p}{]}\PY{o}{==}\PY{l+s+s1}{\PYZsq{}}\PY{l+s+s1}{True}\PY{l+s+s1}{\PYZsq{}}\PY{p}{)}\PY{p}{]}\PY{p}{)}
\end{Verbatim}
\begin{Verbatim}[commandchars=\\\{\}]
{\color{outcolor}Out[{\color{outcolor}419}]:} 33
\end{Verbatim}
    As we can see from the output value returned by the length function,
this new dataframe is much smaller - only 33 entries long. If we scroll
to the far right of the dataframe, we can see that all the values
contained in the 'subject' column are now all 'True'. Let us visualize
the data in \texttt{seaborn} and see if this new constraint on the
subject has affected our interval distribution:




\begin{figure}[H]
\vspace{1.5em}
    \centering
    \adjustimage{max size={0.6\linewidth}{0.9\paperheight}}{methods_files/methods_68_0.png}
    \caption{ Subject Descending Minor Seconds in c minor fugue in cents. }
\end{figure}    Interestingly, while the previous distribution containing all minor
seconds (both subject and non-subject) only showed a gentle skew towards
the wider measurements, this new distribution of minor seconds within
subject material is skewed much stronger, with a majority of the
intervals at 108 cents, only a few intervals at 102 and 96 cents, and no
intervals at 90 cents. Without getting too much into the analysis side
of things yet, such a dramatic change in the distribution might lead us
to speculate as to whether there is anything special or deliberate about
Bach's choice of semitone in the subject motifs, but more on that in the
following analysis chapters.

Let us constrain the dataset one final time: if we analyze the subject
statement of the c minor fugue, we can see that three of the four
descending semitone instances essentially belong to the same opening
subject motif, which is stated three times within each appearance of the
subject material. The fourth descending semitone in the subject material
does not belong to this opening statement, but rather is merely a
passing semitone within a descending scale at the ending of the subject.
Because this fourth semitone is motivically different, and more
importantly, lacks the motivic significance of the first three, we have
a good reason to remove it from the subject dataset, and only focus on
the first three semitone statements in the subject material. This can be
done through selecting values in the 'subject offset' column less than
or equal to 4 (rough translation - we are cutting off any instances of a
semitone after the third iteration of the subject head motif). Here is
the resulting dataframe, and its corresponding \texttt{seaborn} chart:

    \begin{Verbatim}[commandchars=\\\{\}]
{\color{incolor}In [{\color{incolor}12}]:} \PY{n}{c\PYZus{}1\PYZus{}hz}\PY{p}{[}\PY{p}{(}\PY{n}{c\PYZus{}1\PYZus{}hz}\PY{p}{[}\PY{l+s+s1}{\PYZsq{}}\PY{l+s+s1}{name}\PY{l+s+s1}{\PYZsq{}}\PY{p}{]}\PY{o}{==}\PY{l+s+s1}{\PYZsq{}}\PY{l+s+s1}{m2}\PY{l+s+s1}{\PYZsq{}}\PY{p}{)}\PY{o}{\PYZam{}}\PY{p}{(}\PY{n}{c\PYZus{}1\PYZus{}hz}\PY{p}{[}\PY{l+s+s1}{\PYZsq{}}\PY{l+s+s1}{direction}\PY{l+s+s1}{\PYZsq{}}\PY{p}{]}\PY{o}{==}\PY{l+s+s1}{\PYZsq{}}\PY{l+s+s1}{False}\PY{l+s+s1}{\PYZsq{}}\PY{p}{)}\PYZbs{}
                \PY{o}{\PYZam{}}\PY{p}{(}\PY{n}{c\PYZus{}1\PYZus{}hz}\PY{p}{[}\PY{l+s+s1}{\PYZsq{}}\PY{l+s+s1}{subject}\PY{l+s+s1}{\PYZsq{}}\PY{p}{]}\PY{o}{==}\PY{l+s+s1}{\PYZsq{}}\PY{l+s+s1}{True}\PY{l+s+s1}{\PYZsq{}}\PY{p}{)}\PYZbs{}
                \PY{o}{\PYZam{}}\PY{p}{(}\PY{n}{c\PYZus{}1\PYZus{}hz}\PY{p}{[}\PY{l+s+s1}{\PYZsq{}}\PY{l+s+s1}{subject offset}\PY{l+s+s1}{\PYZsq{}}\PY{p}{]}\PY{o}{\PYZlt{}}\PY{o}{=}\PY{l+m+mi}{4}\PY{p}{)}\PY{p}{]}\PY{o}{.}\PY{n}{head}\PY{p}{(}\PY{l+m+mi}{8}\PY{p}{)}
\end{Verbatim}
\begin{Verbatim}[commandchars=\\\{\}]
{\color{outcolor}Out[{\color{outcolor}12}]:} 
\end{Verbatim}
\begin{singlespace}
\begin{table}[H]
\centering
\tiny
\begin{tabular}{|lllrrrlllr|}
\hline
\textbf{{}} & \textbf{file} & \textbf{     key} & \textbf{ cents} & \textbf{ cents from just} & \textbf{ semitones} & \textbf{name} & \textbf{direction} & \textbf{subject} & \textbf{ subject offset }\\
\hline
0   &  c\_1 &  C minor &    108 &               -4 &          1 &   m2 &     False &    True &             0.0 \\
5   &  c\_1 &  C minor &    108 &               -4 &          1 &   m2 &     False &    True &             2.0 \\
10  &  c\_1 &  C minor &    108 &               -4 &          1 &   m2 &     False &    True &             4.0 \\
72  &  c\_1 &  C minor &    102 &              -10 &          1 &   m2 &     False &    True &             0.0 \\
77  &  c\_1 &  C minor &    102 &              -10 &          1 &   m2 &     False &    True &             2.0 \\
82  &  c\_1 &  C minor &    102 &              -10 &          1 &   m2 &     False &    True &             4.0 \\
161 &  c\_1 &  C minor &    108 &               -4 &          1 &   m2 &     False &    True &             0.0 \\
166 &  c\_1 &  C minor &    108 &               -4 &          1 &   m2 &     False &    True &             2.0 \\
\hline
\end{tabular}
\caption[C minor Fugue Horizontal Intervals Dataframe, Subject Descending Minor Seconds Before Offset 4 Only. ]{C Minor Fugue Horizontal Intervals Dataframe, Subject Descending Minor Seconds Before Offset 4 Only. (First Nine Columns)}
\end{table}
\normalsize
\end{singlespace}





\begin{figure}[H]
\vspace{1.5em}
    \centering
    \adjustimage{max size={0.6\linewidth}{0.9\paperheight}}{methods_files/methods_77_0.png}
    \caption{ Subject Opening Motif Descending Minor Seconds in c minor fugue in cents. }
\end{figure}    The final resulting dataframe is only 25 entries long (roughly 3/4 of
33, which would make sense), and through the series of conditional
statements represents a very specific subset of the original dataframe
that characterized the entire piece. Essentially, we have limited and
modified the original dataframe to return an analysis for the subject's
opening semitone motif through just one line of code, and we can see how
constraining the problem results in a far more specific and detailed
analysis and focused distribution of intervals. The final, resulting
\texttt{seaborn} chart directly above shows a distribution that is
drastically skewed, with a vast majority of its values at 108 cents
(88\%), and the remaining 12\% at the next widest interval value of 102
cents. Compared to the earlier distribution of all descending minor
seconds across the entire fugue, this subject head motif distribution is
undoubtedly different.

This method of searching through these dataframe structures and
constraining the data through conditionals is the main model for
organizing the data for my dissertation's analysis, which will now begin
with the following chapter.


    % Add a bibliography block to the postdoc
    
    
    
