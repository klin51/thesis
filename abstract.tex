    
    
    
    

    
    The main focus of this thesis is to explore the role that the
well-tempered tuning system had upon Bach's compositional process in the
Well-Tempered Clavier, specifically the minor mode fugues of the first
volume, and to apply quantitative methods to demonstrate that
temperament is essentially encoded in the music itself. By approaching
analysis through the lens of temperament and examining the statistical
distributions of specific types of intervals, as well as looking at the
deeper implications these intervals exert upon companion musical
domains, we are able to arrive at a closer understanding of the musical
motivations behind why certain subjects and motifs are used for specific
keys, and how these work together with larger scale compositional
structures to contribute to a piece's---and key's---character,
expressivity, and integrity.

This dissertation utilizes a hybrid approach of computational and
traditional theoretical methods of score analysis, and a data-driven,
exploratory, tiered approach to scope, starting from non-parameterized,
global analysis, and proceeding to more localized and fugal analyses
based on the findings at the previous level to simultaneously establish
both the statistical and musical significance of our findings. The
results show a systematic, consistent correlation between temperament
and composition on all levels of scope, allowing us to confidently
reject the null hypothesis and support an alternative hypothesis stating
that there exists a significant, musical connection between the system
of well- temperament and Bach's compositional approach given key.

The cumulative goal of the analysis is to determine the importance, and
place, that temperament has amongst the other musical agents that
contribute to the musical experience, and to address the question as to
why temperament is important, not only to the scholar, but to the
performer as well, and why we should seek a greater understanding of its
connection to the structure and makeup of a musical work. The main claim
that this dissertation seeks to champion is that, just like harmony,
rhythm, structure, form, and other compositional elements, temperament
is an essential dimension of musical perception and aesthetic experience
that should not be neglected if we seek a complete representation and
understanding of the compositions that employ it.


    % Add a bibliography block to the postdoc
    
    
    
