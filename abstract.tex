    
    
    

    
    The main focus of this thesis is to explore the role that the
well-tempered tuning system had upon Bach's compositional process in the
Well-Tempered Clavier, and to apply quantitative methods to demonstrate
that temperament is essentially encoded in the music itself. By
approaching analysis through the lens of temperament and examining the
statistical distributions of specific types of intervals, as well as
looking at the deeper implications these intervals exert upon underlying
musical structures, I believe that we can come closer to understanding
the musical motivations behind why certain subjects and motifs are used
for specific keys, as well as how certain modulations and contrapuntal
devices contribute to a piece's tonal integrity.

Through a joint approach of computational and traditional theoretical
methods, score analysis will span the range from global corpus analysis
to more localized analysis, which will focus on individual fugues, or
subset of fugues, with specific attention to the extreme ends of the
temperament spectrum in the minor mode fugues. The importance of the
former is the determine statistical significance, its further purpose to
examine and characterize the distribution of certain types of intervals
in relation to pitch and key area. The latter, localized analysis
provides a closer insight into how the elements enumerated in the former
are achieved and manifested within the musical fabric of the score.

The cumulative goal of the analysis is to determine the importance, and
place, that temperament has amongst the other musical agents that
contribute to the musical experience, and to address the question as to
why temperament is important, not only to the scholar, but to the
musician as well, and why we should seek a greater understanding of its
connection to the structure and makeup of a musical work. The main claim
that this dissertation seeks to espouse---or negate, if not enough
adequate evidence can be concluded---is that, just like harmony, rhythm,
structure, form, and other compositional elements, temperament is an
essential dimension of musical perception and aesthetic experience that
should not be neglected if we seek a complete representation and
understanding of the compositions that employ it.


    % Add a bibliography block to the postdoc
    
    
    
