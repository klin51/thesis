    
    
    

    \hypertarget{Analysis I: Global Key Analysis}{\chapter{Analysis I: Global Key Analysis}\label{Analysis I: Global Key Analysis}}


    \section{Global Analysis Objectives}\label{global-analysis-objectives}

    This first chapter dedicated to analysis begins unfolding the narrative
on temperament that will span the next three chapters, all of which,
through the careful examination of various types of distributions of
intervals as a function of different musical elements, on varying scales
of generality, will cumulatively result in a deeper understanding of the
role temperament plays in shaping the underlying musical structures and
elements of compositions. The starting point of all of this will be
looking at distributions of intervals on a global, nonparameterized
scale, examining the relationship between temperament and key on a
compositional level, and how this trends across the volume as a whole.

While the distributions that we will be dealing with in this chapter are
very broad, and on the level of compositions as the unit of study, it is
on this macro level that we will be able to provide a proper bird's eye
view on temperament's relationship with key, and subsequently, whether
or not these effects extend to actual underlying compositional elements.
It is also only on this level that we can reliably establish these
effects to be statistically significant and have the properties of
consistency, and not just local phenomenons that may be due to a third,
unforeseen variable. Any of these large scale structures relating
temperament to key established in this primary section will serve as an
foundation for the increasingly localized subsequent chapters, that will
progressively shift the focus from a statistics-heavy narrative, to more
nuanced discussions on interpreting the data on a more musical level.

This chapter opens with looking at general trends across the entire set
of minor mode fugues in the WTC I to establish key division on a macro
scale. From the most general level of analysis, I will demonstrate a
division of the minor mode fugues into two rough categories, Pythagorean
and Meantone, which is not only a phenomenon that arises through the
tuning system itself, but more importantly a division that informed
Bach's approach to composition within each category--i.e.,
compositions written within the Pythagorean class possess compositional
qualities that make them more favorable to their class even when
compared to Meantone compositions modulated to the same keys, and vice
versa. The work done in this chapter will then set up the analyses to be
performed in the next chapter, which will then narrow the focus to
examine these two sets of minor mode fugues--Pythagorean and
Meantone--separately to further parse out differences amongst the
individual keys contained within each.

This successively movement from general to local is a trend that I will
continue throughout the entire portion of analysis; following these
initial two chapters on more global distributions will be a final
chapter focused more upon local structure of intervals and their usage
in relationship with fugal elements--primarily the subject and
countersubject matter of a fugue--and detailed analysis of individual
fugues to demonstrate that these trends are indeed fractalized, that is
to say, that the involvement of temperament in the hand of composition
permeates its every structural level. This consistency at each level of
scope is imperative in that it not only establishes statistical
significance of the data, but also provides a sort of logical pathway
that we can then retroactively trace to determine the musical forces
involved in shaping these distributions, which in turn illustrate the
wider musical significance of statistical trends.

\subsection{Outline of Chapter
Analysis}\label{outline-of-chapter-analysis}

All of the data presented in this chapter--and likewise, most of the
data provided in all three analyses chapters will be presented in forms
of various types of plots to aid in a more straightforward visualization
of the information provided. In cases involving the comparison of
different pieces, all plots have been normalized to account for duration
variations within pieces, to ensure that the length of the piece does
not become an artifact that could insidiously skew the data. Each plot
will be explained and interpreted both from a statistical as well as
musical point of view, at each points raising discussion as to the
connection between the numbers being presented and the greater possible
musical implications underlying the particular distribution and figures.
Below is the following outline of the analysis and types of graphs
presented in this chapter:

\begin{enumerate}
\def\labelenumi{\arabic{enumi}.}
\tightlist
\item
  Temperament "fingerprint" plots for composition-independent key
  visualization: Bar plots charting the distribution of interval cents
  contained in each minor key in WTC I to visualize how temperament
  inherently affects key independent of composition
\item
  KDE (Kernel Density Plots) for composition-independent and
  composition-dependent key visualization: These graphs map out the
  distribution of interval purity (cents from just) as a probability
  density function for each individual minor fugue and their
  transpositions. The goal of these plots is to:

  \begin{enumerate}
  \def\labelenumii{\arabic{enumii}.}
  \tightlist
  \item
    Visualize temperament's effect on key in terms of purity profile,
    and be able to also visualize how keys systematically relate to each
    other given the well-tempered tuning system.
  \item
    In the case of transposition KDE plots, determine whether or not the
    inherent properties of purity and impurity of intervals dictated by
    temperament are more than just confined to pure temperament, but are
    mirrored in the actual corresponding composition themselves. To
    demonstrate a connection between composition and temperament, we
    will apply the test of transposition, in that in pieces transposed
    to the same arbitrary key, we will witness keys and their close
    neighbor key and temperament group (Pythagorean, Meantone) retaining
    their original properties of purity profile in comparison to more
    disparate keys, or keys belonging to the other group. If temperament
    were merely an artifact, with no influence upon the underlying
    composition itself, we should see a random effect when compositions
    are all transposed to a certain arbitrary key, instead of trends and
    the retention of certain patterns.
  \end{enumerate}
\end{enumerate}

Following this outline for analysis, the resulting outline of the goals
to be achieved are as follows:

\begin{enumerate}
\def\labelenumi{\arabic{enumi}.}
\tightlist
\item
  Illustrate on a general scale the raw effects of well-temperament on
  key profile to develop an intuition of the characteristics of each
  key, and the emergence of two main groups of key type, Pythagorean
  (characterized by pure fifths and narrow thirds), and Meantone
  (characterized by narrow fifths and close-to-pure thirds).
\item
  Determine if these relations between temperament and key are more than
  just a pure effect of the temperament itself, and furthermore
  reflected in the compositional makeup of a piece written for a
  particular key.
\item
  Determine whether or not there is a specific directional domain in
  which we can observe stronger temperamental effects on composition.
\end{enumerate}

\subsection{A Word on Vertical vs. Horizontal
Domains}\label{a-word-on-vertical-vs.-horizontal-domains}

Before I delve into the analysis and discussion of the data and plots,
it serves to clarify that I will be presenting horizontal plots and
vertical plots in more or less a side by side fashion, in many instances
providing the results of a search on the horizontal front, followed by
the same search on the vertical dimension, especially for the initial
global plots. The motivation behind this simultaneous presentation of
both the dimensions of data is largely due to perception: while these
domains are easily distinct axes on a pure merit of their auditory
properties, the aggregate musical effect, and the perception of the
musical structure that emerges from the combination of the both domains
is one that is deeply intertwined and interactive. Furthermore, in
Bach's case, and in the case of a fugue, the relationship between the
vertical and horizontal dimensions of musical composition can be very
systematic, and guided by rules of counterpoint. In the Baroque
tradition, the horizontal domain can roughly be seen as more of an
primary, motivating variable, with the vertical structure rising from
the combination of independent, underlying linear structures on the
horizontal domain. Still, to declare one domain as a purely independent
variable, and the other dependent would be a misrepresentation, as the
relationship between the two variables are more accurately a two way
street; the vertical domain and contrapuntal rules dictate the
possibilities of vocal combinations, therefore constraining the
horizontal domain, and creating a sort of musical feedback loop between
the two interacting dimensions.

As the data becomes increasingly focused and localized in this
dissertation, however, more of the intervals being analyzed and focused
upon will be from the horizontal dimension--with increased
parameterization through limiting material to subject and countersubject
material; while the increase focus upon the horizontal domain is
partially due to the complication of the task of reliably coding the
vertical searches, as the limitations placed by the search code makes it
trickier to extract relevant data from noise, it is more primarily due
to the dominance of linear structure over vertical structure in the
fugal genre, making horizontal structures more apparent, and easier to
parse out (as we will observe from the results of this section). From
these horizontal structures, much can also be deduced about the vertical
domain; however, this is not to discount the importance of vertical
structure in these fugal works, rather, selecting a domain in which we
have more control over the identifying salient intervals over noise, as
well as being able to control for hierarchical structure, will provide
more reliable data that most closely mirrors our auditory experience of
the underlying complex musical structure.

    \section{Global Tempered Key Plots}\label{global-tempered-key-plots}

    This section serves as an overview and illustrative aid to help in the
visualization of well-temperament, the variety of intervals that it
creates, its relationship to key, and how this progression creates
smooth changes when cycling through the circle of fifths. This section
serves to demonstrate how, by virtue of temperament alone, keys each
have their own separate and unique interval profiles (i.e.,
distributions of tempered intervals), and just how much remote ends of
the key spectrum differ from each other in terms of these interval
profiles.

Through looking at bar charts of intervals for each minor fugue, we can
also see the emergence of two rough categories of keys, keys that
resemble the medieval Pythagorean model of tuning, and keys that
resemble the meantone model of tuning by virtue of specific tempered
intervals reminiscent of their respective historical tuning traditions.
For this dissertation, I will come to recognize and refer to these two
groups of key by name of the historical tuning system that they
respectively resemble: "Pythagorean keys", or the "Pythagorean group"
referring to the keys that more resemble Pythagorean tuning, and
"meantone keys" or "meantone group" for the keys that more closely
resemble the meantone temperament scheme. These two categories will be a
recurring theme that will be explored throughout the various levels of
analysis of this thesis, and how Bach treatment of composition varies
according to the key category that he is writing for through musical
forces such as choice of interval, vocal texture, and note duration, and
how these elements are systematically connected to and motivated by
temperament.

Before we delve into the specifics of Pythagorean and Meantone
categories of keys through and their respective interval characteristics
in regards to temperament, let us visualize the intervallic forces made
available through well-temperament. The following chart shows the
distinct variety of intervals within the octave created through the
Werckmeister III tuning system; the leftmost column (not counting the
indices in bold) indicates the semitone class (the number indicating the
degree of separation between the two notes) the center column returns
the corresponding value in cents, and the rightmost column gives the
calculation of that particular interval's cent value from its just
counterpart. For example, for the semitone class 3, which is largely
dominated by (but not limited to) the minor third, we can see that
Werckmeister III contains four varieties of the interval, ranging from
294 cents all the way to 312 cents. The widest minor third, at -4 cents
from just, is very close to pure, and is associated with the meantone
tradition of close to pure thirds, and in Werckmeister III, is assigned
to the diatonic keys (in this dissertation, what we will also refer to
as the meantone keys), comprising the tonic third of a minor, and the
dominant third of d minor. The narrowest minor third, at -22 cents from
just (the difference of exactly one syntonic comma), is the Pythagorean
minor third, named after the Pythagorean tuning system from which it is
derived, and in Werckmeister III, found in the tonic thirds of the
remote, chromatic keys (Pythagorean keys) of b-flat minor, d-sharp
minor, and f minor.

The following chart shows all available varieties of intervals found in
Werckmeister III--40 in total, counting the unison. In contrast, equal
temperament only contains 12 different distinct varieties of intervals,
as each interval comes only in one size, and besides the unison, no
interval is exactly pure.


\begin{singlespace}
\begin{table}[H]
\centering
\small
\begin{tabular}{|lrrr|}
\hline
\textbf{{}} & \textbf{ semitones} & \textbf{ cents} & \textbf{ cents from just }\\
\hline
0  &          0 &      0 &                0 \\
1  &          1 &     90 &              -22 \\
2  &          1 &     96 &              -16 \\
3  &          1 &    102 &              -10 \\
4  &          1 &    108 &               -4 \\
5  &          2 &    192 &              -12 \\
6  &          2 &    198 &               -6 \\
7  &          2 &    204 &                0 \\
8  &          3 &    294 &              -22 \\
9  &          3 &    300 &              -16 \\
10 &          3 &    306 &              -10 \\
11 &          3 &    312 &               -4 \\
12 &          4 &    390 &                4 \\
13 &          4 &    396 &               10 \\
14 &          4 &    402 &               16 \\
15 &          4 &    408 &               22 \\
16 &          5 &    498 &                0 \\
17 &          5 &    504 &                6 \\
18 &          6 &    588 &               -2 \\
19 &          6 &    594 &                4 \\
20 &          6 &    600 &               10 \\
21 &          6 &    606 &               16 \\
22 &          6 &    612 &               22 \\
23 &          7 &    696 &               -6 \\
24 &          7 &    702 &                0 \\
25 &          8 &    792 &              -22 \\
26 &          8 &    798 &              -16 \\
27 &          8 &    804 &              -10 \\
28 &          8 &    810 &               -4 \\
29 &          9 &    888 &                4 \\
30 &          9 &    894 &               10 \\
31 &          9 &    900 &               16 \\
32 &          9 &    906 &               22 \\
33 &         10 &    996 &                0 \\
34 &         10 &   1002 &                6 \\
35 &         10 &   1008 &               12 \\
36 &         11 &   1092 &                4 \\
37 &         11 &   1098 &               10 \\
38 &         11 &   1104 &               16 \\
39 &         11 &   1110 &               22 \\
\hline
\end{tabular}
\caption{All Werckmeister III Intervals in Cents }
\end{table}
\normalsize
\end{singlespace}
    Just for comparison, here are the intervals, in cents, and cent from
just, for equal temperament. Because there only exists one instance of
interval for interval type, keys in equal temperament, from the point of
view of interval sizes, are mathematically identical across
transposition. Furthermore, besides the unison class, none of the
intervals in equal temperament are just.


\begin{singlespace}
\begin{table}[H]
\centering
\small
\begin{tabular}{|lrrr|}
\hline
\textbf{{}} & \textbf{ semitones} & \textbf{ cents} & \textbf{ cents from just }\\
\hline
0  &          0 &      0 &                0 \\
1  &          1 &    100 &              -12 \\
2  &          2 &    200 &               -4 \\
3  &          3 &    300 &              -16 \\
4  &          4 &    400 &               14 \\
5  &          5 &    500 &                2 \\
6  &          6 &    600 &               10 \\
7  &          7 &    700 &               -2 \\
8  &          8 &    800 &              -14 \\
9  &          9 &    900 &               16 \\
10 &         10 &   1000 &                4 \\
11 &         11 &   1100 &               12 \\
\hline
\end{tabular}
\caption{All Equal-tempered Intervals in Cents }
\end{table}
\normalsize
\end{singlespace}
    Returning to the Werckmeister III well-temperament chart, a few things
of significance to note are:

\begin{itemize}
\tightlist
\item
  Omitting the unison, of the 11 interval groups ranging from m2 to M7,
  4 of those groups contain instances of the interval in its pure, or
  just, form. These intervals are the perfect intervals (P4 and P5), and
  the Major second and its inversion, the minor seventh. The purity of
  these intervals are reminiscent of the older, Pythagorean tuning
  tradition that was based on the retention of the purity of Perfect
  intervals (fourths and fifths, at 3:2 and 4:3 ratios respectively),
  and as a result from stacking just fifths, Major seconds as well
  (9:8).
\item
  There are two groups of fifths and fourths, just intervals, and
  intervals at ±6 cents from just. In Werckmeister III, 8 out of the 12
  available fifths are just, and as we will observe later, most of the
  just fifths are contained in the remote, Pythagorean keys (meaning
  these fifths fall on prominent intervals within a key's tonal
  construction, such as the tonic, dominant, and subdominant scale
  degrees), while the impure intervals are assigned to the diatonic,
  meantone keys.
\item
  The most dramatic difference in terms of cents from pure for intervals
  found in Werckmeister III occur primarily at the imperfect intervals,
  that is, the thirds (Major and minor), and sixths, at ±22 cents, or
  the difference of a syntonic comma. This wide Major third at +22 cents
  from just (or 81:64 in terms of ratios) and narrow minor third at -22
  cents from just is known as the Pythagorean (Major or minor) third, as
  this type of major third is derived from the superimposition of 4 just
  fifths. These accordingly occur predominantly in the remote,
  Pythagorean keys of b-flat minor, d\# minor, and f minor. In contrast,
  the closest the imperfect intervals get to purity is 4 cents (4 cents
  sharper for the Major third, and 4 cents flatter for the minor third);
  this closest-to-pure third occurs as the tonic third of a minor, with
  its neighboring keys (diatonic, meantone group) centered on similarly
  closer-to-pure thirds of -10 cents.
\item
  the semitone also has a similar cents-from-just range as the imperfect
  intervals, ranging from -22 to -4 cents. The narrowest of the
  semitones, at -22 cents, is equivalent to the Pythagorean minor
  semitone, or the Pythagorean limma. Werckmeister III assigns these
  narrow semitones to harmonically functional scale positions in the
  Pythagorean fugues, and a subsequent section in this chapter will
  analyze how the scale degree placement of these semitones accentuate
  linear direction through voice leading and leading tones in the
  melodic/horizontal domain in the Pythagorean minor fugues, and comes
  to define the modal nature of these keys.
\end{itemize}

It is important to note here that, while Werckmeister III is the system
that I will be using as representative of well-temperament for my
analysis, it is by no means the only existing system of historical
well-temperaments, as there are indeed a vast body of others. With this
said however, these different varieties of well-temperament still
preserve in common the fundamental principles of the system itself, that
is, it is a system of the gradual alteration of thirds from pure to
Pythagorean (dictated by the syntonic comma), with the purer end of the
spectrum reserved for the diatonic keys, and alterations to the interval
increasing with progression through the circle of fifths. Thus, while
different systems of well-temperament might present a slightly different
collection of cent values for intervals, this fundamental principle of
the purity of the diatonic keys (in terms of thirds), and smooth
modulation, is unchanged, and the relationship between temperament and
key, as well as the relationship between keys, is not, on a macro scale,
drastically altered by changing which system of well-temperament we
chose to employ.

    \subsection{Temperament "Fingerprint"
Plots}\label{temperament-fingerprint-plots}

Having provided the brief overview of the variety of intervallic forces
available through well-temperament, we can now turn to visualizing how
this translates into the interval profile for keys through looking at
key maps; because this dissertation is focused on minor keys, we will
only be looking at key maps for the 12 minor fugues in Book I. Unlike
its predecessors of Pythagorean tuning and the various meantone
temperaments (as well as equal-temperament), well-temperament's system
of containing a variety of different sizes of each interval yields a
system that makes the scales of each key mathematically and physically
different and unique from one another. Because the shift of interval
size occur in gradients, this also means that shifts between keys happen
in a systematic and gradual fashion as well, resulting in keys with
closer harmonic relation to bear more similarities to one another than
remote keys with little harmonic relation. The musical implications of
this system are vast, which will be discussed in more of a theoretical
and postulatory manner in this section, setting up questions and
hypotheses that will subsequently be addressed and answered through the
analysis in the chapter's subsequent sections.

The key interval plots presented in this section will all be plots of
intervals on the vertical domain; the motivation behind this is that
interval usage on the vertical domain is more variegated, and hence more
balanced, which provides us with a better tool for interval
visualization. The horizontal domain, which is largely dominated by
smaller intervals and stepwise motion, will contain an inherent skew to
the shape of the distribution, as most of these smaller
intervals---both Major and minor seconds, and minor thirds in
Werckmeister III---range on the flat end of the spectrum (their
cents vary from just, to flatter than just), as the wider intervals of
Major sixths, and both Major and minor sevenths range from just to
sharper than just. So while much of the dissertation's analysis will
eventually focus on the horizontal domain for virtue of its more
reliably extractable structure, the more even distribution of intervals
along the vertical domain make it the more suitable option for
visualizing the general relationship between key and temperament.

The anatomy of the key interval plots presented in this section are as
follows:

\begin{itemize}
\tightlist
\item
  Intervals presented in their simple form, and are laid out on the
  x-axis in accordance to their semitone value from 0 (unison/octave) to
  11 (Major seventh), and cent value. On the y-axis are measurements in
  units of duration, normalized by the total duration of the piece.
  While this normalization factor is not as important for the graphs in
  this section, it will prove vital for plots that require looking at
  distributions of intervals compiled across various compositions, as
  without normalization, we can run into the danger of confounding the
  weight of an interval's actual presence with the sheer duration of the
  piece that the interval is taken from.
\item
  Intervals of the same type (m2, P4, etc.) are grouped and coded by
  color. Transparency levels (alpha level) of bars correspond to their
  purity, with level of transparency scaling according to an interval's
  cents from its just counterpart. The more transparent the bar, the
  closer to pure its corresponding interval it is; the more opaque the
  bar, the more distant from pure. Transparency levels reflect absolute
  values, so an interval that is -22 from just will have the same
  opacity level of an interval that is 22 cents from just.
\end{itemize}

\subsubsection{Pythagorean and Meantone Examplar Plots and Temperament's
Relationship with the Division of Minor
Keys}\label{pythagorean-and-meantone-examplar-plots-and-temperaments-relationship-with-the-division-of-minor-keys}

The following graph places side by side for comparison the interval maps
of b-flat minor and a minor. These two keys are in many way
exemplar\footnote{This is somewhat of a simplification, as we will show in subsequent chapters that a singular meantone exemplar key is a fuzzier notion, and shifts depending on which defining interval one chooses to focus on, nonetheless, this serves as a good elementary illustration of the temperamental polarization between the Pythagorean (remote) and meantone (diatonic) groups of keys.}
keys, representing the two extreme ends of the Pythagorean-Meantone
spectrum:



\begin{figure}[H]
\vspace{1.5em}
    \centering
    \adjustimage{max size={0.9\linewidth}{0.9\paperheight}}{analysis_1_files/analysis_1_17_0.png}
    \caption[Examplar Pythagorean (b-flat minor) and Meantone (a minor) Temperament Plots. ]{ Examplar Pythagorean (b-flat minor, left) and Meantone (a minor, right) Temperament Plots. Note the dramatic difference in the distribution envelopes.}
\end{figure}    From the two graphs laid out side by side, it is evident the marked
difference of the Pythagorean key compared to the meantone one, even
through just a general observation of the shape of each distribution's
envelope. The envelope of the Pythagorean key is a great deal more
extreme than that of the meantone key, with intervals either belonging
to the pure variety, such as in the case of fifths, fourths, Major
second and minor seventh, or the extreme sharp/flat (Pythagorean)
variety, in case of the minor second, minor third, Major third, and
minor and Major sixths. In contrast, the meantone key's treatment of
pure/tempered interval distribution is far more egalitarian, with
intervals evenly distributed amongst the spectrum of pure and tempered
on the domain of fifths and fourths, and the imperfect intervals leaning
towards pure (as to be expected in line with the meantone tradition),
but not nearly as dramatically so as the Pythagorean group's sharp
predilection for the extreme tempered versions of these intervals.

Musically speaking, this theoretically translates to the Pythagorean
keys being more extreme in their minor modality, as thirds are more
narrow, increasing the overall "minor" quality of these keys, and the
perfect intervals are pure, creating not only more sonic resonance in
these keys, but also allowing for more tonal stability in these keys,
granting the musical acceptability for pieces to move at slower speeds,
while continuing to retain linear motion. This is further potentially
accentuated by the narrow semitones at 90 and 96 cents, which
facilitates voice leading and the resolution power of leading tones.
Additionally, the dramatic difference between major and minor quality
for the imperfect intervals (namely the thirds) result in more dramatic
color shifts when toggling between major and minor key centers within a
particular piece. The results of such a polarization of the minor keys
and their major counterparts can unfold in interesting manners, and
approach to composition involving modulation in these Pythagorean minor
modes may involve more deliberate caution in regards to shifts to
parallel and relative major modes. One such implication is that a
composition in a Pythagorean minor key may tend to shy away from
frequent, quick bouts of modulation into Major modes, especially when
retaining the same motivic material, as this would cause too dramatic a
disruption in the overall aesthetic cohesiveness of the piece, but
shifts that do occur into the major realm would be done in a manner that
accentuates the contrast for a desired effect.

If the Pythagorean keys, with its built in ability for slower speeds and
longer note durations through the aid of just perfect intervals, and
narrow thirds and semitones serve to accentuate the effect of a sort of
19th century exemplar of "minor" modality and a tendency towards the
more depressive emotions, the meantone keys achieve a sort of opposite
musical effect with the purity of the thirds against tempered fifths.
The purity of the minor and major thirds closes the gap between these
two intervals, making shifts between major and minor modes within pieces
more even keeled, furthermore, the wider minor third, combined with the
wider semitone, makes the feeling of the minor third more stable, and
decreases its tendency to resolve inwards. Few just intervals on the
domain of the fifths decreases sonic resonance, which favors faster
motion and shorter note durations. These aggregate forces result in an
overall faster, more upbeat minor mode of a distinctively different
flavor than its Pythagorean counterpart, and a minor mode that is
reminiscent of a tradition of minor modes as dignified and stately,
rather than evocative of more depressive emotions.

While all of these musical implications are, at this moment of analysis,
still predominantly theoretical in nature, they comprise a bulk of the
theories that I will test in a concrete manner through looking the
various distributions and charts in the subsequent sections and chapters
of this dissertation. However, through just looking at the shapes of
these initial distributions, certain musical relationships between
temperament and key are already coming to the forefront, and it is up to
the subsequent analysis of this dissertation to test whether or not
these temperament and key trends are reflected in and correspond with
the underlying musical structures, and whether or not there actually
exist these systematic connections and correlations between these
professed musical elements (note duration, semitone placement and voice
leading, predilection towards purity of fifths) that support these
effects of temperament.

Another salient thing of note is that, while these graphs are taken
purely from the actual interval distributions of the B-flat minor and a
minor fugues, they are not exceptional to these particular instances of
these unique pieces. In other words, the overall envelope of the shape
is composition-independent; B-flat is the actual key that maximizes the
usage of pure fifth and Pythagorean minor thirds, making it the most
extreme example of a minor fugue of the Pythagorean variety (pure
fifths, narrow thirds), while a minor is the key that maximizes the
usage of the closest-to-pure third, making it a model meantone minor key
(close-to-pure thirds, narrower fifths).

It is also important to observe that, while the pureness of the fifth
and narrowness of the third lines up pretty well on the Pythagorean end
of the spectrum, that is to say, that the extreme keys of B-flat minor
and D\# minor both contain the highest instances of pure fifths as well
as narrow thirds, the synchronization between pureness of thirds and
flatness of fifths do not line up as exactly in the meantone minor keys.
In terms of pureness of thirds, the peak falls at the key of a minor,
however, the peak for tempered fifths falls two keys away on the circle
of fifths, at g minor. This staggering creates a degree of fuzziness in
the meantone keys in regards to which key fully exemplifies the traits
of meantone minor (which we will unpack in greater detail in the section
regarding scale degree analysis in the next chapter); however, it still
remains true that the region of diatonic keys (i.e. c minor - e minor,
with c minor and b minor being on the boundaries) embody roughly the
region of meantone minor, while the chromatic keys (i.e. f minor to
f-sharp minor, with b-flat and d-sharp at the pinnacle, and f minor
lying on the boundary in terms of purity of fifths) comprise the
Pythagorean group.

\subsubsection{All Key Area Plots and Averaged
Plots}\label{all-key-area-plots-and-averaged-plots}

The following two tables now chart the interval maps of all minor keys,
progressing through the circle of fifths in retrograde motion (circle of
fourths, starting from f minor and ending on c minor). This ordering of
keys in retrograde enables us to hit the Pythagorean keys first, and
progresses into the meantone group, but does not have any other
significant meaning beyond that; in all other regards, prograde cycling
would have been just as acceptable. In actuality, these keys do not lie
on a line but rather a circle, with c minor at the end wrapping around
the f minor at the beginning. With all the graphs presented in
succession, we can now observe the gentle gradient and shift as the
shape of the envelope gradually morphs from the extreme, jagged outline
in the Pythagorean keys to the smoother curves in the meantone region,
with peak effects at b-flat minor and a minor.

The first set of the tables below will chart actual fugues, and the
second set will display actual fugues in the left column, compared to a
chart of the composite of all fugues transposed to that given key, and
averaged and renormalized. These envelope of these graphs represent the
expected envelope of these keys, in other words, how we would expect a
key to look like in the system of well-temperament independent of
additional compositional elements that may tweak the shape of the
distribution. Evoking an example to illustrate more clearly what is
going on in these averaged graphs, the averaged graph of b-flat minor
takes every fugue that Bach wrote for the WTC I (which is 12 fugues, one
in each minor key), transposes each one of them from their original keys
to the key of b-flat minor, and combines these new 12 fugues, all
pitched in b-flat minor by adding up the normalized duration of each
interval corresponding to their cent value (the normalization is
important to prevent the varying lengths of pieces from producing
artifacts in the resulting distribution), and again re-normalizing at
the end.



\begin{figure}[H]
\vspace{1.5em}
    \centering
    \adjustimage{max size={0.9\linewidth}{0.9\paperheight}}{analysis_1_files/analysis_1_20_0.png}
    \caption{Minor Fugues Temperament Plots (Pythagorean Keys), Actual Fugues. }
\end{figure}

\begin{figure}[H]
\vspace{1.5em}
    \centering
    \adjustimage{max size={0.9\linewidth}{0.9\paperheight}}{analysis_1_files/analysis_1_22_0.png}
    \caption{Minor Fugues Temperament Plots (Meantone Keys), Actual Fugues. }
\end{figure}


\begin{figure}[H]
\vspace{1.5em}
    \centering
    \adjustimage{max size={.63\linewidth}{0.9\paperheight}}{analysis_1_files/analysis_1_25_0.png}
    \caption[Minor Fugues Temperament Plots (Pythagorean Keys), Actual Fugues and Averaged Fugues Compared. ]{Minor Fugues Temperament Plots (Pythagorean Keys), Actual Fugues (left) and Averaged Fugues (right) Compared.}
\end{figure}


\begin{figure}[H]
\vspace{1.5em}
    \centering
    \adjustimage{max size={.63\linewidth}{0.9\paperheight}}{analysis_1_files/analysis_1_28_0.png}
    \caption[Minor Fugues Temperament Plots (Meantone Keys), Actual Fugues and Averaged Fugues Compared. ]{Minor Fugues Temperament Plots (Meantone Keys), Actual Fugues (left) and Averaged Fugues (right) Compared.}
\end{figure}    The important result that is achieved in these graph of averages is that
it decorrelates possible effects from other musical structures with key,
essentially filtering out these factors, and allowing us to look only
upon the effect that pure temperament has on key. The implications and
usefulness of such a function is twofold: it allows us to check whether
or not trends that we are attributing to temperament are true
correlations, and not confounded by other, hidden variables, and perhaps
more importantly, it allows us to test in the opposite direction:
through the comparison of actual graphs against averaged graphs, the
residual differences give us a powerful tool in testing the relationship
between temperament and other compositional elements, determining
whether or not the musical elements of the composition itself are
working in conjunction with temperament to achieve certain desired
effects. These averaged graphs, as well as the transposed graphs in
isolation will serve as an important litmus tests for correlative power
throughout this dissertation's analysis sections, especially the next
section using Kernel Density Estimate plots, enabling us to check
whether or not certain trends that we profess to arise through musical
elements are indeed present, instead of merely just a result from tuning
itself.

\subsection{Observations, Section Summary, and
Conclusions}\label{observations-section-summary-and-conclusions}

The key maps in this section, both of the actual fugues and the averaged
fugues, display the effect of the two rough groups of key types,
Pythagorean and Meantone, that emerge through well-temperament, and how
the transition from key to key via the circle of fifths is a gradual
effect. Because these effects of gradual, systematic transition of the
shape of the distribution across the circle of fifths, and the side by
side comparison of the actual fugues and the averaged fugues bears a
similar profile for each key, we can conclude that a key's particular
distinct interval distribution is a direct result of temperament, and
furthermore, that the system of well-temperament divides keys into these
two rough groups of Pythagorean and meantone.

Verifying that the shapes of these actual graphs and the averaged graphs
to be similar is an important first step in establishing temperament's
effect on key, but closer examination is necessary to codify the effect
temperament exerts on actual compositional forces, which will be the
driving force behind all of the analysis for the remainder of the
dissertation. In the next chapter, we will revisit portions of these key
maps (as well as portions of horizontal key maps), breaking them apart
and examining closer the distributions of some of the intervals this
time across the function of key area, but a closer look at the graphs in
this section already show differences between the actual graphs and
their averaged counterparts. Using the perfect intervals as an example,
fifths and fourths of a specific group's home variety (pure in the case
of Pythagorean, and tempered in the case of meantone) are maximized and
more favored in their actual fugues as opposed to their averaged
counterpart.

The following are some immediate observations from the charts:

\begin{itemize}
\tightlist
\item
  In the Pythagorean keys, we can visibly, and consistently, observe
  higher peaks in the actual keys as opposed to the averaged
  counterparts for the pure variety of fifths for all keys (6 out of 6
  keys, spanning from f minor to f-sharp minor). 4 out of the 6 keys
  have lower values for the tempered intervals in the actual pieces
  compared to their averaged counterparts, and 5 out of 6 keys have a
  higher pure-to-tempered ratio in for the actual fugues, f minor being
  the exception.
\item
  For fourths in the same Pythagorean keys, 4 out of the 6 keys have
  higher peaks for pure fourths in their actual keys, with these same 4
  out of 6 keys having a higher pure-to-tempered ratio as well in
  comparison to the averaged fugues.
\item
  For the meantone keys, the trend is reversed for the fifths - for the
  three keys in which the average graphs' tempered interval heights
  exceed pure (c, g, d), two out of the three contain a greater ratios
  of tempered to pure on their actual keys, and the two keys in which
  pure fifths exceed tempered on the averaged graphs (a and e), a
  minor's actual distribution finds the heights of the bars to be pretty
  much equal, and e minor actually reverses the trend, favoring tempered
  intervals over pure.
\item
  For minor thirds on the extreme Pythagorean keys of b-flat minor and
  d-sharp minor, both of the actual graphs have higher peaks on the
  Pythagorean (-22 cents) variety as opposed to the averaged graph, as
  well as higher overall ratios for Pythagorean minor thirds (-22 cents)
  against all other varieties of thirds.
\end{itemize}

All graphs are completely normalized, so the heights of bars essentially
represent ratios, and the y-axis set to the same limit, making
comparisons on height fair and not skewed by any other factors that
might introduce differences in the data, such as overall duration of the
piece. Again, segments from these charts will be revisited in a later
section, along with exact numbers of the bar heights and ratio values.

These differences between actual graphs and their averaged counterparts,
especially if consistent, systematic, and span across entire groups of
fugues, are important, because if we believe in a null hypothesis that
temperament does not exert any effect on the underlying compositional
structure and elements of a piece, we should not expect to see any
patterned trends between differences of an temperament graphs of
original fugues and the expected graphs of their keys, especially ones
that adhere to our expectations of key traits given their key
group--instead, the data should be randomized. However, what we are
seeing here (and what this dissertation will continue to demonstrate),
is that when compared against a graph of expected averaged values, the
original piece accentuates temperamental traits and idiosyncracies of
their keys. Indeed, one of the keys to rejecting the null hypothesis
lies within the logic of this transposition test: if we can establish
that there are systematic traits of original keys that are retained
across transposition (or demonstrated against expected values), and that
these traits can be reliably linked to musical elements that are
relatable, and correlate in some manner to the structure of
well-temperament, then we have a strong reason to believe that
temperament is not a mere independent artifact, but embedded in
compositional structure of the piece itself. We will look at this is
greater detail in clarity in the section now to follow.

    \section{KDE Plots and Histograms}\label{kde-plots-and-histograms}

This section explores the profile and distribution of pure and tempered
intervals in keys through the usage of KDE plots and weighted
histograms. KDE plots, or Kernel Density Plots, are closely related to
probability density functions (PDFs) and histograms, both of which
characterize the total distribution of a dataset as a normalized
probability function, with the total area under the curve (or in the
case of a histogram, the total area of the histogram bars) summing up to
1. KDE plots and histograms are useful because they allow for intuitive
visualization of the shape of the data; I have employed both in this
section because the continuous nature of the KDE plots allow for easier
visualization of the data, but the KDE plots in Python do not have a
function for weighting, which is necessary for accounting for note
duration in the case of the dataset that I will be looking at, so I will
also be looking at weighted histograms for increased accuracy of the
data. In some instances, such as the horizontal domain, the trends are
strong enough that the additional weights of duration will not effect
the outcome of the data much, but in certain vertical domain graphs, the
increase of accuracy of the histograms might present some salient
modifications.

Similar to the plots seen in the previous section, the KDE plots and
weighted histograms are a representation of all the intervals present in
a single fugue, and furthermore, a representation of that key's
particular interval profile in well-temperament. While the interval maps
in the previous section offer a more specific breakdown in terms of
actual intervals documented and their actual cent values, the KDE plots
and histograms in this section are a little more coarse, as they only
chart cents from just. Because of this, there is less information in
regards to parsing out certain intervals from others, however, this is
traded for a more apparent and intuitive representation of the key's
pureness/temperedness balance. Furthermore, these KDE plots and
histograms of cents from just allow for a more effective means of
comparing multiple keys on the same graph, as using cents as a basis for
the x-axis would be presenting too much information at once.

It is also important for these KDE plots and histograms that we look at
the total representation of all intervals present in a specific piece as
a whole, or at least close to total, as to get a correct reading on the
actual balance and distribution, devoid of any confounding artifacts. If
we consider the KDE plot and a histogram as measure of percentage, we
can greatly misrepresent the data if we choose only a subset of
intervals, or only one interval to look at, as we will have no way of
being able to retain a sense of how that interval fits into the piece as
a whole. In this following hypothetical example, if we isolate the
fifths and examine them with KDE plots, we will see a distribution with
peaks only at 0 cents and 6 cents from just. If we transpose all fugues
into b-flat minor, we may see a high peak at 0 cents for the e minor
fugue, and from that make a deduction that the e minor fugue, when
transposed to b-flat, may maximize pureness for the intervals of fifths.
However, because the KDE plot forces the area of the curve to sum to 1,
we have no way of knowing how much of the original e minor piece was
occupied by fifths; if fifths were a negligible and relatively
unimportant interval for the e minor fugue to begin with, it may very
well be that the peak of that pure fifth, in comparison with other
fugues, is actually very low, but we have no way of taking that into
consideration when we look at this KDE plot charting isolated fifths. Of
course, there is value and purpose as well in looking at isolated
comparisons of pure vs. tempered intervals, and such charts will be
explored in later sections of this chapter and the next, but it is
crucial to establish earlier patterns on a macro scale before moving on
to looking at the data in a more localized, specific fashion.

I will be presenting KDE plots and histograms for the vertical domain
first, followed by the horizontal domain. Similar to the previous
section, plots will be presented for all keys, but this time
superimposed upon one graph for better comparison. Tables of transposed
graphs will then follow, with the purpose to determining whether or not
any patterns can be detected after transposition. All bandwidths of KDE
plots will be set to the same level, as will bin sizes for histograms to
ensure that all plots are consistent with one another, and not
confounded by any visual artifacts from inconsistent binning or
bandwidth levels.

This section will be mainly focused on the 0 cents from just (absolutely
pure) category of intervals, partially because this category is the most
prominently featured class in the KDE plots and histograms, and
therefore the easiest and most reliable class to observe and detect
patterns and trends for in a comparison setting. There also exists a
deeper and underlying musical motivation for focusing on this class of
interval, in that the category of absolute just intervals plays an
integral role in terms of preservation of consonance across any key.
Subsequent sections will closer examine the functional effect,
placement, and musical importance of these just intervals within the
setting of specific pieces, but the goal of this present section is
focused upon providing more evidence to reject a null hypothesis that
temperament does not have inherent and systematic ties to compositional
strategy through demonstrating the existence of consistent trends after
transposition.

While these primary sections may be less musically exciting, per se,
than the latter sections in this dissertation that will deal more
directly with localized and specific musical structures and elements,
this section is especially crucial in establishing the existence of a
primary correlative relationship between temperament and the other
compositional elements. Once this relationship is cemented, the final
task is to identify these contributing compositional forces, and
determine whether or not there exists strong and compelling enough
theoretical reasons that links the musical functions of these
compositional elements to the acoustical properties dictated by the
temperament.

Lastly, it should be said that, while the most pronounced trends and
correlations in this section will be found on the horizontal domain, the
vertical domain will be presented first as to provider a smoother
transition from the previous section. Indeed, throughout the analysis in
this dissertation, we will continue to observe consistently clearer
trends and structures in the horizontal domain. This may not necessarily
mean that the vertical domain does not contain these structures in such
a pronounced fashion; it could be that the existing framework for the
vertical dimension might be constructed in too general a fashion, making
these structures more difficult to parse out. Alternatively, more
apparent and readily parseable structures in the horizontal domain may
be linked to the predominant linear approach to compositional structure
of the Baroque era, especially in the case of the highly contrapuntal
and linear genre of the fugue, where is dictated by the horizontal
subject kernel. This hierarchical fugal structure, which radiates
outwards from the subject kernel, means that structures on the fugal
level (subject, countersubject) dictate the construction of the
horizontal domain, which in turn directly impact the vertical dimension,
so musical choices in terms of recurring motifs and intervals on the
horizontal domain do not remain isolated and local, but serve to guide
the piece as a whole.

    \subsection{KDE and Histogram Overview
Plots}\label{kde-and-histogram-overview-plots}

\subsubsection{Vertical Dimension Plots}\label{vertical-dimension-plots}

The graph below displays the Kernel Density Plots of our two exemplar
keys, b-flat minor and a minor, coded in blue (b-flat) and a (red)
respectively, superimposed on a singular axis.




\begin{figure}[H]
\vspace{1.5em}
    \centering
    \adjustimage{max size={.8\linewidth}{0.9\paperheight}}{analysis_1_files/analysis_1_34_0.png}
    \caption{B-flat minor and A minor Vertical KDE Plots.}
\end{figure}    The traits of Pythagorean and Meantone mode in well-temperament that was
previously discussed in the prior section are elegantly captured in the
KDE plots. The x-axis charts cents from just, with pure being at the 0
marker in the middle of the axis. For the case of the KDE plots, the
heights and shape of the curve are determined by counts of intervals
(number of instances of occurrence of the interval across the entire
piece), with the actual value on the y-axis reflecting the fact that the
data is normalized so that the area under the curve will integrate to 1.
The histograms presented later in this section will present the same
data, but pass the counts through a weighting function that weights by
the individual interval's duration for an increased accurate
representation of the data, but the overall shape of the histograms and
KDE plots remain essentially the same, as they are fundamentally
charting the same thing.

In this graph, we again observe the marked differences between the shape
of the Pythagorean b-flat minor KDE curve and the Meantone a-minor
curve; b-flat minor is a key of extremes, with most of the area either
gathered under the pure, or the extreme ends of the spectrum at -22 and
22 cents from just. Contrasting this, a minor has its area distributed
more evenly across the spectrum, with a much lower peak at 0, and
falling in an orderly and gradual fashion as it tapers off at the
extreme ends of the spectrum. As we have observed in the graphs from the
prior section, b-flat minor and a minor lie on the very extreme ends of
the temperament spectrum, and we will continue to see this translated in
the KDE plots and histograms of this current section.

This next graph adds two more plots to the axis, the averaged fugue plot
for b-flat minor (i.e., again, all the fugues from WTC I transposed to
b-flat minor, summed, and averaged), and the averaged plot for a minor
as well. The color of the averaged plots correspond to the key that they
belong to, but they are distinguished with lighter (more transparent)
lines. I have also provided the weighted histogram of this same
functions as well. Both plots (KDE and histogram) represent the same
data, with the exception of the histogram being weighted by interval
durations. In other words, one can think of the area under the curve for
the KDE plots as representative of percentage of raw occurrences of an
interval's appearance in a particular piece, whereas the area of the
bars of the histograms for a value represents the percentage of the
total time (i.e. duration) spent on the interval. Because of this, the
weighted histograms are a more accurate depiction of the data, as it
takes into account that, for example, an interval sustained for a
quarter or a half note should be given more weight than an interval
presented as passing material in a sixteenth note passage, and provides
them with these weights according to their duration values, rather than
the KDE plots which represents both these intervals with the same weight
as one count.

While there are cumulative and perceptible differences between the two,
as we should expect, viewing the KDE plot and the histograms side by
side show that the overall shape and trends of the two types of graphs
are comparable, with the KDE plots not only doing a thorough job of
representing the data, but in some cases might possessing advantages
over the weighted histogram in that its continuous nature allows for a
more uninterrupted and simplistic view of the data. In summary, the KDE
plot and the histogram both serve the represent the same data, but just
in two slightly different formats, the KDE being a continuous
representation, and the histogram being discretized. Because each format
possesses its advantages over the other---namely, the continuous KDE
being better for visualizing the overall shape of distributions, and the
discretized histogram containing more accurate values, and allow for
closer examination of individual x-axis values---I will be
presenting both types of graphs in succession for the remainder of the
data provided in this section.




\begin{figure}[H]
\vspace{1.5em}
    \centering
    \adjustimage{max size={.8\linewidth}{0.9\paperheight}}{analysis_1_files/analysis_1_38_0.png}
    \caption[B-flat minor and a minor Actual/Averages Comparison Vertical KDE Plots. ]{B-flat minor and A minor Actual/Averages Comparison Vertical KDE Plots. Actual fugues are indicated by darker lines, averaged fugues indictated by lighter lines.}
\end{figure}


\begin{figure}[H]
\vspace{1.5em}
    \centering
    \adjustimage{max size={.8\linewidth}{0.9\paperheight}}{analysis_1_files/analysis_1_41_0.png}
    \caption[B-flat minor and a minor Actual/Averages Comparison Weighted Histograms. ]{B-flat minor and A minor Actual/Averages Comparison Weighted Histograms. Actual fugues are indicated by darker lines, averaged fugues indictated by lighter lines.}
\end{figure}    Two important observations can be deduced from these graphs:

\begin{enumerate}
\def\labelenumi{\arabic{enumi}.}
\tightlist
\item
  The similar shapes that the averaged plots bear to the plots of their
  actual fugue counterparts corroborate the findings in the previous
  section that temperament exerts a strong effect on key independent of
  composition.
\item
  In both cases, the curve of the actual graph is the more exemplary
  version of the key's characteristic curve as opposed to the that of
  the actual plot. If you observe the blue curve of the b-minor fugue,
  the extreme profile of high peaks at just and Pythagorean Major/minor
  (-22 and 22 cents) are maximized by the actual fugue's curve.

  \begin{enumerate}
  \def\labelenumii{\arabic{enumii}.}
  \tightlist
  \item
    The peak at 0 cents from just is pronouncedly higher for the actual
    b-flat minor fugue (and even more so on the weighted histogram), as
    well as the peak at -22 cents (minor third and minor second), with
    the intermediate dips during the middle values of the spectrum
    around ±6 to ±10 cents lower on both the positive and negative ends
    of the spectrum. These exact mid-values can be more clearly observed
    in the weighted histogram.
  \item
    Adhering to its own end of the spectrum, the gradual, even shape of
    the meantone keys is maximized by the actual a minor fugue's curve,
    with its peak at 0 cents from just, as well as the curve at the far
    extreme ends of the spectrum being lower than the averaged
    counterpart, and the mid-values of the curve rising higher.
  \end{enumerate}
\end{enumerate}

From the graphs of these two keys, we can see that in both instances,
the actual fugues written with the specific key in mind both accentuated
the effects of their respective key's temperament profile compared
against the expected values from the graphs of averages. This strongly
suggests that the compositional elements of these two fugues are acting
somewhat in sympathy with the underlying trends dictated by temperament;
again, if the null hypothesis were to be true, we should have no reason
to see any systematic difference between the interval profile of a fugue
written with a certain key in mind, and an averaged graph of a
collection of pieces all transposed to that particular key.

Having looked at these two examples, and developed a sense for the
function of the KDE plot, the following plots present the KDE plots and
histograms of all of the minor keys on the same axis.




\begin{figure}[H]
\vspace{1.5em}
    \centering
    \adjustimage{max size={.8\linewidth}{0.9\paperheight}}{analysis_1_files/analysis_1_45_0.png}
    \caption{All Minor Fugues Vertical KDE Plots.}
\end{figure}


\begin{figure}[H]
\vspace{1.5em}
    \centering
    \adjustimage{max size={.8\linewidth}{0.9\paperheight}}{analysis_1_files/analysis_1_48_0.png}
    \caption{All Minor Fugues Vertical Weighted Histograms.}
\end{figure}    With all the keys now present, we can see a smooth---and rather
elegant---gradient from Pythagorean to Meantone, cycling through the
circle of fifths. For this graph, and for many of the graphs in the rest
of this thesis, I have assigned the Pythagorean keys to shades of blue,
and meantone to shades of red, with the darkest shade falling on the
most extreme of the keys for both groups (dark blue for b-flat, and dark
red for a minor), and gradually getting lighter as keys progress away
from these centers along the circle of fifths. With this system, the
most extreme ends of the Pythagorean keys will be represented with the
darkest shades of blue, and the extreme ends of the meantone spectrum
with the darkest reds, with the transition/boundary keys light shades of
either blue or red, depending on what end of the spectrum they fall.
This color scheme not only reflects the circular nature of the fifths
progression, and wrap-around at the edges, but more crucially it is
important to recognize that, besides for sheer color choice, the
systematic pattern of shading for the keys is not arbitrary, nor was it
motivated or adjusted in response to the trend set forth by the graph.
Rather, it is merely a reflection of the circle of fifths (as observable
in the legend in the upper right hand corner), and any emerging pattern
that results in the graph in terms of smooth gradient transitions is a
direct result from the key structures of well-temperament.

Indeed, the structure of the keys here is apparent: the sharpest peaks
at 0 and -22/22 are kept by the darkest shade of blue, falling off
gradually and trasitioning into the red tones, which cluster towards the
mid-ranges of the graph. As expected, at the extreme ends of the
spectrum on the x-axis (-22 and 22 cents), the lowest corresponding
y-axis values are kept by the darkest shades of red. It is also expected
that the lowest point of the peaks at 0 cents belongs not to the darkest
shade of red (a minor), because while a minor is the key that maximizes
the purity of the minor third, it is not the key that maximizes the
usage of the tempered fifth. Rather, the keys that maximizes the
tempered fifth is centered around g minor, and we can see this reflected
in the graph accordingly.

Now let us consider the same graph, but of the averaged fugues instead:




\begin{figure}[H]
\vspace{1.5em}
    \centering
    \adjustimage{max size={.8\linewidth}{0.9\paperheight}}{analysis_1_files/analysis_1_52_0.png}
    \caption{All Averaged (Expected Value) Minor Fugues Vertical KDE Plots. }
\end{figure}


\begin{figure}[H]
\vspace{1.5em}
    \centering
    \adjustimage{max size={.8\linewidth}{0.9\paperheight}}{analysis_1_files/analysis_1_55_0.png}
    \caption{All Averaged (Expected Value) Minor Fugues Vertical Weighted Histograms. }
\end{figure}    The graph yielded by a plot of the averaged fugues bears a close
resemblance to the graphs of the actual fugues, but in many ways is an
idealized graph of smooth and consistent key change. The spacing between
the fugues is even, the gradient order is perfect in regards to the
Pythagorean (blue) fugues on the 0 cents domain. Here are the graphs
side by side, with the y-axis set to the same upper limit value for
accurate comparison:


    \begin{center}
    \adjustimage{max size={0.9\linewidth}{0.9\paperheight}}{analysis_1_files/analysis_1_57_0.png}
    \end{center}
    


\begin{figure}[H]
\vspace{1.5em}
    \centering
    \adjustimage{max size={0.9\linewidth}{0.9\paperheight}}{analysis_1_files/analysis_1_59_0.png}
    \caption[Actual and Averaged Minor Fugues Vertical Comparison KDE Plots and Histograms. ]{All Actual (right) and Averaged (left) Minor Fugues Vertical Comparison KDE Plots and Histograms.}
\end{figure}    From laying out these two graphs side by side, with the same upper limit
value on the y-axis, we can start to parse out the differences between
the two graphs. Similar to the earlier, simpler graphs in the beginning
of this section comparing actual fugues with averaged fugues for the two
exemplar fugues, b-flat and a minor, the same trend can be observed in
these graphs of keys tending towards the more exaggerated versions of
their temperament profiles, especially on the dimension of just
intervals. If we look at the y-axis spread for the 0 cents from just
placement on the x-axis, we see the averaged fugues clustered more
compactly towards the mid-values, while the spread of the actual fugues
is greater, with more extreme values on either end of the blue/red
spectrum. Precisely, with exception of the two outer boundary keys, f
minor and c minor, all of the Pythagorean keys have higher peaks at 0
cents from just in their actual fugue plots as opposed to the averaged
fugues, with exception of d-sharp minor, and all of the actual fugues in
the meantone keys respectively have lower peaks. As we will see in this
next subsection repeating these graphs for the horizontal intervals,
effects on the horizontal dimension are even more dramatic, with the
polarization of the two groups of keys even more pronounced for the
graph of the actual fugues.

    \subsubsection{Horizontal Dimension
Plots}\label{horizontal-dimension-plots}

Before I get the following section looking at transposition plots in
greater detail (prior to then, all the transposition plots we have been
looking at are averages), I will repeat all of the previous vertical
subsection's interval plots, but this time, for intervals on the
horizontal domain. As this is the first time in this analysis section
that horizontal plots are being introduced, a few important things of
note on the differences between horizontal and vertical structure are as
follows:

\begin{enumerate}
\def\labelenumi{\arabic{enumi}.}
\tightlist
\item
  As stated in the previous section involving interval maps, because of
  the linear nature of horizontal structure, with the predominant
  melodic motion being stepwise, the intervals most frequently used on
  the horizontal domain are the narrower, step-wise intervals of
  seconds, both major and minor, followed by narrow leaps of minor
  thirds. The combination of this linear structure, and Werckmeister
  III's temperament structure of all intervals under a major third (m2,
  M2, and m3) being either just, or narrower than just, results in
  horizontal graphs that have a majority of their material on the
  negative end of the spectrum, with the information on the positive end
  largely representing major thirds, as leaps wider than a major third
  become increasingly uncommon. In contrast, the vertical domain has a
  more even distributions of intervals, and because of the triadic
  nature of Western tonal music, the majority of the intervals
  represented on the vertical domain are not seconds, but rather fifths
  and their inversions (fourths), and thirds (leaning towards minor
  thirds in minor mode pieces) and their inversions (sixths). In
  Werckmeister III, these intervals and their inversions span both the
  positive and negative end of the spectrum more or less in an equal,
  balanced fashion, resulting in the more symmetric graphs observed in
  the vertical domain.
\item
  Completing the logic outlined in item 1 above regarding the
  differences between treatment of intervals in the horizontal and
  vertical domains: because larger intervals are a rarer occurrence in
  the horizontal dimension, this gives us a unique platform on the
  horizontal domain for analyzing intervals that exceed the width of a
  third (increasingly so for intervals that exceed the octave), as we
  have more of a reason to believe that the presence of these intervals
  in the horizontal domain are more of a result of deliberate
  compositional choice beyond a mere functional role, especially if
  these intervals occur within subject or countersubject material.
  Indeed, this is a prominent motivating factor behind the focus upon
  fifths and fourths on the horizontal dimension in the later sections
  in this dissertation that deal increasingly with the relationship
  between temperament and the underlying compositional forces.
\end{enumerate}

This is not to say that the larger intervals on the vertical domain are
all present because of sheer harmonic function, and that no hierarchical
order of motivic importance can be established on that domain. Rather,
it is merely more difficult for us to parse out this structure; this may
be perhaps due to the current limitations of the framework, or that the
vertical structure for the Baroque fugue is intrinsically more nuanced
than the horizontal, being in many ways a resultant structure of a more
primary horizontal structure. Regardless of the exact reason, the more
general nature of the vertical domain may call for extra parameters to
constrain the data when trying to formulate conclusions about vertical
intervals and their interaction with the structure of other musical
elements. Likewise, when analyzing semitones on the horizontal domain
(which will start to become an increasingly important area of analysis
starting with the latter portion of the next chapter), further
constraint of the data through the addition of specific functional or
musical parameters, such as looking at semitone usage in prominent
locations within subject material, how semitones vary as a function of
note length, or semitones in specific scale degrees implicative of
leading tone function, may be needed to parse out any salient conclusion
about how semitones affect functional and thematic musical structures.
For this section on KDE plots and histograms though, such additional
musical parameters are not necessary in looking at these macro
structures.

One last note before the horizontal graphs are presented in this
section: I have set the bandwidth of the horizontal KDE plots to a
slightly smaller value (1.8 as opposed to 2) than the KDE plots in the
previous, vertical section, so any possible perceived increase in the
frequency or intensity of peak/trough oscillation is merely an effect of
the decreased bandwidth, and not anything intrinsically reflective of
the data itself.




\begin{figure}[H]
\vspace{1.5em}
    \centering
    \adjustimage{max size={.8\linewidth}{0.9\paperheight}}{analysis_1_files/analysis_1_64_0.png}
    \caption{All Minor Fugues Horizontal KDE Plots.}
\end{figure}


\begin{figure}[H]
\vspace{1.5em}
    \centering
    \adjustimage{max size={.8\linewidth}{0.9\paperheight}}{analysis_1_files/analysis_1_67_0.png}
    \caption{All Minor Fugues Horizontal Weighted Histograms.}
\end{figure}


\begin{figure}[H]
\vspace{1.5em}
    \centering
    \adjustimage{max size={.8\linewidth}{0.9\paperheight}}{analysis_1_files/analysis_1_70_0.png}
    \caption{All Averaged (Expected Value) Minor Fugues Horizontal KDE Plots. }
\end{figure}


\begin{figure}[H]
\vspace{1.5em}
    \centering
    \adjustimage{max size={.8\linewidth}{0.9\paperheight}}{analysis_1_files/analysis_1_73_0.png}
    \caption{All Averaged (Expected Value) Minor Fugues Horizontal Weighted Histograms. }
\end{figure}
    \begin{center}
    \adjustimage{max size={0.9\linewidth}{0.9\paperheight}}{analysis_1_files/analysis_1_74_0.png}
    \end{center}
    


\begin{figure}[H]
\vspace{1.5em}
    \centering
    \adjustimage{max size={0.9\linewidth}{0.9\paperheight}}{analysis_1_files/analysis_1_76_0.png}
    \caption[Actual and Averaged Minor Fugues Comparison Horizontal KDE Plots and Histograms. ]{All Actual (right) and Averaged (left) Minor Fugues Comparison Horizontal KDE Plots and Histograms.}
\end{figure}    The same trends observed in the vertical domain in the previous
subsection are also consistent in the horizontal domain, but the effect
of the separation between the Pythagorean and Meantone keys is even more
pronounced here. The following is a list of additional observations for
the horizontal searches:

\begin{enumerate}
\def\labelenumi{\arabic{enumi}.}
\tightlist
\item
  Consistent to the plots generated in the vertical domain, the fugues
  in the horizontal domain observe the same rule of keys tending towards
  their type, notably in regards to treatment of pure intervals; in
  other words, Pythagorean keys group together and tend to maximize
  interval purity, and meantone keys congregate at the other end of the
  spectrum.
\item
  This effect of keys tending towards their home types, especially in
  dealing with just intervals, can be seen in the sharp division that
  exists between groups of keys in the KDE plots and histograms of the
  actual fugues---even more so than what we have observed on the
  vertical domain---which is less present in the graphs of the
  averages that transition more seamlessly between Pythagorean to
  meantone. Additionally, like the vertical graphs, the spread of the
  y-values for the 0 cents from just x-values is far greater for the
  graphs of the actual fugues than their averaged counterparts. On the
  histograms, the values of the three darkest blue markers at 0 cents
  from just span from around .26 - just under .3 on the y-axis of the
  actual fugues' graph, while the highest marker on the averaged graph
  starts around .26, and spans all the way down to .23. Likewise, for
  the meantone values on the same 0 cents from just x-axis value, the
  lowest three red values span from around .08 - .12 on the graph of the
  actual fugues, while the first marker on the graph of averages begins
  after .12.
\item
  For the outer, positive values on the x-axis (16 and 22 cents from
  just), we would expect the blue values to be higher than the red
  values at these outer extreme regions, as the system of
  well-temperament assigns these more extreme tempered intervals to the
  Pythagorean keys. While this is the case for the horizontal graph of
  the averages (as well as consistent with all graphs on the vertical
  domain, as well as the key maps from the previous section), the graphs
  of the actual fugues on the horizontal domain muddies and somewhat
  reverses this effect, with some of the red (meantone) values higher
  than their blue counterparts at the extreme positive end of the
  x-axis. While this may be initially perplexing, we will observe in the
  next section that deals with untempered interval graphs that this is
  due to the comparatively high usage of melodic sixths almost
  exclusively in the meantone keys, so much so that it has created a
  reversal of the otherwise expected placement of the Pythagorean keys
  maximizing these positive, outer extreme tempered values. However,
  what is important to observe from this graph is that these changes and
  trends are being made in a systematic way that preserves key type
  (meantone/Pythagorean) unity, instead of being randomized.
\end{enumerate}

Returning momentarily to the subject of polarization of the two key
types on the domain of just intervals in the actual fugues, in the
previous vertical section, we observed that all of the keys, with the
exception of d-sharp minor, and the boundary keys of f minor and c minor
had more extreme peaks (higher values for the Pythagorean keys, lower
value for meantone keys) in the case of their actual fugue graphs in
relation to the graph of averages. In the case of the horizontal domain,
all keys, d-sharp and the boundary keys included, behave according to
this effect of polarization for the just intervals. Below are isolated
KDE plots and histograms presented in table format, with each key's
actual fugue paired with that key's averaged fugues on the same axis for
comparison. Actual fugues are marked with solid lines, while the
averaged fugues are marked with dotted lines, with colors corresponding
to their previous assigned key colors (Pythagorean keys in shades of
blue, meantone in shades of red).

Note that, in the case of the blue plots, all solid lines lie above the
dotted lines at 0 cents from just, and for the red plots, all solid
lines lie below the dotted lines. The consistency of this result across
all keys in these horizontal comparison graphs is a robust indicator
that temperament is exerting an effect on compositional elements on the
horizontal domain, strongly affirming a positive relationship between
temperament and key on a compositional level, as well as further
bolstering the existence of a presence and division between the two main
key groups.




\begin{figure}[H]
\vspace{1.5em}
    \centering
    \adjustimage{max size={.7\linewidth}{0.9\paperheight}}{analysis_1_files/analysis_1_80_0.png}
    \caption[Actual and Averaged Minor Fugues Comparison Horizontal KDE Plots Isolated by Key. ]{Actual (solid line) and averaged (dotted line) minor fugues comparison horizontal KDE plots isolated by key. Note how for the x-axis position of 0 cents from just, all the solid lines lie above the dotted for the Pythagorean keys (marked in blue), which is reversed for the meantone keys (marked in red).}
\end{figure}


\begin{figure}[H]
\vspace{1.5em}
    \centering
    \adjustimage{max size={.7\linewidth}{0.9\paperheight}}{analysis_1_files/analysis_1_83_0.png}
    \caption[Actual and Averaged Minor Fugues Comparison Horizontal Weighted Histograms Isolated by Key. ]{Actual (solid line) and averaged (dotted line) minor fugues comparison horizontal weighted histograms isolated by key. Note how for the x-axis position of 0 cents from just, all the solid lines lie above the dotted for the Pythagorean keys (marked in blue), which is reversed for the meantone keys (marked in red).}
\end{figure}    \subsection{Transposition KDE Plots and
Histograms}\label{transposition-kde-plots-and-histograms}

\subsubsection{Setup}\label{setup}

Through comparing the graphs of actual fugues with their averaged
counterparts in the above section, we've observed the trend in both the
vertical and horizontal domain that fugues belonging to the same key
type tend to maximize certain defining traits (particularly interval
purity) dictated by the temperament envelopes of that particular key
type, group together, and behave similarly to other keys belonging to
the same group. In short, our preliminary findings using the
transposition/averages function in the previous section suggest that
Pythagorean compositions tend towards Pythagorean, and Meantone
compositions towards Meantone, establishing that there are compositional
elements in play that must be more than just the mere effects from pure
temperament alone.

In this section, I will break apart the structure of the transposed
fugues to demonstrate that this aforementioned structure of key type
unity exists and can be systematically observed in a patterned manner on
the level of individual fugue transpositions. To break this apart for
better understanding, this section will present tables of KDE plots and
histograms for each individual key, for both the vertical and horizontal
domains respectively, similar to the charts seen in the last portion in
the above section. However, instead of presenting a single plot
representing the average of all transposed fugues for each key, I will
omit the final averaging function from these graphs, and instead display
all the transposed fugues as separate curves for a given key together on
the same axis. Effectively, this will result in 12 plots per axis, and
12 graphs per table (one graph per key): for the starting graph of f
minor, all fugues will be transposed to f minor, and plotted together on
the same axis for comparison, then for the next graph, all fugues will
be transposed to b-flat minor, plotted, and continuing in this fashion
until the last graph of the table will plot all fugues, transposed to c
minor.

To make sure this is clear before the tables of all graphs are
presented, let us look quickly at the anatomy of these transposition
plots using the example of the vertical and horizontal KDE plots and
histograms of the key of b-flat minor:


    \begin{center}
    \adjustimage{max size={0.9\linewidth}{0.9\paperheight}}{analysis_1_files/analysis_1_85_0.png}
    \end{center}
    


\begin{figure}[H]
\vspace{1.5em}
    \centering
    \adjustimage{max size={0.9\linewidth}{0.9\paperheight}}{analysis_1_files/analysis_1_87_0.png}
    \caption{All Minor Fugues Transposed to B-flat Minor Vertical KDE Plots and Histograms. }
\end{figure}    In these plots charting the key of b-flat minor, every curve in the
graphs represents a fugue that has been transposed to b-flat minor,
color coded appropriately corresponding to their original key before
transposition. So for example, the dark blue curve for both graphs,
labeled "b-" is the b-flat minor fugue, transposed to b-flat
(essentially our original b-flat minor fugue); the dark red curve (a
minor) is the original a minor fugue, transposed to b-flat minor. As a
bit of a recapitulatory note here, what we were seeing in the earlier,
"averaged" graphs were just the averages of all these different curves
(in this case, the b-flat averaged graph was just an average of all
these fugues, transposed to b-flat minor), but here they have not been
combined, but rather are being presented individually for greater
in-depth analysis.

For the KDE plots, colors and linewidths are left as is, for the
histograms, the original fugue's (in this case, b-flat to b-flat minor)
plot has been made bold and brought to the forefront of the graph.

\subsubsection{Projected Expectations: Null Hypothesis vs. Alternative
Hypothesis}\label{projected-expectations-null-hypothesis-vs.-alternative-hypothesis}

While this might seem like a tedious exercise on the onset, the results
from these graphs are important and can be very telling for the
following reason: If we assume a null hypothesis that temperament has no
remarkable effect on the underlying compositional devices, and that
there is no intrinsic, compositional relationship between temperament
and key, then it should logically follow that transposing a composition
written in a particular key to a different key should render it
untraceable to the key that it was originally pitched in, as there are
no remaining compositional forces, besides temperament, that was
rendering that composition unique to its specific home key. Continuing
this line of logic, if we take a collection of pieces, each pitched in a
different original key, and transpose them all to a same target new key,
we should not be able to detect any patterns or trends between the
resultant pieces based on their original keys or key groups before
transposition, besides for the unifying trait that they would all
resemble the new key in terms of their temperament profiles; in essence,
transposition should effectively "erase" any traits a composition has
that are linked to temperament. In our case, our collection of 12
fugues, one fugue for each minor key, transposed to a new target key
should yield a resulting chart of key plots that are essentially
randomized, with no pattern or trends between plots according to their
original keys.

It is important to specify here that we are not stating that these
fugues do not have each their own unique set of compositional devices
and musical structures, and that, if we do not detect any patterns
through transposition, then the data would then suggest these
compositions to be uniform copies of one another---to make such a
statement would be absurd and simply untrue. Rather, while it may remain
perfectly true that these compositions could differ drastically in terms
of their compositional elements and forces, unless these elements are
correlated with key---specifically, key traits determined through
temperamental influences---we would still have no logical motivation
to reasonably expect detectable trends and patterns in resultant
temperament graphs in a new key after original keys are passed through a
transposition function.

Given the expectations of what should result if we expect the null
hypothesis to be true, what should we then anticipate to see in the case
of a valid alternative hypothesis? If the alternative hypothesis were to
hold true, and there did exist a connection between temperament and
Bach's approach to composition given key, we should then expect to
detect:

\begin{enumerate}
\def\labelenumi{\arabic{enumi}.}
\tightlist
\item
  Patterned trends across the keys, similar to patterns seen in the
  previous subsection of keys grouping together and moving as a unit, or
  plots wanting to retain some form of semblance to the temperament
  envelopes of their original source key.
\item
  Furthermore, we might even expect to see that the shape of the plot
  belonging to the "correct fugue", that is, the specific fugue that is
  transposed into its own key (b-flat to b-flat, c\# to c\#, a to a,
  etc.) is best retained as opposed to the other fugues transposed to
  its key, or at least, best retained by a closely related key. This
  effect would be easiest to parse out at the extreme ends of the
  spectrum (namely b-flat minor and a minor), as we would only have to
  try to observe whether or not the extreme placements on the curves are
  kept by these keys.
\end{enumerate}

With these logical guidelines then in mind, returning to to examine the
b-flat minor group of transposed fugues, we can observe that:

\begin{enumerate}
\def\labelenumi{\arabic{enumi}.}
\tightlist
\item
  In both instances of the vertical and horizontal domain, the original
  0 cents from just peaks are kept by members of the Pythagorean key
  group (blue fugues), and in the instance of the vertical domain, the
  highest peak is still retained by the original b-flat minor
  fugue--its original, extreme position amongst the entire spectrum of
  keys (refer to histogram for easier parsing of the data). For the
  horizontal domain, b-flat minor is the third peak down at 0 cents from
  just, with the most extreme peak occupied by b-flat's subdominant key,
  d\# minor, one degree away via the circle of fifths. Additionally, the
  original b-flat minor fugue occupies the lowest region at the troughs
  along the x-axis of both the vertical and horizontal domain, again
  consistent with its original extreme tempered key profile.
\item
  The separation and "clustering" trend of Pythagorean and meantone key
  groups, as well as key gradation, can be observed in some effect in
  both graphs in their peaks and troughs, and is unequivocally
  pronounced in the horizontal graph, as there is an unmistakable
  division between the two groups of keys at the 0 cents from just point
  on the x-axis, as well as the two outer values.
\end{enumerate}

Before we proceed to the tables containing the transposition graphs of
all keys, let us quickly consider the same plots in transpositions to
the meantone exemplar key, a minor.


    \begin{center}
    \adjustimage{max size={0.9\linewidth}{0.9\paperheight}}{analysis_1_files/analysis_1_89_0.png}
    \end{center}
    


\begin{figure}[H]
\vspace{1.5em}
    \centering
    \adjustimage{max size={0.9\linewidth}{0.9\paperheight}}{analysis_1_files/analysis_1_91_0.png}
    \caption{All Minor Fugues Transposed to A Minor Vertical KDE Plots and Histograms. }
\end{figure}    For these plots, we can observe the similar effects as seen in the
b-flat minor plots:

\begin{enumerate}
\def\labelenumi{\arabic{enumi}.}
\tightlist
\item
  Status of the original key: in the histograms, the values for the
  actual a minor fugue preserve its original shape of lower values in
  the 0 cents from just domain (recall that we should not expect to see
  an absolute minimum at the 0 cents from just marker, as the original
  key of a minor does not minimize this value to begin with), and
  minimal values in the extreme, Pythagorean (± 22 cents) values,
  especially on the negative end of the spectrum where the minor thirds
  lie.
\item
  Grouping/clustering as an overall trend as a function of key and key
  type: The separation of Pythagorean and Meantone groups is again most
  clearly delineated in the horizontal plot, with a completely clean
  separation at 0 cents from just between high blue peaks, and lower red
  ones. This structure in the vertical domain is less pronounced than in
  the vertical plot of the b-flat minor; this is a trend that we will
  continue to see for the vertical plots for the meantone keys: while
  all of the Pythagorean keys on the vertical domain preserve their peak
  placements for the just intervals, this structure breaks down with the
  meantone keys. The horizontal domain, however, consistently preserves
  a clear separation of the keys throughout Pythagorean and meantone
  keys---in many cases, a division that is completely clean, as seen
  with the a minor graphs above. This may be a result of the vertical
  domain's weaker structure---either as an inherent trait, or as a
  result of the limitations of the computational framework---that
  renders only the Pythagorean keys strong enough to hold onto the
  structure of their own keys.
\end{enumerate}

The following tables now present all minor keys and their respective 12
transposition plots, starting from the KDE plots and histograms in the
vertical domain, followed by the same graphs in horizontal domain.




\begin{figure}[H]
\vspace{1.5em}
    \centering
    \adjustimage{max size={.7\linewidth}{0.9\paperheight}}{analysis_1_files/analysis_1_95_0.png}
    \caption{All Minor Fugues Transposed to Every Key Vertical KDE Plots. }
\end{figure}


\begin{figure}[H]
\vspace{1.5em}
    \centering
    \adjustimage{max size={.7\linewidth}{0.9\paperheight}}{analysis_1_files/analysis_1_98_0.png}
    \caption{All Minor Fugues Transposed to Every Key Vertical Weighted Histograms. }
\end{figure}


\begin{figure}[H]
\vspace{1.5em}
    \centering
    \adjustimage{max size={.7\linewidth}{0.9\paperheight}}{analysis_1_files/analysis_1_101_0.png}
    \caption{All Minor Fugues Transposed to Every Key Horizontal KDE Plots. }
\end{figure}


\begin{figure}[H]
\vspace{1.5em}
    \centering
    \adjustimage{max size={.7\linewidth}{0.9\paperheight}}{analysis_1_files/analysis_1_104_0.png}
    \caption{All Minor Fugues Transposed to Every Key Horizontal Weighted Histograms. }
\end{figure}    \subsection{Observations, Section Summary, and
Conclusions}\label{observations-section-summary-and-conclusions}

After observing these transposition graphs for all minor keys, the
application of the twofold rubric of key stasis and/or the formation of
grouping/clustering of key types finds it evident that there are clear
and discernible patterns that exist across keys and key types;
furthermore, the consistency and pervasiveness of these patterns across
both the horizontal and vertical domain, as well as a vast majority of
key areas (all key areas in the horizontal domain, and at least the
Pythagorean keys on the vertical), give a compelling weight of evidence
towards an alternative hypothesis of temperament having a direct role in
determining elements of compositional devices and structures according
to key.

The set of observations, interpretations, and conclusions from these
above tables are as follows:

\begin{enumerate}
\def\labelenumi{\arabic{enumi}.}
\tightlist
\item
  The plots on the horizontal domain are unmistakably patterned and
  correlated with respect to original keys: the existence of clear key
  gradients, and division between Pythagorean and meantone keys on this
  domain is unequivocally clear and striking.

  \begin{enumerate}
  \def\labelenumii{\arabic{enumii}.}
  \tightlist
  \item
    All of the peaks at 0 cents from just are maximized by Pythagorean
    keys, with a complete reversal on the outer, extreme values of -22
    and 22 cents from just. As we will see in later sections, this is
    very possibly due to the high volume of melodic sixths almost
    exclusively in the meantone keys.
  \item
    The division between the two key groups is indisputably clear, with
    beautiful, clean separations between Pythagorean and meantone
    especially on the 0 cents from just domain. On this domain, half of
    the keys graphs (g\#, c\#, f\#, e, a, and g) witness completely
    clean divisions between the groups (no interleaving of red and blue
    lines). The other half of the key graphs (f, b-flat, d\#, b, d, and
    c) have the lower 5 values all occupied by meantone keys, with three
    keys graphs having just one instance of a meantone key rising above
    one Pythagorean key, and three keys graphs have one meantone key
    rising above two Pythagorean keys. Of the three key graphs that have
    the reversal of one meantone/two Pythagorean keys, two of these keys
    are boundary keys (b minor and f minor). All in all, these results
    testify to an objectively clear division of the key groups,
    indicating temperamental influence on key on the compositional
    level.
  \end{enumerate}
\item
  The plots on the vertical domain also display patterns of "clustering"
  structure of key types, namely for the just values in the Pythagorean
  keys, in which the very upper tier of peaks are all occupied by
  Pythagorean keys. For two of these Pythagorean keys---b-flat and
  c-sharp---the peak utmost peak of the key graphs are occupied by
  the plot of the original fugue.
\end{enumerate}

\subsubsection{Summary of Chapter Conclusions and Setup for Intermediate
Chapter}\label{summary-of-chapter-conclusions-and-setup-for-intermediate-chapter}

The various analytical tests employed in this chapter point towards
clear conclusions of a positive relationship between temperament, key,
and general composition backed up by systematic and consistent results,
which are summarized below:

\begin{enumerate}
\def\labelenumi{\arabic{enumi}.}
\tightlist
\item
  Well-temperament creates systematic key effects which are independent
  of composition. The tuning system divides the keys into two large key
  groups: chromatic, or "Pythagorean" keys, and diatonic, or "meantone"
  keys, in which keys from their respective groups resemble the traits
  of their historic tuning structures (Pythagorean keys = pure fifths,
  tempered thirds, meantone keys = tempered fifths, close-to-pure
  thirds). Changes between keys are tonally gradated according to the
  circle of fifths, with a smooth transition between Pythagorean and
  meantone groups. The most extreme (exemplar) versions of the
  Pythagorean group fall on b-flat minor, and on a minor (if we are
  operating on the rubric of maximizing purity of thirds) for the
  meantone group. The division between the key groups is more pronounced
  in the graphs of the actual fugues (as opposed to the graphs of
  averages), indicating that this temperamental structure of key group
  division is further reflected in compositional structure.
\item
  Compositional structure given key is also reflective of temperament.
  This can be observed through our two validity checks of transposition
  and expected value in both the temperament graphs and the KDE plots.
  In other words, relations between temperament and key are more than
  just a pure effect of the temperament itself, but furthermore
  reflected in the compositional makeup of a piece written for a
  particular key.
\item
  These effects of temperament on composition given key are strongest on
  the horizontal domain. Perhaps the plots that offer the strongest
  evidence towards the positive relationship between temperament and
  underlying compositional structures in this section are the horizontal
  KDE and histogram transposition plots, in which we can witness clear
  and consistent divisions between Pythagorean and meantone keys that
  are retained after transposition.
\end{enumerate}

From these results, we have sufficient information to confirm that
temperament exerts an effect on composition given key in a consistent,
systematic, and quantifiable way on the most general level of
composition, completely guarded against any possibility of personal
bias. While our analysis here allows us to draw this conclusion about
statistical significance of the data on a macro scale, this most general
level of analysis does not definitively give us any express information
about the specific underlying intervals or musical elements responsible
for these temperamental effects, nor anything about their function,
placement or role within the score. As so, any claims about musical
significance tied to temperament are still largely conjectures at this
point. However, the information that we have found out about temperament
and key in this chapter will provide a guide to the parameters that we
will set in the next chapter, which will begin to look more closely into
the structures and musical elements that temperament interacts with, and
what the roles of these structures are in regards to musical function
and significance.

From these three key conclusions of this general chapter, the what we
would like to focus on in the chapter to follow is:

\begin{enumerate}
\def\labelenumi{\arabic{enumi}.}
\tightlist
\item
  Test our hypotheses formed in this chapter about the expressive
  differences between Pythagorean and meantone keys through looking at
  the relationship between key and non-intervallic musical elements of
  duration and texture to see if there is any marked difference in
  treatment between the two key groups.
\item
  Looking at the historically defining intervals of the Pythagorean and
  meantone tuning systems (perfect intervals vs. imperfect intervals),
  and analyze their usage in relation to key to see if we can observe
  connections between temperamental structures and the treatment of
  these intervals.
\item
  Because of the stronger correlations on the horizontal domain, we will
  pay an increase of attention to horizontal intervals in the coming
  chapters. The stronger effects of temperament on the horizontal domain
  in this chapter will also motivate us to examine the interval of the
  semitone, which shares a special place in minor mode both in term of
  function as well as affect.
\item
  We have already started to see patterns of keys getting divided in two
  main groups in this chapter; in the next chapter we will see if we can
  fine-tune this division through more thorough key anaylsis through
  looking at scale degree graphs.
\end{enumerate}


    % Add a bibliography block to the postdoc
    
    
    
