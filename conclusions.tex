    
    
    

    \hypertarget{Conclusions}{\chapter{Conclusions}\label{Conclusions}}
    At the height of the European Industrial Revolution, Parisian industrial
organization the Societé d'Encouragement offered a handsome award of six
thousand francs to the individual who could develop an economical and
physically comparable synthetic version of \emph{lapis lazuli}. The
award ultimately went to French chemist Guimet in 1824, and with that,
the first synthetic version of ultramarine was born, and it was this
synthetic ultramarine that would be subsequently used in seminal
painting of the latter portion of the 19th century, among them being
Claude Monet's \emph{Gare Saint-Lazare}, Camille Pissarro's \emph{The
Côte des Bœufs}, and Pierre-Auguste \emph{Renoir's Les Parapluies}.
While this initial version of synthetic ultramarine was almost
indistinguishable from its natural counterpart, at least in
decontextualized, isolated comparison, the color would continue to
undergo development throughout the next century and into the modern era,
with each version targeted with greater precision at recreating the
abstruse complexity of the original, the various chemical formulas and
procedures involved heavily guarded and patented. Indeed, the formula of
the first, early 19th century synthetic versions employed by Monet and
his contemporaries was so meticulously guarded that modern day research
teams, such as the partnership between UK chemist Ian Hamerton and art
scientist Nicholas Eastaugh, have gone to great lengths to studying the
spectroscopy of and recreating a historically accurate and true version
of the original formula, all for the goal of better understanding the
works of these 19th century artistic giants who employed \emph{lapis
lazuli}, a motivation not unlike that pursued by this dissertation in
terms of the study of temperament to arrive at a deeper understanding of
Bach's works.

While the history of the derivation of ultramarine has seen dynamism in
the treatment of the compound across time, the ethos of the quest for
ultramarine has remained unchanged---a strict and cardinal
dedication to the pursuit of a specific configuration of shade and hue,
a staunch faithfulness and unwavering adherence to detail and minute
complexities motivated by the belief that these details carry with them
and are directly responsible for the artistic integrity of the
compositions. Artists, art historians, and scientists alike have
unequivocally and unilaterally recognized the necessity for the careful
preservation of the original integrity of \emph{lapis lazuli}, not just
as a rote scientific pursuit, but more so as an artistic allegiance.
Ultramarine was not just a mere shade of blue that could be substituted
and replaced, but inherently special and indispensable in its
properties, within them encoded a sublime artistic force which would be
irredeemably modified with the alteration of the color. The simple
conviction, put in laymen's terms, was that these minute shades and
properties of colors mattered artistically, so strong was the allegiance
to this principle that artists pre-19th century sacrificed their entire
lives' savings to attain this precious compound. In the modern day, this
assiduous dedication has translated into entire professions being
dedicated to the synthetic replication of the shade, the careful
restoration of these original paintings, and the painstaking research of
the chemical composition of these original versions, all in the effort
to preserve the integrity of ultramarine. Even now, with the
meticulousness of modern advancement and the ready availability of these
synthetic versions of the color, there are still to this day purists and
traditionalists that eschew these substitutes, and instead go to great
laborative and economic lengths to procure ultramarine in its natural
form, grinding the stone themselves. In their eyes, there is still no
worthy substitute for the elusive properties of the stone in all the
depth, complexity, and subtlety contained in its original form.

Today, Vermeer's paintings are dispersed across the globe; the
\emph{Young Woman with a Water Pitcher} hangs in New York's Metropolitan
Museum of Art; \emph{The Milkmaid} and \emph{The Woman with the Blue
Reading Letter} can be found in the Rijksmuseum in Amsterdam, and if one
travels to the Mauritshuis in the Hague, one can peer across time into
the enigmatic eyes of the \emph{Girl with the Pearl Earrings}, as well
as behold the \emph{View of Delft}, the iconic scene that the narrator
of Marcel Proust's \emph{In Search of Lost Time} pronounced "as more
striking, more different from anything else he knew". Owing to the
careful preservation of some of these masterpieces, and faithful
restoration of others, even after the passage of hundreds of years,
Vermeer's use of colors still continues to inspire us to great lengths
of imagination, awe, and wonder. And as our understanding of ultramarine
has deepened over time, scientific knowledge and artistic intuition
unfold in a beautiful feedback loop. Understanding just where and how
Vermeer employed this ethereal color ultimately allows us a deeper
introspection into the artist's mind and revelation of his intimate
artistic process, providing us with a more complete comprehension of the
indispensability of this color to the magic created from his canvasses,
ultimately resulting in a more powerful and authentic experience of
these artworks.

However, time has not been kind to all of Vermeer's paintings; natural
ultramarine is unfortunately prone to "ultramarine disease", in which
the pigments fade and blanch over time, altering the artist's intended
coloration. For one who relied so heavily upon the compound as Vermeer,
and used ultramarine in such holistic, precise, and subtle contexts, the
results of this discoloration have been drastic, and many of his
paintings appear dramatically shifted from their original, intended
states. Indeed, the ramifications of ultramarine disease upon Vermeer's
paintings are not as simple as contained, isolated color shifts, but
instead hold large scale consequences, with effects that are much more
pervasive and insidious, shifting the entire perception of the artistic
integrity behind these paintings. French chemist Michel Eugène Chevreul
first codified this effect of the fickle relativity of color perception
with the term 'simultaneous contrast', stating that human perception of
individual colors are not fixed, but rather dependent on their composite
surroundings. Vermeer was no stranger to this understanding of the
relativity and multi-dimensionality of color, and his usage of
ultramarine are as much a celebration of the color itself as they are an
exercise of how such a powerful, complex color contrasts and alters our
perception of the hues around it, as well as its influence on other
visual dimensions. The dulling and waning of ultramarine draperies of
cloth is not just an isolated, negligible color shift, but a shift in
the entire perception of fabric depth and texture, as well as the
balance and color interplay against surrounding fabrics; the
discoloration of the underlayer of ultramarine on a wall for the subtle
effect of illumination results in the alteration of the lighting
settings of the entire painting, essentially darkening the perception of
the entire room, as well as the perception of light upon the other
colors contained within the painting. Ironically, the blanching of
Vermeer's ultramarine over time has made us appreciate the subtleties of
the artist's usage of the color even more, offering a compelling
testament to the principles of simultaneous contrast and the
interconnectivity between the artistic dimensions, and how the
alteration of one color can change the color scheme, balance, and feel
of an entire painting. In this light, efforts of restoration of
Vermeer's paintings are essential to the holistic experience of his
artworks, and thanks to the tireless efforts of artists and
conservators, we are now able to experience a large portion of the
paintings closer to the way Vermeer originally envisioned them. For the
paintings however in which restoration has not become a current
possibility, we only have our imaginations to rely upon, to where our
knowledge of ultramarine and its properties, and what they meant to
Vermeer, becomes even more crucial, as it remains our last bastion of
mental preservations of these timeless artworks and our faithfulness to
Vermeer's artistic vision.

The story of the historical journey of \emph{lapis lazuli} and its
artistic implications is in many ways a beautiful allegorical parallel
to the story of temperament embarked upon in this dissertation. In this
dissertation, we have demonstrated that temperament to Bach was very
much an auditory equivalent to the subtle sensitivity to color on the
visual domain, even as far as to draw the metaphor that these tempered
intervals were Bach's auditory palette, the bold, idiosyncratic
Pythagorean intervals akin to precious ultramarine. From the careful
analysis of these pieces, we have been able to unequivocally reject the
null hypothesis, as it is evident that Bach cared about temperament, and
structured his compositions and keys choices in ways so that certain
tempered intervals had deep and intense connections with the underlying
musicality and artistic expressivities he sought to convey. Through
observing the consistent, strong correlations between various
compositional elements, key, and interval, it is evident that Bach was
not treating temperament as an auxiliary, discardable historical
artifact, or an imperfection to be worked around, but rather regarded it
as essential component and artistic tool to the musical makeup of the
pieces themselves, affecting musical structures on various levels of
compositional scope.

The first chapter of this dissertation's analysis demonstrated how
temperament exerted effects on composition on the most basic and general
scale, dividing the keys into their historical meantone and Pythagorean
groups, not only reflected in the raw tuning scheme, but more
importantly reflected in and observable in the patterned grouping of the
compositions themselves. It was also in this chapter that we established
more prevalent patterns and noticeable effects in the horizontal domain,
guiding the analysis for the following chapters to focus more upon the
horizontal dimension, and the interval arguably most influential to
function and character affect on this dimension in the context of minor
mode: the semitone. The second chapter witnesses the incipient emergence
of key characteristics through fine-tuning the non-parameterized broad
trends of the first chapter, with the goal of pinpointing the
compositional components involved in this initial detection of key
division. From the intermediate tier, more detailed analysis in this
chapter, we revealed how temperament shares a systematic, correlative
relationship with key through the underlying functional and musical
elements of interval choice mirroring their original, historical
predilections, vocal texture, and tempo settings. Towards the end of the
chapter, we were able to analyze in depth the interaction between the
fifth, third, and horizontal semitone through seeing how temperament
affects scalar structures given key, creating increasingly fine-grained
differences within key group, and essentially setting a unique
coloristic palette for the individual keys. And finally, in the
concluding chapter of analysis, we arrived at the most detailed level of
thematic structure, and witnessed how these different scalar structures
given key were put to masterful use on the thematic level through
temperament's influence on integral compositional forces of contrapuntal
style, fugal techniques of stretto, musical and motivic symbolism,
approach to chromaticism, large scale formal architecture, and thematic
and subject choice.

Perhaps most succinctly and elegantly compelling though to the idea of
musical significance, like the phenomenon of simultaneous contrast for
visual colors, is how Bach has masterfully used these tempered intervals
as interplay against one another, like an exercise of light against
shade, bright red against cool blue, adjusting his approach to
chromaticism to mirror and accentuate the coloristic affects carried
within these tempered intervals, resulting in unique, delicate balances
of perceptual dissonance and consonance that have over time contributed
to the cementing of these key profiles and characteristics into our
collective musical consciousness. In fact, temperament's affect on
Bach's approach to chromaticism---mirrored in the microcosm of the
construction of his fugal subjects given key---is perhaps the most
significant and direct finding of the entire thesis. The b-flat minor
fugue, with a thematic juxtaposition of pure fifths against narrow,
Pythagorean minor ninths and Pythagorean minor tenths, accentuates to
the maximum the expressive qualities of the minor modality, creating a
key of color and pathos; the grandeur of architecture of the c-sharp
minor fugue that emerges from the construction of three contrasting
themes, each deeply symbolic of the story of the passion, their
characters highlighted and bolstered through temperament and set against
a backdrop of uniform, narrow chromaticism conjures a key of solemn,
meditative lamentation, and the dissonant, strange subjects of the b and
f minor fugues are brought to sense and perfected through the high
variation of tempered intervals contained by their home keys, cementing
the idiosyncratic nature of these boundary keys---all of these
fugues and their respective subjects reliant upon their unique
temperamental key palettes for these specific tempered intervals and
their complex color profiles. To divest from Bach's work the color
subtleties of these temperamental forces would be a certain removal of
several musical dimensions of his music, of one of them this crucial,
chromatic dimension that forms the backbone of these deeply expressive
minor fugues, impacting the genetic makeup of these compositions as a
whole. "It is but a minute acoustical shift," one skeptic might say. Or
perhaps, "There is barely anything detectable between the two different
shades if we place the intervals back to back in a decontextualized,
comparison study of sine waves." But to opine measuring musical
significance in such a limited manner betrays an uninformed naïveté
towards the subtlety and complexities of these artistic forces, and just
how interactive, intertwined, and deeply contextual their structures
are. At the end of the day though, strict auditory tests can indeed only
touch as far as the auditory realm; to reach further into the realm of
music, the true litmus test remains our ability to demonstrate that
temperament is embedded into the fabric of these pieces in an
artistically meaningful way, that Bach was employing temperament in ways
that deeply impacted the musical structure and thematic designs of these
underlying pieces, which is precisely what we have shown in this
dissertation.

Alongside these large scale effects of holistic structure though, there
yet remains the beautiful idiosyncrasies of these intervallic colors
themselves. As writer Ravi Mangla phrases eloquently in his article
musing upon the history of ultramarine, in this specific context,
referring to the movement towards the substitution of the natural
compound with synthetic versions: "'A color may be too pure. Modern
shades and colors often appear hideous, ironically, because of their
extreme purity,' writes Alexander Theroux in his triptych of essays The
Primary Colors. 'Old-fashioned blue, which had a dash of yellow in it
\ldots{} now seems often incongruous against newer, staring, overly
luminous eye-killing shades.' In our pursuit of perfection, of unspoiled
coloration, we purged colors of their unique characters. Even the finest
natural ultramarine, ground assiduously by hand, is riddled with odd
minerals: calcite, pyrite, augite, mica. These deposits cause the light
to be refracted and transmitted in subtly different ways. No two strokes
of paint are the same in their fundamental composition. Stand at the
right angle and you might catch a quiet glimmer of white or gold, like a
prick of light from some distant province of the cosmos."

The similarities between the way that these visual artists regard the
natural, complex state of ultramarine, and the nature of the
distribution of interval impurity within well-temperament are striking,
in that the focus is not so much upon extolling the virtues of purity
within intervals or colors, but rather in praise of the natural
impurities of the original material from whence these artistic forces
derive their character. In Bach's case, we have seen that he does not
shy away from these idiosyncratic---oftentimes
non-pure---intervals of the remote Pythagorean keys, but rather
celebrates them, showcasing them in motivically and thematically
important contexts---the opening statement of a fugue, the lone
idiosyncratic thematic interval, the boldness of a dissonant subject.
While there exists the narrative that well-temperament was designed to
preserve purity in meantone keys, yet it is in these Pythagorean keys in
which Bach's pathos and chromaticism shines the brightest---it is in
these remote keys, where the musical ultramarines exist, in which Bach
compositional approach is at its most creative, most expressive, most
memorable. What then would be the ramifications towards these pieces if
we were to enforce upon them a more "correct" equal temperament, as the
standard for tuning? From our understanding of Bach's usage of color and
how it directly relates to temperament, this logic would be as absurd as
to champion the removal of Vermeer's original ultramarine, and placing
synthetic color in its stead, with the reason that synthetic color is
more pure, or even more egregiously, to substitute ultramarine with a
slightly different shade of blue---cobalt, or azure. But even if we
were somehow even able to supernaturally suspend the properties of
mathematical and physical laws of acoustics to accommodate simultaneous
and ubiquitous pure intervals, would Bach have abandoned
well-temperament for this holy grail of acoustical purity? The analysis
from this dissertation asserts that he would not have wanted it so. His
Eden had already been discovered, in all its imperfect perfection.

Equipped with this knowledge of how extensive temperament was to the
construction of Bach's pieces, how can we not then dedicated time and
effort to temperament-focused analysis, and to the restoration of these
compositions in performance under their original temperamental settings?
The players and proponents of the visual arts have ubiquitously
recognized the value of this, and it is time that we in the auditory
arts follow suit. Art scientist Eastaugh, referring to the artistic
importance behind the painstaking spectroscopic research on the
composition of ultramarine, certainly concurs: "If you want to see how
Van Gogh or Monet could paint the way they did, then you need to do
these kinds of studies; if you want to protect our cultural heritage,
work out why it is falling apart and what to do about it, then you need
information from investigations like this." Likewise, we as artists on
the auditory domain have a comparable responsibility to strive to
understand Bach's compositions from the eyes of temperament if we are to
continue to champion and preserve the artistic integrity of these
precious works.

This now begs the question of realization: should we continue to perform
Bach's works under the setting of equal temperament, even if our
convictions and the data all indicate the musical importance of
well-temperament---would this not be artistic compromise? Should we
not be championing for a revolution towards well-tempered tuning for
pieces realized under well-temperament, at least in the case of the
pieces that we have studied? Certainly in an ideal world, performing
Bach's works in well-temperament would be the clear goal, and it is a
direction that we should be heading towards, just like restoration is a
goal for paintings, especially once we have knowledge of how these
colors play an eminent role in the artistic structures. However, in the
musical case, from a feasible standpoint, the current structures simply
do not support such a radical shift, given the deep dichotomization
between the modern tuner and performer, as well as our puerile
understanding of the sensitivities to the complexities of
well-temperament, having been long saturated in the tradition of equal
temperament. Under this light of current impracticality, does this then
render temperamental analysis, and our goals for the restoration of
well-temperament zealous and radical? If not for a push towards the goal
of temperament restoration, what value does our temperament based
analysis hold for the theorist, and perhaps even more saliently, for the
performer? As a response, an apt analogy can be drawn to the performance
of a piano transcriptions of orchestral works, and the responsibility
and importance for the performer's understanding of the orchestral
origins of the work for a meaningful and artistically accurate
performance. The knowledge of the original sounds and sonic intentions
are not only valuable but indispensable to the performer who wishes to
recreate with faithfulness the richness and variety of the original
sonic tapestry of flutes, trumpets, drums, and violins, even though the
physical forces through transcription are but felt hammers against metal
strings. But herein lies the power and importance of musical
imagination: it is the precise principle that the artworks exist first
in the minds of the performer; only then can they be adequately
translated to the listener. Indeed, if we as creators are not able to
have the rigorous construct of the original form of the artistic vision
first in our minds, what can we hope for in the validity of our
translations, and the resulting understanding of our audience? In the
same way, the intimate knowledge of temperament, its intervallic forces,
and their respective interactions with the underlying musical structures
of the compositional substrate must be first established in the
collective consciousness of the minds of musicians before we can hope to
translate these artworks, and their musical messages, to the general
population. We have witnessed the same principles in regards to the
Vermeer paintings in which original color has not yet been restored, in
that, in these cases, our mental understanding of the complex nature of
ultramarine and the artist's usage of the color remains crucial in
connecting us to Vermeer's original artistic intent. Likewise, our
understanding of well-temperament and its creative agents, in the
possible lack of current restoration, is the only force aiding us
towards the composer's original sonic intent, and it is this very mental
construct of these original sounds that will ultimately guide and inform
performances to more faithful renditions of Bach's authentic artistry.
Lastly, it is towards our continual evolution as artists and artistic
thought that we must strive to keep the artistic past alive through
faithful preservation. Newton once uttered the now famous words, "If I
have seen further than others, it is by standing upon the shoulders of
giants". Analysis helps us learn from the giants of the past, for
without faithful understanding and recreation of this artistic past, we
lose sight of where we have come from, what achievements have been made
and revelations discovered for us, so that we do not have to mire
ourselves in the process of rediscovery, but can instead take upon us
the torch of progress and move forward in continual innovation. More
than anything, this is why analysis is powerful, this is why it is
important, this is why it is irreplaceable.

In a larger philosophical framework, what I have sought to illuminate
through this dissertation are these neglected dimensions of Bach's
music---a heightened sense of color engendering an emotional affect
that is inextricably tied to the subtle variation of interval
sizes---that has been essentially forgotten in an era that has
eradicated this diversity through the hegemony of equal temperament. The
results of this dissertation have demonstrated that there is more
artistic import to tuning than meets the eye, and our modern system of
seeing the tuning and tempering of the keyboard as a scientific and
mechanical endeavor separate from artistic production is doing a great
disservice to our sensitivity to tonality, harmony, and musical
coloration. Historically, the integral role that tuning and temperament
played in regards to its control over harmony and coloration was
understood and appreciated by composers and theorists, and tuning was
viewed as more of an art rather than a science. Furthermore, there was
no dichotomy between the role of a tuner and the role of a
musician---composers and performers were expected to be able to tune
their own instruments, and although there were general guidelines to
follow for the procedure, the subtle nuances were not codified, thus
allowing for degrees of variation in tempering across different
individuals, very much akin to variations that arise across different
performers of the same piece due to artistic voice. Additionally,
systems that featured different sized semitones---leading to wider array
of intervals, which in turn govern key coloration---were favored
over the uniformity and blandness of equal temperament.

Ultimately, the modern era with its dissociation of the tuner from the
performer, as well as the widespread standardization of equal
temperament to the point of the rendering unequal systems obsolete, has
divested us of certain awareness and understanding of the full artistic
implications that the composers intended. It is my hope that this
dissertation will open up this new avenue of temperamental analysis,
leading to a more complete understanding of Bach's music, not just as a
form of reenactment and rediscovery, but also as a platform for a more
diversified way of looking at harmony and its creative agents,
transforming our experience of music that we come in contact with and
future compositions that we aspire to create.


    % Add a bibliography block to the postdoc
    
    
    
