    
    
    

    \hypertarget{History of Temperament}{\chapter{History of Temperament}\label{History of Temperament}}
    \section{Significance of Temperament and Need for
Study}\label{significance-of-temperament-and-need-for-study}

Before moving forward to a discussion on research methods and analytical
framework in the next chapter, I would like to take this chapter to
focus on an in-depth discussion on temperament, which is, as the reader
must have garnered, central to this dissertation. This section not only
seeks to help the reader develop an intuitive understanding of what
temperament is, and provide background information about the history of
temperament, but will more importantly address the issues as to why
temperament is so important to discourse on Bach's works, why we need to
be incorporating it as a mode of analysis if we wish to arrive at a more
complete understanding of the Well-Tempered Clavier, and why I have
chosen to make it a focal point of analysis for this dissertation.

The outline of this section is as follows:

I will first foster a general understanding of temperament and its
significance through answering the two questions:

\begin{enumerate}
\def\labelenumi{\arabic{enumi}.}
\tightlist
\item
  What is temperament?
\item
  Why is temperament important?
\end{enumerate}

Once these questions are answered and expounded upon, I will proceed to:

\begin{enumerate}
\def\labelenumi{\arabic{enumi}.}
\setcounter{enumi}{2}
\tightlist
\item
  Provide a brief history of the origin and development of temperament
  and its usage in the Western tonal tradition.
\item
  Discuss the role of temperament in the modern day musical spheres of
  performance and theory, as well as provide a survey of the types of
  methods and extent it has been employed as a vehicle of score analysis
  in the recent decades.
\end{enumerate}

Having established these items of foundational knowledge on the history
and current state of musical study of temperament, I can then proceed to
the final leg of this chapter, which is to:

\begin{enumerate}
\def\labelenumi{\arabic{enumi}.}
\setcounter{enumi}{4}
\tightlist
\item
  Discuss the need and significance of my particular aim for study on
  temperament in the context of the current state of literature in the
  field, and what this dissertation with its focus on temperament wishes
  to contribute to the understanding of Bach and current models of
  theoretical analysis as a whole (this is essentially the problem and
  need statement of the dissertation). This will lead smoothly into the
  detailed discussion on method that will comprise the third chapter of
  this dissertation.
\item
  To further elucidate what will be covered in this dissertation---and
  what is outside its range of discussion---I will also include a short
  section of delimitations at the conclusion of this chapter and clarify
  the scope that this dissertation covers.
\end{enumerate}

In the discussion of question b---the importance of temperament---in the
outline above, I will broaden the scope for a moment to include
discussion on musical traditions and genres outside of Western Classical
canon (which the rest of this dissertation is pretty much limited to) in
order to provide the reader with a larger perspective that the artistic
intents and ideas behind temperament---the deliberate usage of imperfect
intonation to achieve a specific artistic effect---are something that is
not just confined to temperament in Western Classical discretized
pitched instruments, but rather something that is far larger and more
universal, on par with musical elements such as harmony and rhythm. It
is important to note here that the broader discussion of other musical
traditions in this section is not meant to dilute the focus of the
dissertation, but rather provide a compelling illustration of the
universality of the musical ideas and effects imbedded in temperament in
order to underscore the significance of consideration of temperament as
a valuable tool in discussion of artistic intent in theoretical musical
analysis.

Another important point that the reader will see built in to this
section is the delineation between a focus on temperament as an artifact
of an imperfect solution to an auditory phenomenon (which, in some
interpretations, it very much is), and a discussion centered on the
utilization of temperament as a deliberate, artistic choice. This
distinction is a very important one to draw, and one in which will be a
guiding factor to the method of analysis used in this dissertation,
which will be detailed in chapter 3.

Perhaps it is due to this history of its development in Western musical
tradition as partially an endeavor to solve the physical problem of
intonation that the term ``temperament'' conjures up a very technical
schema. Indeed, even in modern musical spheres, the term has become so
specialized that it has been essentially relegated to jargon that is
reserved for the more so-called ``technical'' fields---music
technicians, early music historians, auditory engineers---while
simultaneously rendering it somewhat unpalatable to the more general
sphere of performers and music theorists, who have tacitly eschewed it
as irrelevant to the understanding of the aesthetics of modern
performance. This is the very viewpoint that I wish to question and
challenge directly in this dissertation, and to demonstrate how the very
opposite holds true. Not only should we not confine the discussion to
temperament to these very limited sub-fields, but such a quarantine of
the subject is debilitating to our general musical understanding,
especially of works from eras such as Bach's which were imagined under
temperaments which were unequal and very unlike our modern one. For,
while the notion of temperament and the impetus of tempering in tuning
probably did arise, in some ways, as a sort of sonic compromise in order
to combat the mathematical problems within intonation, artists,
musicians, and theorists during the day were able to recognize that it
ultimately transcended this functional role, and were able to embrace it
as a powerful artistic device and tool in composition. If this is the
case, we as modern artists, musicians, and theorists need to start
reconsidering---and rediscovering---temperament in such a light as well.

\subsection{What is Temperament?}\label{what-is-temperament}

The very first foundational brick to establish before any other
discussion in this dissertation can unfurl is the definition of
temperament and tempering, addressed directly with the question: what is
temperament?

Owen Jorgensen in his treatise on tuning defines temperament (and
tempering) as:

\begin{quote}
A musical scale in which the sizes of one or more of its natural
intervals has been altered (q.v., tempering) so that all or at least a
large portion of its intervals can be made to fit within an artificial
man-made pattern such as the conventional organ, harpsichord, or piano
keyboard arrangement.
\end{quote}

\begin{quote}
\textbf{Tempering} is the act of altering the size of a just interval.
The interval is thus made a small amount either larger or a smaller, and
this alteration puts the interval out-of-tune to a tolerable degree and
causes it to beat.
\end{quote}

To unpack all of that, \textbf{tempering}, in short, deals with taking a
just, or ``pure'' interval, and altering it ever so slightly so that the
resulting interval is either a bit narrower or wider than the original.
It is important to stress that tempering deals with small changes, and
does not change the class of interval that it alters; that is to say, a
tempered interval retains its nominal identity (e.g. a tempered minor
third is still a minor third, it doesn't become a major third or a major
second). Secondly, tempered intervals, while slightly altered from their
just counterparts, are still considered to be perceptually acceptable to
the listener. Essentially, through tempering, many different versions of
the same interval are created, each easily recognizable as belonging to
that class of interval, but in terms of actual frequency, individually
distinct. An easy illustration of this (and the one provided in the
preamble of this dissertation) is if one were to draw an analogy with
colors, we might consider a musical interval as a single color, and the
various versions of that interval derived through tempering as hues and
shades of that color.

The resulting musical scale of discretized pitches from this procedure
of tempering is a temperament. Like intervals created through tempering,
it is important to remember that different temperaments of a specific
musical scale does not create a new, different, scale. Rather, the key
here is that temperament creates a variety of different versions of the
same scale. Moreover, an unequal temperament, a discretization of a
musical scale that divides up the octave into unequal segments (rather
than an equal temperament, which divides up the scale into equal
segments), leads to an interesting result that each scale starting on a
different pitch within that temperament has its unique interval profile,
as opposed to scales within an equal temperament, which, aside from a
frequency modification, all have the same interval profiles preserved
across transposition. The fact that unequal temperaments create versions
of the same scale and intervals that are idiosyncratic to individual
keys is a key factor to this dissertation, which I will explain and
expound upon in full in later sections of this chapter.

On a fundamental level though, the main idea to take away from this
definition is that temperament deals directly with two key ingredients
to the experience of music: 1) music intervals, and their respective
sizes (i.e. making them narrower or wider), and 2) human perception
(think: shades of the same color) of these intervals.

\subsection{Why is Temperament
Important?}\label{why-is-temperament-important}

Having demonstrated the intimate relationship that temperament has on
interval and interval perception through unpacking its definition, we
can now adequately explore the importance of tuning and temperament and
the integral role they have to play in our aesthetic experience of
music. Because tuning and temperament govern interval production and
size, this role cannot just be practical; in other words, tuning cannot
just be relegated to a prerequisite condition to be satisfied before the
compositional and performance processes. Rather, it is of tantamount
importance to any other compositional device employed that should
contribute to a piece's aesthetic and emotional import. This is due to
the very fact that anything that governs interval creation in essence
comprises the foundation of harmony. Thus, tuning is not only a
scientific endeavor, but more importantly also an artistic one.

However, it is important to realize that the experience of aesthetic
import in music is not limited to, and in fact, far transcends mere
harmonic function. In fact, this is the very reason that discussions
about affect and key characteristics are so difficult to formalize, as
they deal predominantly with these elusive subtleties that can be, in
many ways, more closely linked with temperament. One of these main
subtleties, and perhaps even closer to the heart of what tuning and
temperament affects in terms of aesthetic perception, is the notion of
musical color. If we are to use the illustration of different musical
intervals (e.g. the minor third, perfect fifth, diminished seventh,
etc.) being somewhat synonymous to different colors on a palette, then
it is not difficult to see how temperament, which dictates and governs
the sizes of these intervals, has a very direct and immediate role to
play when it comes to musical coloration.

Consider that there are twelve different nominal simple intervals (not
exceeding the octave) that exist in traditional Western tonal practice.
Under the tuning system of equal temperament, there are indeed only
twelve different types of intervals (there is only one size major third
at 400 cents, one perfect fifth at 700 cents, etc.) However, under a
system of unequal, well-temperament such as Werckmeister III, there are
a total of 39 different types of intervals (four different sizes of
semitones, four different major thirds, two different sizes of fifths,
etc.)---more than threefold of an increase.

At this point, to briefly simplify and recap the line of thought
followed so far in this section: temperament is important because it
exerts direct control over intervals, which play a significant role to
the listener's aesthetic and musical experience. Moreover, the system of
tempering creates a finer gradation of interval sizes, thus broadening
the spectrum of musical ``colors'' that can be experienced.

Thus, if we believe that music is in part an emotional
experience---regardless of what that experience may be---and that one of
the meta-purposes behind theoretical score analysis is to understand how
certain musical elements that we believe salient relate intimately to
the sublime nature of that emotional musical experience, then it is not
only of scholarly interest that we pursue analysis on temperament, but
somewhat imperative that we do so, especially if we are to hope to gain
the most complete understanding of a piece which so directly employs it.
And just as theoretical score analysis on form or harmony, or any of
these other musical elements, does not attempt to directly engage
discourse on the philosophy and science behind music and emotion,
neither does my inquiry on temperament. Such a discussion on just what
the nature of what ``musical emotion'' is, and how it is physically is
a---as well as incredibly complex---and a valid question, but just not a
question that this dissertation seeks to answer.

Having established that the importance behind temperament lies in its
inextricable relationship to the musical experience, it is fitting that
we must address the salient issue of perception. In other words, just
because there exists a physical difference of the stimulus does not
necessarily mean that it translates to a human response, and in order to
determine true importance, we must have reasonable belief that the human
system can not only perceive of these fine grain differences that are
created through temperament, but also be able to respond to them
accordingly.

Skeptics may argue for a categorical ``no'' in response to this, or
perhaps the more conservative would hedge their argument, saying that,
even if the system is able to perceive it, it does not translate as any
sort of significant emotional import in the similar manner in which the
system differentiates and assigns unique responses to the standard
harmonic intervals (standard 12 interval division of the octave). While
the analysis in this dissertation is dedicated to providing a
quantifiable way to go about providing support for this argument that
the small grain differences produced by tempering was not only
discernible to the ear, but perceived as having producing artistic
differences in listener responses, and thus utilized as such, a very
intuitive way of illustrating this notion, and one of the most
convincing and immediate arguments for the fact that our systems not
only can reliably resolve differences in microtones, but assign
emotional and musical importance to them, can be demonstrated through
looking at the pitch language of musical traditions outside of the
modern Western tonal genre.

While it is true that a large portion of modern Western tonal music is
composed almost solely within the confines of the scale that limits the
division of the octave to 12 fixed intervals, with even our
non-discretized pitch instruments preoccupied with this notion of
``in-tuneness'' of adhering as closely to this strict pitch set as
possible, such is not the universal case across musical languages
geographical and ethnic origins. In fact, the opposite often holds to be
true---through looking at the musical traditions such as the
Afro-American blues, Japanese shigen singing, and Indian raga music, we
observe that musicians from these traditions go out of their way to
utilize and emphasize pitches and intervals that deviate significantly
from just, and in many cases, purposely take a pitch that could easily
be played as close to pure, and distort it in a way that is obviously
discordant with the intention of producing a specific musical effect.

With the blues, this sort of encoding of color and pathos within subtle
variation of pitch is expressed through the artistic bending of notes
that is central to the language of this tradition. Japanese shigen
singing utilizes similar bends, with musicians from this tradition
favoring this flexibility of pitch over the strict discretization of the
Western scale and tradition, in that it grants the musician more range
of expressivity. Traditional Indian raga music incorporates a musical
scale that is discretized into 22 pitches, or shrutis, (what we would
recognize in our Western musical vernacular as a microtone), each
produced at mathematically defined positions, and served as the
foundation of Indian music for centuries.

Expounding upon the blues as an example for a moment: the blues as a
genre is extremely earthy and expressive, and while a whole host of
musical elements have a role to play in contributing to this raw
emotion, a great amount of the music's emotional power has to do
directly with the bending of the ``blues notes'' as well as the major
and minor thirds. Just how a note is bent, how much it is bent, the rate
in which the bend happens, and how much the bend gets resolved by are
all components that can---and do---make a world of difference in terms
of expressivity, direction, tension, and release in a certain blues
composition. Surely, one could say that a minor third is a minor third,
and that the degree in which a blues guitarist has bent the interval
would not have much impacted a large-scale harmonic analysis of the
piece, but to view a blues performance in this light would to be all but
missing the point of the musical message conveyed, as a huge portion of
emotion in the blues is derived immediately and directly from these
pitch bends.

Returning to the subject of temperament in the Western tonal tradition,
it is commonly accepted among musicians that different types of
intervals have the power to elicit different affectual responses;
historically, this was extended to encompass different qualities and
sizes of the same interval. Prior to the late 19th century, tuning by
ear largely meant playing an interval melodically, and listening to the
quality of the interval instead of playing an interval harmonically and
listening to beats. During the development of equal-beating meantone
temperament in the mid-17th century, tuners began to listen to the
beating of fifths, but thirds, which they thought were too narrow an
interval to be able to reliably tune through beating, were still tuned
melodically, listening to the quality. The quality, or color, of
intervals was of immediate importance to tuners, musicians, and
theorists, who would discuss and compare different sizes of thirds and
their corresponding differences in affect (e.g. the calmness induced by
a pure, 5:4 ratio major third as opposed to the frenetic, excitable
effect created by a wide, 81:64 ratio Pythagorean third).

    \section{Scientific and Historical Overview of
Temperament}\label{scientific-and-historical-overview-of-temperament}

At this point, the question inevitably arises: how did all these
different sizes of one type of interval come into use? It does not make
much immediate sense that, given the pure, simple-ratio intervals that
arise naturally from the harmonic series, why music would not have
developed to just utilize those intervals that are most harmonious and
found readily in the natural world. The shorthand answer to this is
that, this indeed was the way in which people first went about
discretizing the pitch continuum, but quickly ran into a problem if they
wanted their intervals to possess the quality of invariance across
transposition.

\subsection{The Discrepancy of the Comma, the Need for Temperament, and
Pythagorean
Tuning}\label{the-discrepancy-of-the-comma-the-need-for-temperament-and-pythagorean-tuning}

If we look at the harmonic series, we can see a close relationship
between the first two generative intervals (this precludes the octave,
which generates no new pitch class of notes) that are found in the
series---the fifth and the major third---and the way harmony and chords
are constructed in Western tonal practice. The roots and evolution of
the Western tonality, as well as the diatonic scale itself, lie directly
in the physics of acoustics. The way that pitches of the diatonic scale
have been selected over the centuries is directly related to the physics
of sound and the overtone series, and our perceptual system's favoring
of simple ratios. The earliest tuning system---just intonation---took
its notes directly from the simple ratio intervals found in the harmonic
series. The reason behind this was that these simple ratio intervals
allowed for a complete integration of sound devoid of beating---a
perfect, harmonious blend. However, the problem of just intonation was
that, if one is to preserve the purity of all intervals, that is, to use
only just intervals in the construction of the diatonic scale of one
particular key, these same intervals in another key would be hopelessly
out of tune. In other words, perfect, simultaneously, harmoniousness
across all keys is a mathematical impossibility. The illustration
detailed in the following paragraph will attempt to shed more light on
this concept:

The fifth---the first interval found at the third partial in the
harmonic series---enjoys an especially privileged role in Western tonal
music. Essentially, Western tonality is a system that is built up on
fifths. Because of the importance of this interval, musicians and
theorists quickly realized the importance that these fifths be invariant
and uniform across different starting pitches. One of the earliest forms
of just tuning-\/-\/-Pythagorean tuning-\/-\/-the dominant system of
tuning during the medieval period of history, strove to preserve the
purity of these fifths as paramount, which was also reflected in the
contrapuntal style of the time which favored quartal harmony and the
heavy usage of perfect intervals on both melodic and harmonic musical
domains. In Pythagorean tuning, the chromatic scale was constructed by
stacking pure fifths, expanding on both ends until the circle was
completed. However, the problem was that when the circle was complete
(i.e. if one were to start on C, and after a series of fifths, arrive
back at C), the final pitch was not the same exact pitch as the initial
one, but rather sharper---a difference of about 23.5 cents, known as the
Pythagorean comma. In order to preserve the justness of the octave, the
last fifth in Pythagorean temperament (traditionally G\# -- E-flat) was
not truly a fifth but a wolf diminished sixth that was too narrow for
the ear to reasonably interpret as a fifth. This interval was completely
out of tune, rendering any key that utilized it subsequently unusable.

It was precisely this tradeoff between purity of intervals and key
accessibility, caused by the mathematical phenomenon of the comma, that
necessitated the art of tempering in search of a balance between the two
interests.

\subsection{Meantone Temperament, the Development of Well-Temperament,
and the Characteristics of
Keys}\label{meantone-temperament-the-development-of-well-temperament-and-the-characteristics-of-keys}

Sometime in the late 15th century there was a shift between favoring the
sound of the major third over the fifth, and with this came the
development of meantone temperament. The more precise definition of
meantone has to do with the placement of a whole tone at exactly the mid
point between its flanking neighbors, but more practically, meantone
came to be known as a temperament that sought to temper fifths in equal
amounts (not counting the wolf diminished sixth) in order to achieve
pure major thirds. This was also systematically reflected in the shift
of compositional style during the Renaissance, which started featuring
tertiary harmony over the quartal-based harmony that dominated the era
preceding it.

In order to achieve equal fifths, tuners had historically employed a
system called equal-beating, in which they would ensure that each fifth
beat at the same amount, and believed that this method would
mathematically yield fifths of the same size. This was, in fact,
mathematically incorrect---in order to obtain equal size fifths at
different starting pitches, beating rate must vary according to the
frequency of the starting pitch. However, this mathematical principle
was not known to scientists and tuners during the development and
widespread use of meantone, so, instead of achieving equal sized fifths,
fifths varied in size, and in a gradual and systematic way. As a result,
the major third also varied in size, with the major third of the tonic
triad for F Major, C Major, and G Major being closest to pure, and
gradually increasing in width as one moved away from these keys. This
system also supported the characteristic of keys, as the major third of
the tonic triad of different keys were of different sizes. However,
while meantone temperament was favorable in that it preserved the purity
of diatonic keys and supported key characteristics, modulation was still
limited. In meantone temperament, the intervals of C\#-F, F\#-B-flat,
and G-sharp-C, and B-E-flat, were wolf diminished fourths, about 13
cents wider than Pythagorean thirds, far too wide for the ear to be able
to accept as in tune. As a result, a third of the keys were still
unusable, and modulation to certain chords or key areas was restricted.

The system of well-temperament was first recorded, and expounded upon in
detail, by Andreas Werckmeister in his treatise Musikalische Temperatur,
in 1691. It was a system that was essentially the logical completion of
meantone temperament, and historically, it is more accurate to classify
well-temperament as modified meantone temperament, as it was a
compromised version of meantone that allowed for free modulation across
all keys. Well-temperament still preserved the purist thirds for the
diatonic keys, and had the size of these thirds gradually increase as
one modulates further away. However, what well-temperament did was that
it got rid of the wolf-diminished sixth, thus making the remote keys of
B, F\#, C\#, and A-flat usable, as the thirds that were previously wolf
diminished fourths in meantone temperament were now Pythagorean
thirds---wider and completely different in sound than just thirds, but
now acceptable to the ear.

\subsection{The Reign of Well-Temperament and Movement Toward Equal
Temperament}\label{the-reign-of-well-temperament-and-movement-toward-equal-temperament}

Throughout the duration of the subsequent musical periods after the
advent of well-temperament marked by tonality, it was well-temperament
that was utilized by musicians as the predominant method of tuning, as
it served the ideologies of tonality well through the graduated unequal
tempering of the thirds, which simultaneously provided harmonic purity
through the preservation of close-to-pure thirds in the most commonly
used keys, as well as strengthened the perception of tonal centers
through offering color variance between key areas. While movement
towards balancing out the difference between the sizes of the thirds
across the keys to achieve a more ``equal'' sound between the remote
ends of the key spectrum continued to progress, the musical community
still eschewed the usage of equal temperament because of its harsh
thirds, and the overall blandness of color that resulted from equalizing
all the thirds.

The conception of both volumes of the Well-Tempered Clavier (volume I
was composed in 1722, and volume II some 20 years later, between the
years of 1739-1742) took place during the height of the development and
evolution of the well-temperament tuning system. Certainly, the
Well-Tempered Clavier lends itself well to a survey of temperament and
how it interacts with key color, as the collections systematically cycle
throughout ``all tones and semitones'', including the remote keys of
F\#, C\#, and A-flat major that were not usable in the older, meantone
system, and thus a scarcity in musical composition during their age of
conception. Harmonically, the WTC was at the very vanguard in terms of
the types of dissonances used and the widespread chromaticism throughout
the work, making it a fitting study for color, use of chromaticism, and
affect.

True equal temperament was not implemented until the early 1900's---1917
to be exact---and was largely done as a response to music's growing
movement towards the dissolution of tonality, as well as to satisfy an
industry's demand to standardize the tuning procedure, as the
difficulties of working with the modern day steel framed instrument had
capitulated, for the first time, a sharp delineation between the
professions of performer and technician. With the standardization of
equal-temperament, what was gained was a sort of consistency and
invariance of sound, but the subtleties of well-temperament, and the art
of tuning that was so crucial to tonality, key, and the compositions
that were realized under that system, was systematically lost and, for
awhile, lay dormant and forgotten.

In terms of talking about Bach's compositions in conjunction to
temperament, the field has seen a shift of viewpoint and increased
interest in considering temperament in such discussions over the past
fifty years. Prior to 1950, the general consensus amongst theorists and
scholars was that unequal temperament was more of an artifact of the
past, and, for the most part, was not a necessary component to consider
when dealing with musical analysis or aesthetic experience. Specifically
in pertinence to the Well-Tempered Clavier, many even considered the
``well-temperament'' that Bach referred to in the title of the work as
synonymous to equal temperament, a claim that has, for the large part,
been debunked through more recent research over the past couple decades.
It was partially due to the advent of microtonalism in the 1960s, which
encouraged the exploration of alternative tuning systems for artistic
expression, as well as an increase of interest and awareness of
non-western musical languages and the pitch and scale systems that came
along with them, that sparked this movement to reexamine previous
beliefs about a subject of historical temperament that was, before then,
largely overlooked and dismissed. With this impetus, research and
analysis involving temperament became more mainstream in the spheres of
musicians and scholars; discussions about which exact tuning system
certain pieces were conceived under were raised, as well as performances
of compositions in their original tuning settings (e.g. Owen Jorgenson's
temperament recitals) championed.

As much as the last fifty years has seen a shift towards incorporating
temperament back into musical dialogues and analysis, and a substantial
amount of groundwork laid down, the main interest in such discussions
has been somewhat factionalized, with discourse about key characteristic
and overall aesthetic integrity delegated to analysis and research of
more qualitative bent, and with research involving quantitative methods
and systematic, statistical analyses of the corpus more concerned with
dealing with more technical issues, such as determining whether or not
there is internal evidence for temperament encoded within a given piece
to determine what deductions could be made as to pinpointing the
specific type of temperament used.

    \section{Current Literature on and Approaches to Temperament and
Analysis in Relation to the Well-Tempered
Clavier}\label{current-literature-on-and-approaches-to-temperament-and-analysis-in-relation-to-the-well-tempered-clavier}

The current state of the field involving discussions about Bach's
compositions (the Well-Tempered Clavier included), and temperament is
slightly dichotomized; research and analyses that employ a more
quantitative approach, such as statistical analysis or the application
of mathematical theories and models to try to measure certain ``levels''
of a musical element (e.g. amount of dissonance in a piece), largely
place temperament on the dependent variable side of the equation,
looking at what can be surmised about the specifics of temperament from
analyzing the music. Quite the opposite holds true in more aesthetic
studies and discourses, which are concerned more with the question of
how temperament affects musical elements, such as color and affect. It
is understandable how the subject of historical temperament is a tricky
one to explore in a systematic, methodical manner; to have any sort of
standardized, objective discussion about it would involve exact numbers,
but this is something that is a bit of an unknown, as much of these
intervals and tuning methods were done by ear and guided by artistic
intuition, and furthermore passed down orally from generation to
generation. For this reason, not much was discussed nor known about the
specifics of these historical tunings or how they sounded like in the
latter half the 20th century.

It was not until the 1960's, with the growing interest in construction
of historical keyboard instruments, that the issue of temperament became
of interest amongst scholars and musicians. In terms of Bach and his
compositions, while people harbored a general idea of the history behind
historical tunings, the specifics of which exact temperament Bach used
still remained unknown. A variety of methods were employed to try to
ascertain what specific tuning system Bach had in mind; these include
historical surveys, such as looking at the writings of Bach's
contemporaries, Kirnberger (Kelletat, 1960), and Werckmeister (Rasch,
1985), as well as looking at numerical and mathematical symbolism within
the composition (Kellner, 1977), and more recent work also involving
symbolism, but this time examining the design of the scrolls that Bach
sketched above the title of the work, and determining how much to temper
an interval given a certain number of spirals drawn on the interval's
alleged corresponding scroll (Lehman, 2005).

In terms of quantitative analysis done to ascertain temperament type,
Barnes (1979) pioneered a method of statistical analysis, trying to
determine Bach's temperament given the shape of the distribution and
frequency of types of major thirds across the Well-Tempered Clavier, a
method that will be, in part, expanded upon in the analysis of this
dissertation. Not much work has been done since then using statistical
analysis of intervals as a platform to analyze Bach's compositions in
regards to temperament, although this type of analysis is widely, and
successfully, used in research in areas of music cognition and
perception to address issues such as key recognition
(Krumhansl-Schmuckler Key-Finding Algorithm (Krumhansl, 1990), as well
as Temperley's Bayesian approach to the model (Temperley, 2005)).
Recently, methods of looking at the minimization of total dissonance
(Sethares, 1992) as a determining factor of Bach's temperament were
applied in a dissertation by Ruiz (2011), showing that total dissonance
in the Well-Tempered Clavier can be minimized using well-temperament
(Kirnberger II) over equal temperament, giving us reason to believe that
this, or something similar, might have been Bach's temperament.

While some of the aforementioned studies do touch upon discussions about
how temperament governs color variation, this is not, for the most part,
the main focus of the research. While I wish to employ methods of
statistical analysis very similar to Barnes's, the goal of my
dissertation is to consider compositional decisions and musical affect
and integrity as a function of well-temperament, and far less concerned
with determining which specific type of well-temperament. The goals and
purpose of this dissertation were very much influenced by the work of
Owen Jorgensen, who, in his tome, Tuning (1991) thoroughly documented
the entire history of tuning from Pythagorean to the standard system of
today's equal-temperament, provides an extensive, exhaustive list of
meantone, well, and quasi-equal temperaments, the procedure for laying
the bearings and tuning, and documents the different sizes of actual (as
opposed to theoretical) thirds in each as recorded by Alexander Ellis in
1885. Aside from being a comprehensive and thorough treatise on the
matter of tuning and temperament, the central goal of Jorgensen's book
is to provide a compelling argument that historical temperament is the
true basis of the `characters of the keys' through a combination of
historical evidence as well as direct discussion of how width of
interval influences key characteristics by employing direct examples of
compositional analysis from the vantage point of intervals.

Another text with a similar goal to champion the importance of
temperament, that has also encouraged me to embark upon this particular
focus for analysis, and has also influenced my approach to method, is
Johnny Reinhardt's treatise, Bach and Tuning (2004); the main thesis of
Reinhardt's text is that irregular, well-temperament is responsible for
a large portion of color in Bach's music, and is a dimension that would
be missing if we were to adopt an equal tuning system.

Like Jorgensen, Reinhardt utilizes historical survey as an approach to
analysis, but he also incorporates elementary elements of statistical
intervallic analysis, such as recording the opening melodic interval
(and its corresponding size in cents) of every movement in the six
Brandenburg Concertos, and explaining how certain correlations have
implications on the affect of the corresponding motif (e.g. Bach only
selecting the largest semitone for every instance of an opening semitone
gesture as a way to accent the grandeur of the opening statement). This
type of statistical analysis focused upon the purpose of looking at
underlying musical elements and implications is a method that I will be
utilizing in my dissertation, coupled with elements from Barnes's
method. Another conclusion from Reinhardt's treatise is that Bach's
compositions are best represented with Werckmeister III tuning, and that
we have reason to believe that this was the tuning system he used.

    \section{Statement of Problem}\label{statement-of-problem}

In light of the current state of the literature, which also applies to
the analysis of the Well-Tempered Clavier, the goal of this dissertation
is to apply these quantitative, statistical approaches to temperament as
a platform to analyze what would traditionally be more
intuitive/qualitative topics, such as key characteristics and the
emotional import of a theme or a motif, in a more systematic and
concrete fashion. While this dissertation deals with temperament and
statistical analysis of intervals, it is not primarily a mathematical
discourse, but rather uses these devices as a platform and angle for
discussing aesthetic issues such as affect, emotion, structural
architecture, motivic salience and symbolism, and key color and
characteristics in music.

Additionally, although this statistical approach to analysis deals
heavily with looking at distributions of certain classes of intervals,
this project is not focused on the minimization of dissonance or the
avoidance of dissonant or imperfect class intervals, to determine Bach's
temperament. Rather, the purpose is to strengthen the connections, and
provide a measurable means of evaluating the relationship between
temperament and certain musical elements that we derive emotion and
meaning from---harmonic implication, chromaticism, dissonance and
consonance, tempo, themes and motifs, large-scale architecture and
structural arch---and examine how these in turn contribute to the
overall effect of key characteristic and musical affect.

    \section{Delimitations}\label{delimitations}

\subsection{On Method}\label{on-method}

Anytime one deals with a subject matter that seeks the synthesis of a
variety of other, sometimes seemingly disparate elements, there will not
be one all-inclusive method to approach the matter, but rather, the
subject will involve various fields and angles of study. Such is the
case with the subject of this dissertation, which seeks to tie together
elements of tuning and temperament, the construct and structure of
musical language, and the aesthetics of affect and key characteristics,
which collectively incorporate aspects of the technical, mathematical,
aesthetic, and experiential altogether.

This dissertation chooses to look at the temperament and the
Well-Tempered Clavier using joint computational and theoretical means of
score analysis to examine intervals and their relationship to various
musical structures within individual pieces. Although one of the goals
of this dissertation is to draw conclusions about factors that
ultimately influence broader, more extra-compositional musical elements,
such as key characteristics and affect---rich subjects that heavily
involve the arenas of philosophy, aesthetics, and history---my project
is not one of rigorous philosophical discourse or historical
investigation. Of course, to offer the proper context for this
dissertation, I will inevitably be incorporating elements of philosophy,
aesthetics, and history, and might make gentle conjectures about what
sorts of conclusions could possibly be surmised in these certain fields
from results that I generate, but on the whole the methods employed in
these fields are out of the limit of those utilized in this
dissertation.

Because my study deals heavily with temperament in the context of a)
human sensitivity to being able to resolve the differences of fine
grained intervals and b) how this impacts certain higher-level cognitive
responses (e.g. emotional response, musical meaning, perception of time,
perception of larger-scale architecture and structural cohesiveness), it
automatically puts it in close contact with the fields of music
cognition and perception (and in terms of temperament, auditory
perception), that opens up further avenues of approaches that include,
but are not limited to, cognitive neuroscience, physics of acoustics,
and physiology of human hearing and the auditory system (which recruit
both top-down and bottom-up models). Again, while my statistical models
and modes of analysis will account for human perceptual and cognitive
mechanisms (e.g. the parameterization of placement of intervals within
the score-\/-\/-e.g. selecting for intervals within subjects, or opening
statement intervals-\/-\/-in the third chapter of analysis to parse for
hierarchical importance), my study is not primarily a study in these
arenas. Furthermore, while psychophysics would be a very direct, viable,
and fitting way to look at many of these questions that I am asking,
this dissertation does not employ psychophysics or immediate human
subjects as a data gathering means.

\subsection{On Scope}\label{on-scope}

In terms of the scope of this project, it must be asserted up front that
music, like language, is multi-layered and complex, and it goes without
saying that one factor alone cannot be solely responsible for an effect
as broad and subjective as affect and key character (even more concrete
elements such as musical structure and phrasing contain great margins of
subjectivity, as well as tap into vast number of contributing cognitive
functions). In many ways, this makes research that deals in the subjects
of the cognitive representations and meaning in music a thorny issue, as
there are, at any given moment, various networks of competing and
cooperating inputs.

Specifically in regards to my dissertation, the primary variable that I
am looking at is temperament, and how compositional and musical elements
subsequently are a function of temperament. This is in no way to say
that temperament is the only relevant variable responsible for these
aforementioned things, for surely, a plethora of variables, ranging from
more immediate ones such as harmony, rhythm, and phrase structure, to
more complex ones such as learned, cultural schemas (not inclusive),
contribute to the end result of overall musical meaning. I do not see
this factor in any way undermining the importance of my claims though,
as the main point of my dissertation is to assert that temperament is a
\emph{necessary}, but in no ways \emph{sufficient} input variable to the
final resulting of musical and aesthetic expression.


    % Add a bibliography block to the postdoc
    
    
    
