    
    
    

    \hypertarget{The Research Objective}{\chapter{The Research Objective}\label{The Research Objective}}
    The eyes of Sassoferrato's Virgin are downcast in prayerful devotion,
her countenance, half illuminated from the light above, and half
enshrouded in the shadow of her ivory veil, aglow with mysterious
divinity. Standing in front of her, one is overcome with the distinct
feeling of being prayed over, the soft earth tones and warm reds from
her visage and raiment emanating forth the Virgin's kind and tender
passion towards humanity. Contrasted against this muted softness is the
brilliant blue of the Virgin's cloak, a deep, mesmerizing shade
evocative of the heavens that define her divinity, just as much as the
understated hues of earth and red attest to her humanity. It is this
brilliant blue that has come to define and immortalize Sassoferrato's
painting of the Virgin in prayer, as it arrests the eye and commands it
to stop, transfixes the spirit to join in the Mother's wonder and
devotion, and remains in our consciousness long after we have departed
from the scene of her prayerful vigil.

Upon the altar wall of the Sistine Chapel is adorned Michelangelo's
iconic scene of the Last Judgement, the heavens seemingly parting before
the onlooker's eyes, unfolding a revelation of the second advent of the
Christ, and the eternity beyond. John the Revelator may have penned that
upon the second coming, all eyes shall be fixed upon the Savior, but
what rivets the eye to Michelangelo's earthly translation of the scene
is the sheer depth and dazzling iridescence of the blue canopy of
sky---a profound, resplendent, ecstatic blue that appears to stretch
into the realm of infinity, its enthralling translucence pulling the
gaze beyond the reality of our own space and into the divine space
beyond.

The unifying element in both the masterpieces portrayed above is the
prominent usage of the color blue, and it is no coincidence that the
blue from the Virgin's cloak and the blue of Michelangelo's sky are the
same blue, derived from the exact same source. This source is
\emph{lapis lazuli}, a semi-precious stone from the remote, northern
Afghanistan mountains, which after an extensive and complex process of
grinding produces the brilliantly blue pigment known as ultramarine
(Finlay, 2007, p.281). Described by Cennino Cennini as "A noble color,
beautiful, the most perfect of all colors" (Cennini, 2018, p.47),
ultramarine is truly the holy grail of all blues, an intense hue of pure
idealistic abstraction, the closest attempt we have at capturing the
flickering heart of a flame, or bottling the sky. For much of history,
ultramarine was worth more than gold, and due to this exorbitant cost,
the painters of the Renaissance and Baroque reserved it for the highest
of divine subjects---the robe of the Virgin, a case in which the
symbolism has endured over time (Plesters, 1966, p.64)---or in cases
in which no other shade could suffice, such as depicting the
expansiveness and elusive translucence of the heavens (Finlay, 2007,
p.288).

While most artists of the Renaissance and Baroque, either through
economic constraint or a desire towards a certain, coloristic effect,
reserved ultramarine for specific contexts of divine raiment or sky,
there was one artist that stood out for having a different---even
unorthodox---vision for ultramarine. This artist was Johannes
Vermeer, a painter of the Dutch Golden Age, and to say that Vermeer had
a predilection towards ultramarine would be an understatement---the
color appears in almost every one of his canvasses, even if its presence
does not quite directly meet the eye (Kühn, 1968, p.175). To date,
nobody quite understands how Vermeer was able to acquire such copious
amounts of ultramarine that he so liberally applied to his paintings;
many suspect that his practices certainly ran his family into an
enormous amount of debt (Levy-van Halm, 1998; Montias, 1998). Still,
like so many artists of his day in their pursuit of this perfect color,
he remained undeterred by the ramifications of cost in staunch
dedication to his artistic vision.

In some of his paintings, Vermeer's employment of ultramarine is more
straightforward, traditional even. His \emph{Girl with the Pearl
Earring}, \emph{The Milkmaid}, and \emph{Woman in Blue Reading a Letter}
features ultramarine in contexts not unlike that of the iconic raiment
of the Virgin Mary, its primary function to make an immediate statement
of color. In the case of the \emph{Woman in Blue Reading a Letter}, the
woman's entire gown---from the areas illuminated by light to the
shadows of its creases---are comprised entirely of ultramarine, with
only varying quantities of white and black incorporated to adjust for
shade (Costara, 1998, p.157), attesting not only to the sheer power of
the color, but also its versatility and complexity. However, Vermeer's
usage of ultramarine in other paintings is much more subtle, the pigment
incorporated in ways hidden from direct attention and detection,
employed in contexts that were entirely original for his time. In the
\emph{Young Woman with a Water Pitcher}, the painter mixes ultramarine
with other tones to create not a direct coloristic effect, but one of
light and shade (Costara, 1998, p.157). In this painting, one's
perception of ultramarine is not the usual experience of intense blue,
but rather an experience of luminance in the form of transparent window
panes, cool lighting cast upon white walls, and most enchantingly, the
paradoxical radiant shadow within the folds of the woman's head scarf.
Even more oblique a usage perhaps is the employment of the color as an
undercoat to the striking red dress of \emph{The Girl with the Wine
Glass}, the intense blue hues obscured completely from our conscious
view, the color instead manifesting itself as texture and realistic
dimension of the creases and folds of the dress (Villa, 2012, p.64). And
truly masterful is Vermeer's application of ultramarine in \emph{The
View of Delft}---a painting to which many consider as one of the
most beautiful cityscapes in western art---not as immediate color of
sky or water, but rather to the foliage of the leaves and shingles of
the rooftops, creating a sense of depth and a distance across both space
and time---an uncanny coexistence between intense realism and the
dreamy abstraction of nostalgia for the painter's hometown.

The study of Vermeer's unorthodox treatment of ultramarine is an
ultimate study of the interconnectivity of color and the other domains
of creative artistic forces, an elegant testament to the artist's unique
innovation and contribution to the way we view color and an artist's
palette, and the breadth of its expressive implications. While some of
Vermeer's applications of a color so rare, costly, and bold as
ultramarine---such as using it as an undercoat, or mixing it in with
other colors until it is barely visible in a decontextualized
setting---may have seemed initially counterintuitive or profligate,
the resulting artworks speak volumes towards the artist's intuitive
understanding of the vast applications and multi-dimensionality of
color's aesthetic forces, and just how symbiotic the artistic experience
and our perceptual systems are. Transcending Vermeer though, the study
of ultramarine and its usages are a larger testimony to the power of a
single color and its irreplaceability, and the importance of an artist's
dedication to the fine details of their craft, as small changes in one
domain can lead to whole scale shifts of the composite artistic result,
as the aesthetic integrity of these above artworks would be
significantly altered without the force of ultramarine and its sublime,
unique properties.
\iftoggle{pretty}{}{\vspace{\baselineskip}}
\par\noindent\hfil\rule{0.5\textwidth}{.4pt}\hfil\par
\iftoggle{pretty}{}{\vspace{\baselineskip}}
    This is not a dissertation on visual color. However, the discourse above
bears a significant metaphorical relationship to the central focus of
this thesis, which is essentially a study of color and its creative
agents within the auditory domain. Absolute analogies between different
perceptual domains can be tricky to draw, and the connection between the
exploration of auditory color in this thesis and the illustrations of
visual color above are in no ways meant to be exact and literal
translations, but rather serve to prime the reader to the types of
motivations and thought processes that will guide the objectives and
analyses of this thesis.

On the highest level of ideological abstraction, this is a dissertation
on musical color---more specifically, an inquiry into the role that
the subtleties of musical color have to play on impacting and shaping
larger scale musical structures, and ultimately, their relationship to
the aesthetic integrity and artistic significance of the resulting
compositions. The agents of musical color---which I am in no way
claiming to be a sufficient agent---that are the focus of this
discourse are the sonic building blocks of musical intervals, macro
musical intervals (akin to larger color groups), and even more
specifically, micro-variations of intervals in the form of musical
temperament (akin to the nuances of shade and hue), the latter in which
I will discuss in Chapter 2. Comprehensively, this dissertation is
interested in the same questions of significance, irreplaceability, and
utility of musical color from the vantage point of the nuances created
by tempered musical intervals.

While I wish to be able to establish and characterize the relationship
between musical intervals and artistic significance in the course of
this thesis, the analytical approach adopted is designed to safeguard
against the subconscious initial promotion of a preconceived agenda, and
does not push towards an early positive truth value of the relationship
between musical intervals and artistic significance. Instead, it poses
these ideas stated above as questions, approaching the subject from as
agnostic a point of view as possible, and allowing the underlying
structure of the musical composition to unfold on its own, and dictate
the conclusions that we will ultimately arrive at by the end of the last
analysis chapter. The analytical forces that I will employ are
interdisciplinary and leaning heavily towards the quantitative,
supplementing traditional score analysis with statistical, and
computational modalities, and may wax technical at points; however, this
dissertation is at its heart a musical discourse, interested in musical
meaning and significance, and any analysis embarked upon will always
contain an application and relation towards larger musical goals and
interpretations. Finally, on a broader scale, through the inquiry,
analysis, and results of this thesis, I hope to bolster through
quantitative means the validity of artistic intuition towards this
subject in both believers and skeptics alike, and to foster a more
complete and comprehensive understanding and appreciation towards the
musical elements that contribute towards our rich and complex artistic
experiences as theorists, artists, performers, and collective recipients
and guardians of these invaluable works of art.

    \section{Objectives}\label{objectives}

On a more concrete level, the central focus of this project is to
examine the role and importance that temperament exerts in the
compositional formation and musical integrity of J. S. Bach's
\emph{Well-Tempered Clavier}---specifically, the objective is to
demonstrate that the tuning system that Bach was working with was not
only a determining factor in shaping his compositional process given a
certain key, but more importantly an integral, indispensable element of
the composition's overall musical affectation upon sonic realization.
The main claim that I wish to test in this dissertation is that Bach's
compositional process was distinctly a function of temperament. In other
words, Bach was writing the Well-Tempered Clavier with a well-tempered
turning system in mind, as opposed to a system that more resembled
equal-tempering, and furthermore that this factor of tuning manifests
itself to the listener through producing tensions and harmonies that are
unique to that particular system. Essentially, the claim is that the
tuning system---which governs the sizes of intervals---exerts direct
control over the affect and musical integrity of each individual piece
in the collective work, and that this was Bach's express intent, and to
remove it and use a different tuning system in realization of the piece
would not only effectively eradicate a dimension of the music, but also
in fact alter the very aesthetic makeup and integrity of the piece as a
whole.

Because of its tonally complete and cyclical nature, the Well-Tempered
Clavier lends itself well to the objective study and exploration of both
temperament and key (Mies, 1948, p.12; Steblin, 2002, p.7), the two main
elements in which this dissertation seeks to determine a connection
between, and for the sake of consistency of the data, and also in
interest of scope, I will only be focusing on the minor mode fugues of
the first volume. The reasons for these specific choices and parameters
will be elaborated and expounded upon more in chapter three of this
dissertation, which addresses in detail hypotheses and methods of
analysis, essentially picking up where the outline of objectives and
questions in this chapter ends after a brief historical second chapter.

A broader goal of this dissertation is the underscoring of the value
that analysis dealing directly with temperament has upon achieving a
more complete musical understanding of pieces realized under these
specific temperamental settings, in this case, Bach's compositions and
the well-tempered system. Just as we parse a musical composition for
elements of harmony, rhythm, and motivic structure in order to gain a
deeper understanding of a piece's underlying musical integrity, this
dissertation seeks to demonstrate salient reasons why we need to be
asking similar questions and conducting similar types of analyses with
focus on temperament. At the core of what this dissertation really seeks
to show is that temperament is an integral and indispensable component
that has played a very immediate role in Bach's artistic vision for the
Well-Tempered Clavier, and just as much we consider other compositional
elements to be essential factors to musical integrity, it is equally
important to consider temperament. Temperament as a musical dimension
has largely been ignored in musical scholarship, and is something that
we must be engaging in if we are indeed to develop a full understanding
of a piece of music.

Finally, a tandem analytical/methodological goal is the engagement and
application of computational musicology as the main method of analysis
in conjunction with traditional means of score analysis, demonstrating
the importance that computational and quantitative methods have upon
providing a more measurable and concrete way of addressing the
traditionally abstract and under-analyzed realm of temperamental
analysis. The computational framework consists of expanding upon Michael
Cuthbert's symbolic music analysis Python toolkit \texttt{music21}
(Cuthbert, 2011), the current lingua franca for musical symbolic data
analysis (Tymoczko, 2013; Quinn, 2014).

The details on the motivation behind focusing on computational
musicology will be unpacked in the third chapter regarding methods, and
the necessity of an inquiry from this particular vantage point addressed
in the next chapter focusing more upon history and literature, but
immediately, the main virtue of computational musicology is that it
allows us to look over a vast corpus to determine trends and statistical
significance of the data that we would otherwise not be able to by
traditional/manual means, allowing us to draw statistically significant
conclusions as well as test our hypotheses with the absence of human
bias (Meredith, 2016, p.25; Volk, 2011, p.138; Huron, 2002), the latter
which especially becomes more difficult to control for as data sets get
larger. By combining computational means with traditional, I hope to
cover the most ground to simultaneously address the topic of temperament
in terms of both statistically and musically significant discourse. To
date, scholarship and discussions on temperament have been mostly
qualitative, or in the case of quantitative studies, done manually,
which limits researchers to a specific region of the corpus, making it
difficult to draw systematic and statistically significant conclusions,
as well as reliably guard against preconceived notions or agendas. It is
my hope that recruiting the relatively recent field of computational
musicology and a novel framework of analysis that involves the
amalgamation of computational, statistical, and traditional methods that
has not yet been applied to temperamental studies will bring to the
table a different type of authority into the discussion of the role of
well-temperament within the musical experience, serving to not only
bolster our understanding of the significance of temperament in musical
discourse, but also bring to attention the capabilities and virtues of
computational musicology, and the potentially important contributions
and role that it can play within the existing cannon of musical
analytical methods.

    \section{Research Questions}\label{research-questions}

The following questions are those that I wish to address through my
analysis, that are systematically laid out to ultimately address the
issue of temperament's role and importance in the compositional process
and resulting musical integrity of a piece.

To properly set up the proceeding questions though, the following
premises should be provided to help establish historical and literary
context; these points are not assumptions, but are rather derivable
through historical context and literature, as will be discussed in
Chapter 2, but are first presented here to establish a foundation of
what is already known and understood in the field for the objectives and
questions of this thesis to build from and stand upon:

\subsection{Premises}\label{premises}

\begin{enumerate}
\def\labelenumi{\arabic{enumi}.}
\tightlist
\item
  We have strong evidence that the Well-Tempered Clavier was conceived
  under some form of well-temperament, as opposed to meantone or equal
  temperament (Reinhardt, 2004; Ledbetter, 2008).
\item
  There is a precedence for the relationship between historical tuning
  systems and compositional styles and interval predilection during
  their respective eras prior to the advent of well-temperament,
  specifically the medieval period and Pythagorean tuning, and the
  Renaissance period with meantone temperament (Lindley, 1980; Schulter,
  2002).
\item
  There exists a robust body of qualitative, theoretical, and anecdotal
  support from scientists, philosophers, theorists, composers, and
  performers that attests to the system of well-temperament being an
  important and desired dimension of artistic influence and resource
  (Jorgensen, 1999).
\item
  There are partial, manually generated quantitative studies that point
  towards the notion that Bach's compositional and contrapuntal process
  (e.g. the raw frequency of usage of certain intervals, as well as
  varying approaches to texture and voicing that either draw attention
  to or away from certain intervals) was at least, in part, a function
  of well-temperament (Barnes, 1979; Reinhardt, 2004).
\end{enumerate}

\subsection{Questions}\label{questions}

\begin{enumerate}
\def\labelenumi{\arabic{enumi}.}
\tightlist
\item
  Are we able to observe statistically significant, robust, and
  systematic internal evidence from the score alone that attests to
  temperament's relationship with key? What does the effect of
  temperament look like in relationship to key? Can we extend this to
  composition beyond mere key effects, and can this be demonstrated in a
  consistent, large-scale, and comprehensive way through quantitative
  methods (i.e., of analyzing temperament through the aid of
  computational methods and statistical distributions)?
\item
  If (1) is affirmed, on what levels of scope and parameterization are
  these effects of temperament observable, and what compositional
  elements of the music do temperament exert effects upon? How is this
  specifically demonstrated through Bach's compositional choices (i.e.
  what are the correlations between key, interval choice and usage,
  motifs, and treatment of counterpoint)?
\item
  If (2) is affirmed, can we establish that temperament is truly an
  integral factor to the overall musical integrity of a specific piece.
  If so, how is the structure of a piece inherently altered through
  temperament and tempered intervals? Or alternatively, are the effects
  more functional, and is the aesthetic and affectual implication of a
  piece essentially unaltered by temperament?
\item
  How do these aforementioned tempered intervals, musical elements, and
  resulting larger scale structures affect a composition's aesthetic
  outcome, and dissonance/consonance profile? Furthermore, how does
  temperament tie directly into the experience of affect and key
  characteristics in the Well-Tempered Clavier?
\end{enumerate}

In a broader, contextualized sense:

\textbf{Can we affirm through quantitative means the qualitative
accounts and intuitive conviction that artists, theorists, and
philosophers have had throughout the centuries that the notion of key
characteristics is in fact a statistically justifiable phenomenon, and
is inextricably linked to well-temperament?}

The purpose of addressing each of these questions is to underscore the
importance of considering temperament when discussing the underlying
large-scale structural elements within Bach's Well-Tempered Clavier, to
illuminate what is gained in terms of understanding and affect when one
analyzes and performs the work under the lens of well-temperament, and
conversely what is lost when this is neglected, and a different system
(e.g. equal temperament) is employed in its stead.

    \section{Layout of Inquiry and
Argument}\label{layout-of-inquiry-and-argument}

Having now provided the research questions of this dissertation, the
concise, step-by-step layout of the overarching logical process that my
inquiry will follow is delineated below. My central goal is to determine
whether or not there exists statistical support for and a reasonable
belief to a positive correlation and causation amongst these following
independent, intermediate, and dependent variables.

This dissertation will analyze and examine how---

\begin{enumerate}
\def\labelenumi{\arabic{enumi}.}
\tightlist
\item
  \textbf{Musical temperament} (\emph{input, independent variable}), in
  this case, unequal, well-temperament, directly controls the
\item
  \textbf{Statistical distribution of intervals and choice of interval
  selection} (\emph{second level, intermediate variable}), which in
  turns affects the approach to and treatment of:
\item
  \textbf{Compositional elements} (\emph{third level, intermediate
  variable}) which include

  \begin{enumerate}
  \def\labelenumii{\arabic{enumii}.}
  \tightlist
  \item
    Temporal and durational elements
  \item
    Textural thickness
  \item
    Harmonic implication
  \item
    Higher level, large-scale structural architecture
  \item
    Motivic salience and symbolism
  \item
    Fugal elements and techniques (subject and countersubject choice,
    usage of stretto)
  \item
    Approach to chromaticism
  \item
    Balance of dissonance and consonance (motion and stasis)
  \end{enumerate}
\end{enumerate}

...that ultimately exerts a determining influence upon our perception of
4. \textbf{Musical Affect, Key Characteristics, and Aesthetic Integrity}
(\emph{output, dependent variable}) in J. S. Bach's Well-Tempered
Clavier.

    \section{Overview of Chapters}\label{overview-of-chapters}

Chapter 3 of this dissertation will go into these objectives in greater
depth, defining more clearly the dissertation's parameters of inquiry,
providing a comprehensive outline for the analytical and computational
methods employed and their respective goals, and framing the thesis in
terms of null and alternative hypotheses with sets of expectations that
should result from each scenario. However, before we delve into these
methodological details, it is important to give a brief overview of the
history of tuning and temperament in order to put things into context.
Chapter 2 will discuss temperament, providing a comprehensive
explanation of what it is, why it exists, and most importantly, why it
is important---and deserving---of the focus of our attention, and what
can be gained from such an angle of inquiry. The next chapter will also
address the significance of a study on temperament, and provide reasons
as to why such a study, especially from the vantage point that is taken
in this dissertation, is valuable and very much needed. Lastly, it is
important to stress that the discourse on related literature and
historical overview in chapter 2 is not so much for didactic purposes as
it is to demonstrate evidence supporting the critical premises listed in
this chapter, from which the construction of the main questions of this
thesis rest upon, as well as support the choices for the analytical
methods in Chapter 3.

Following the historical and methodological chapters 2 and 3 will be the
analysis chapters (4-6), divided into three portions organized by the
scope, progressing from broad/general to narrow/local, with the
analytical inquiry and goals of each chapter directly informed from the
findings of the previous one. Chapter 4 serves as a survey on the
broadest level, aiming first to illustrate the effects of
well-temperament from the data (i.e. intervals within the minor mode
fugues of the WTC I) alone, and establish any broad trends of
correlations between temperament and key within the corpus. Chapter 5
takes the trends observed in chapter 4 and analyzes them on a more
detailed level, adding certain musical parameters such as duration,
texture, specific intervals, and scale degrees to ascertain the
relationship between specific musical elements and temperament. Lastly,
chapter 6 will narrow the aperture down even further to the motivic and
fugal level, focusing on how temperament has affected these compositions
on a thematic and musical level. Each analysis chapter also provides a
summary and conclusion of the studies contained within it, and chapters
5 and 6 include an opening portion which summarizes the conclusions from
the chapter preceding, designed to aid the reader interested in a more
modular approach to parsing the dissertation. Finally, a conclusion
chapter will synthesize and present the cumulative conclusions derived
across the three analysis chapters on the scope and implications of
temperament's relationship with the compositional process and musical
impact of Bach's WTC I, and end with a discussion of the greater
significance of the study in its relationship with the performer as well
as the wider musical and artistic audience.


    % Add a bibliography block to the postdoc
    
    
    
