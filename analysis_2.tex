    
    
    

    \hypertarget{Analysis II: Intermediate Analysis}{\chapter{Analysis II: Intermediate Analysis}\label{Analysis II: Intermediate Analysis}}


    \section{Intermediate Analysis
Objectives}\label{intermediate-analysis-objectives}

    Having established in the previous chapter a foundational relationship
between temperament, key area, and composition, this intermediate
section of analysis will continue the investigation on these broad,
correlative elements through narrowing the distributions to that of
select, relevant subsets of intervals and musical elements, with the
purpose of examining how temperament interacts with these specific
musical forces to enhance and inform underlying musical structures. As a
brief recapitulation, the summary of the three principles established by
the previous sections are:

\begin{enumerate}
\def\labelenumi{\arabic{enumi}.}
\tightlist
\item
  The raw effects exerted by temperament on key are systematic and
  gradated according to the circle of fifths, and have the effect of
  dividing the keys into two rough groups, Pythagorean and Meantone,
  based on their specific just and tempered interval profiles.
  Moreover-\/-\/-
\item
  These temperament traits are mirrored in the underlying compositional
  structure and preserved across transposition, indicating that
  temperament exerts a certain amount of influence on the choice of
  intervals and compositional elements employed based on key area.
\item
  These temperamental effects are more pronounced on the horizontal
  domain than vertical.
\end{enumerate}

With these three principles from the previous chapter in place, the
overarching objective of this intermediate chapter is to begin asking
questions regarding the musical significance and implications of these
general, statistical trends. More specifically, this chapter will begin
to explore which underlying musical elements, intervals, and features
have been impacted by temperament, how these elements further contribute
to the trends and division between Pythagorean and Meantone observed in
the KDE plots and histograms in the previous chapter, and ultimately
discuss how the aggregate effect of this interaction impacts deeper
musical and thematic structures that we will explore in the final
chapter of analysis.

Below is the following outline further breaking down the types of
analyses employed in this chapter, and their respective significances
and goals:

\subsection{Outline of Analysis}\label{outline-of-analysis}

\begin{enumerate}
\def\labelenumi{\arabic{enumi}.}
\tightlist
\item
  Untempered and Tempered General Plots:

  \begin{enumerate}
  \def\labelenumii{\arabic{enumii}.}
  \tightlist
  \item
    Non-interval musical elements: This first subsection will
    momentarily remove the focus from intervals, and instead examine
    distributions of other musical elements that are related to
    sonority, namely note duration and vocal texture, and see if there
    are any divisive patterns between the way these elements are laid
    out in the Pythagorean and meantone key groups.
  \item
    Untempered intervals: examine the frequency of usage of select
    untempered versions of intervals (perfect and imperfect class
    intervals) as a function of individual pieces across the entire
    volume to further verify if distributions contain information to
    support relations between key and temperament patterns reminiscent
    of historical tuning traditions postulated by the KDE plots and
    histograms of the previous chapter, or if distributions are
    uniform/random.
  \item
    Tempered intervals: break down the previous untempered intervals
    into their tempered versions, and compare these against their
    transposed averaged counterparts to ascertain if there are any
    systematic predilections towards certain types of tempered
    intervals.
  \end{enumerate}
\item
  Tempered Scale Degree Plots: Plots frequency of the occurrence of
  select tempered intervals (P5, m2, m3) as a function of scale degree,
  both within a specific piece, and across pieces of differing keys.
  These plots further narrow and focus the information obtained by the
  previous untempered an tempered general plots by relating harmonic and
  melodic function to temperament, looking at the placement of where
  certain tempered intervals lie within the scale degrees of specific
  keys to examine connections between temperament and voice leading,
  potential for a key's harmonic sonority, and modal stability.
\end{enumerate}

Following this outline for analysis, the resulting outline of the goals
to be achieved are as follows:

\begin{enumerate}
\def\labelenumi{\arabic{enumi}.}
\tightlist
\item
  Identify specific musical intervals and elements, and their respective
  trends across minor fugues that further demonstrate the division
  between Pythagorean and Meantone keys, discuss the deeper musical
  significance of these trends, and how they relate to a key's overall
  profile of purity and dissonance.
\item
  Further narrow the window of focus through examining key differences
  within specific key types, with focus on the Pythagorean group,
  through looking at placement of tempered intervals in relation to
  scale degree. This section will discuss how Bach uses temperament to
  aid the melodic force of voice leading, and what effect the specific
  placement of tempered thirds, pure fifths, and semitones along a key's
  scale exerts upon the modal stability, textural sonority, and
  chromatic scheme of a particular key. Ultimately, the combined forces
  of these three intervals in the scale degree graphs will allow us to
  develop a more nuanced understanding of key structure and key traits
  within the larger two key groups that we have already established from
  chapter 1.
\end{enumerate}

A note before I proceed with the analyses of this chapter: the nature of
the graphs presented in this chapter are still more or less global in
that they look at intervals spanning across multiple pieces; either
throughout the entire volume, or in an instance of the subsection
dealing with scale degrees, across pieces belonging to the Pythagorean
group of keys. The main difference though in terms of scope between this
current chapter and the previous one is that the previous chapter looked
at interval distributions in their entirety, with no filtering, in
attempt to characterize a rough temperament profile for each key, while
the current chapter is primarily concerned with looking at subsets of
intervals and musical elements, and how these vary across the different
keys and pieces. The last chapter of the three analysis chapters will
further constrain the data to a far more local and/or fugal scope, with
a focus of looking at intervals constrained by subject or thematic
context.

    \section{General Tempered and Untempered Interval
Graphs}\label{general-tempered-and-untempered-interval-graphs}

This first section explores subsets of musical elements and intervals,
and how they interact with key area, first looking at elements devoid of
temperament, then moving on to their tempered versions. Based on the
division of Pythagorean and Meantone key groups determined through the
interval key maps and KDE plots of the previous chapter, the analysis in
this chapter focuses on examining specific musical elements and
intervals that may serve to underscore this divisive effect on a
compositional level.

To put the inquiry into question format, the main questions, and the
musical motivations behind them, that this section seeks to address are:

\begin{enumerate}
\def\labelenumi{\arabic{enumi}.}
\tightlist
\item
  In the previous chapter, key profiles of the Pythagorean class have a
  much larger percentage of just intervals in comparison to their
  meantone counterparts, and when analyzed using transposition KDE plots
  and histograms, Pythagorean class compositions were found to maximize
  the retention of pure intervals over the meantone compositions by
  virtue of composition alone. Due to the nature of pure intervals
  (especially the perfect class intervals comprised of octaves, fifths,
  and fourths), more of these intervals present in a composition
  translate to more power of resonance from an acoustical standpoint.
  The first subsection asks the question of whether or not this
  potential for greater acoustical resonance in the Pythagorean keys
  will be exploited by certain musical elements-\/-\/-namely note
  duration and vocal texture-\/-\/-to allow for perhaps slower moving
  compositions with thicker textures to showcase this sonority, and
  whether or not we can see differences between treatment of these
  elements between the two main key types. More simply put: do
  Pythagorean keys as a whole tend to have average longer note
  durations, as well as a greater number of voices as opposed to
  Meantone keys?
\item
  This characteristic of greater power of resonance in the Pythagorean
  keys also begs investigation on the interval domain. The question here
  is whether or not Bach, in dealing with the Pythagorean compositions,
  chose to employ proportionally more perfect intervals in these pieces
  than those of the meantone class, as these perfect intervals in
  Pythagorean keys, being mostly pure, would substantially boost the
  composition's overall resonance. Furthermore-\/-\/-
\item
  The tradition of Pythagorean tuning, a system that retained the purity
  of the perfect intervals at the expense of the imperfect ones,
  resulted in a musical style that was essentially perfect interval
  centric. Meantone temperaments, on the other hand, are dictated by the
  reverse trend, which upholds the purity of imperfect intervals at the
  sacrifice of the perfect ones, and likewise saw a shift in a musical
  style and harmonies that emphasized thirds and sixths as opposed to
  fifths and fourths. On an acoustical level, this is in line with the
  retention of purity, however, the musical styles of those respective
  eras were not so much motivated by the avoidance of the less pure
  intervals (as they were not by any means close to wolf intervals), but
  about emphasizing the strengths and characteristics of the pure ones.
  This leads us to the following question: Given this predilection of
  pure perfect intervals in the tradition of Pythagorean tuning, and
  pure (or in the case of well-temperament, close to pure) imperfect
  intervals in the tradition of Meantone temperaments, is this likewise
  reflected in the well-temperament division of Pythagorean and Meantone
  keys? In other words, can we observe a trend of greater proportions of
  fifths and fourths in the Pythagorean key fugues, and a reversal of
  that-\/-\/-a larger proportion of sixths and thirds-\/-\/-in the
  Meantone keys?
\item
  The previous two points (2 and 3) pose questions involving
  distributions of intervals with the dimension of temperament
  collapsed, i.e., they ask the question of, "Did Bach use more fifths",
  rather than, "Did Bach use more pure fifths in relation to tempered
  fifths"; this last subsection will bring back the dimension of
  temperament to these intervals, and essentially focus on the latter
  question. Following the logic of the previous two points, Bach has two
  ways of accentuating the purity/tempered profile of a key: he can
  operate on a non-temperament level, and maximize the overall usage of
  certain intervals in their entirety (e.g., using more fifths overall
  in Pythagorean keys), or he can operate on a more subtle, temperament
  level and choose specific types of intervals based on their level of
  tempering (e.g., using a greater ratio of pure fifths than expected by
  the value of the averaged fugues).
\item
  We have witnessed consistently stronger temperamental effects on the
  horizontal domain as opposed to the vertical in the previous chapter,
  motivating us to analyze the semitone as a potentially important
  interval contributing to temperament's relationship to function and
  musicality, especially in terms of the subject of chromaticism and
  voice leading in minor mode fugues. Just as we have reason to believe
  in the importance of the perfect and imperfect class intervals in
  regard to key for reasons of historical precedence, what does the
  usage and placement of the semitone add to our understanding of the
  characters of keys in an even more nuanced way? We have seen the large
  division between Pythagorean and meantone key groups in the previous
  chapter; through looking at scale degree graphs in this chapter, are
  we able to come up with finer divisions within key groups, and if so,
  what are the traits of these key subgroups? Lastly, in regards to
  idiomatic intervals and historic tuning styles, the Pythagorean limma
  (narrow Pythagorean semitone) was the semitone that belonged to the
  Pythagorean tuning system (just like the just fifth and the
  Pythagorean minor third); what can scale degree graphs tell us about
  the placement of this interval within the Pythagorean class keys, and
  how does this placement in different Pythagorean keys possibly create
  finer grained distinctions within this key class? Cumulatively, what
  can we garner about key structure from looking at fifths, thirds, and
  semitone placement collectively?
\end{enumerate}

    \subsection{Non-Interval Musical Elements
Plots}\label{non-interval-musical-elements-plots}

There are two plots presented in this subsection, the first which looks
at average note duration as a function of key across the entire volume
of minor fugues in book I, and the second which looks at the number of
voices in each fugue.

    \subsubsection{Note Duration Plot}\label{note-duration-plot}

Below is the graph that lays out the mean value of the note durations
for each minor mode fugue. The y-axis contains the note duration values
in the unit of quarter lengths (1 = 1 quarter note, .5 = eighth note.
etc.), and the x-axis has the fugues laid out in order of the circle of
fifths, starting from f minor and progressing in retrograde throughout
the Pythagorean keys, then the meantone keys, until the cycle terminates
at c minor. It is important to underscore that in this graph, x-axis
values do not represent summed values, but averaged values. The
assignment of keys to color and shade are exactly consistent to the
previous chapter, with the Pythagorean keys assigned shades of blues,
and meantone shades of red, and the exemplar keys of each category
possessing the darkest shades.




\begin{figure}[H]
    \begin{center}
    \adjustimage{max size={.7\linewidth}{0.9\paperheight}}{analysis_2_files/analysis_2_9_0.png}
    \caption{Mean Note Duration in Minor Fugues.}
    \end{center}
\end{figure}
    
    Looking at the mean values for the fugues assembled in the above graph,
the following is a list of observations and interpretations:

\begin{enumerate}
\def\labelenumi{\arabic{enumi}.}
\tightlist
\item
  All of the means for the Pythagorean (blue) keys lie above the means
  for the Meantone (red) keys, with the exception of the boundary key of
  f minor
\item
  The mean value for f minor is the lowest value for the Pythagorean
  keys, and is equivalent to the value of the highest bar for the
  meantone keys, g minor.
\item
  The variance of the mean values for the Pythagorean group is much
  greater than that of the meantone group, which values are more
  compactly clustered together. Three of the Pythagorean fugues, b-flat,
  c-sharp, and f-sharp, have extreme high peaks that are roughly twice
  as high as the other Pythagorean values, and three times as high as
  the meantone values.
\end{enumerate}

The straightforward, initial summary we can arrive at from observing
means in the duration graph is that the average durations of notes are
significantly higher in Pythagorean keys than in the meantone keys.
However, if we want to look closer at how individual datapoints
(representing individual note durations) are distributed within each
fugue, we can break down the data into the format of box and whisker
plots. This will provide us much more detailed information on the median
of the data, as well as how the individual quartiles are distributed,
and their respective ranges, which will allow for a deeper understanding
on which types of note durations Bach is overall choosing, and in what
approximate frequency, for each specific fugues.



\begin{figure}[H]
    \begin{center}
    \adjustimage{max size={0.9\linewidth}{0.9\paperheight}}{analysis_2_files/analysis_2_12_0.png}
    \caption{Note Duration in Minor Fugues Boxplots.}
    \end{center}
\end{figure}
    


\begin{figure}[H]
    \begin{center}
    \adjustimage{max size={0.9\linewidth}{0.9\paperheight}}{analysis_2_files/analysis_2_14_0.png}
    \caption{Note Duration in Minor Fugues Boxenplots.}
    \end{center}
\end{figure}
    
    \subsubsection{Vocal Texture Thickness
Plot}\label{vocal-texture-thickness-plot}

This next plot will chart the number of voices each fugue contains as a
rough means of representing the level of thickness of vertical/chordal
texture for each piece. It must be said that obviously, such a general
parameter can only provide us with very broad information, as there are
numerous additional intricate and nuanced factors that contribute to
textural thickness in a composition. Nonetheless, the number of vocal
forces is a salient compositional factor of a fugue, and one that not
only correlates directly with vocal thickness-\/-\/-which impacts
sonority and resonance-\/-\/-but also is very much a compositional
choice on Bach's behalf, so it stands that it is a reasonable starting
point of evaluation.




\begin{figure}[H]
    \begin{center}
    \adjustimage{max size={.7\linewidth}{0.9\paperheight}}{analysis_2_files/analysis_2_18_0.png}
    \caption{Number of Voices in Minor Fugues.}
    \end{center}
\end{figure}
    
    While not as dramatic as the duration charts in the previous chapter
(which, given the limited range and values that the parts can take on,
is to be expected), it is still evident that the number of voices tend
to be higher in the Pythagorean fugues as opposed to the meantone group.
Observations are as follows:

\begin{enumerate}
\def\labelenumi{\arabic{enumi}.}
\tightlist
\item
  The number of voices in each key type's fugues, presented in a vector
  cataloging values from large to small is: Pythagorean:
  {[}5,5,4,4,4,3{]}; Meantone: {[}4,4,4,3,3,2{]}, showing a clear weight
  of greater number of voices in the Pythagorean fugues.
\item
  The two fugues, b-flat and c-sharp, contain both the highest number of
  voices (5 voices), as well as the longest overall note durations (and
  markedly so). Both these musical traits stand out as anomalous in
  comparison with the other fugues, even of their own Pythagorean ilk;
  the fact that both these special features (5 voices, and extra long,
  sustained notes) coincide upon the same two Pythagorean class fugues
  suggests that these two traits are not only connected on some musical
  level, but were also deliberate decisions placed upon the compositions
  to collectively bolster an effect on the acoustical sonority of those
  keys.
\item
  These same two fugues were also the only ones in the previous
  KDE/histogram section whose original keys maximized the percentage of
  just intervals over all other fugues transposed to that key on the
  vertical domain. Interestingly, while both these fugues have the
  forces of 5 voices, and create these chorale-like textures through
  sustained notes, they each win out in terms of purity in their home
  keys, i.e., the b-flat minor fugue, when transposed to c-sharp, does
  not do as well as the original c-sharp minor fugue in terms of
  maximizing purity of intervals, and likewise the c-sharp minor fugue,
  when transposed to b-flat, loses out to the original b-flat minor
  fugue in the domain of purity. (\emph{See KDE plots and histograms
  below.}) This elegant result suggests that, while Bach composed two
  fugues that would both feature sonorous and resonant textures
  partially through the usage of sustained notes and thickening the
  vocal forces, each composition was somehow structured in a way that
  its original keys would maximize the purity of its intervallic forces.
\end{enumerate}



\begin{figure}[H]
    \begin{center}
    \adjustimage{max size={0.9\linewidth}{0.9\paperheight}}{analysis_2_files/analysis_2_21_0.png}
    \caption[B-flat minor and c-sharp minor transposition comparison vertial KDE plots and histograms. ]{B-flat minor and c-sharp minor transposition comparison vertial KDE plots and histograms. The left graphs contain both b-flat and c-sharp minor fugues transposed to the key of b-flat; the right graphs contain these same fugues both transposed to the key of c-sharp. Note that in both cases, the original key maximizes the value on the purity dimension (0 cents from just on the x-axis).}
    \end{center}
\end{figure}
    

    \begin{center}
    \adjustimage{max size={0.9\linewidth}{0.9\paperheight}}{analysis_2_files/analysis_2_22_0.png}
    \end{center}
    
    \subsubsection{Subsection Summary}\label{subsection-summary}

The various note duration and vocal texture plots presented in this
section yield a definitive result that these compositional elements vary
in systematic ways as a function of key, as well as the two key groups,
Pythagorean and Meantone. Specifically, Pythagorean keys are
characterized by overall longer note durations and thicker vocal texture
(as measured by number of voices contained within the fugues), and
Meantone keys are characterized by shorter note durations and sparser
vocal texture (fewer voices).

These trends are in direct support of the musical and acoustical
interpretation of the Pythagorean class keys maximizing the pure class
intervals to heighten the effects of sonority and resonance built into
these particular minor keys through the system of well-temperament, as
displayed in the interval maps and KDE plots of the previous chapter.
This connection between these elements and a key's built-in purity
potential is further bolstered by the example of how the elements of
extreme long note durations, high number of voices, and maximum ratio of
pure intervals coincide in the two keys of b-flat and c-sharp minor, and
furthermore, how these keys individually maximize their purity ratios
above all other transposed fugues.

Lastly, without unpacking too much in detail yet, the musical elements
explored in this subsection, with their underlying musical and
acoustical effects, could be seen to increase a certain quality of
"minorness" in these Pythagorean class minor keys; we will continue to
break this notion down further in the next large section of this chapter
that deals with scale degrees, and again in the next chapter involving
local and fugal analysis on individual Pythagorean fugues. However, what
we can observe quite readily at this particular juncture is the
emergence of this notion of fledgling "key characteristics" embedded
within these minor mode fugues; in more concrete terms, underlying
compositional elements that correlate directly with key centers and
profiles defined by temperament, that give rise to different perceived
elements of musical expressivity and affects.

    \subsection{Untempered Intervals}\label{untempered-intervals}

Having diverged from intervallic analysis for a bit in the previous
section, this section returns the focus back to examining intervals,
more specifically, looking at how particular intervals, or small subsets
of intervals, vary as a function of key, or key group. The section above
dealing with non-interval musical elements, along with the previous
analysis chapter on KDE plots, already begin establishing the
relationship between key-\/-\/-in particular, key group (Pythagorean or
meantone) and composition. The results from the analysis in the previous
section demonstrate that there are notable differences between the way
fugues from the two different key groups, meantone and Pythagorean,
treat their approaches to overall note duration and vocal texture, in
that fugues from Pythagorean keys tend towards longer note durations and
thicker vocal textures than their meantone counterparts. For the
Pythagorean group of keys, these two factors work in conjunction with
the larger ratio of pure perfect intervals to contribute to the effect
of greater sonic resonance in these compositions.

This section will continue to explore the possible differences between
the Pythagorean and meantone group of keys, this time focusing on the
differences of interval usage to see if Bach uses certain intervals more
frequently in some keys that others, and whether or not these variations
follow some sort of musical pattern dictated by key. This can be
approached in two manners, the first of which looks at intervals
regardless of tempering, which will be the interest of this section, and
the second of which looks at distributions of tempered versions of
intervals, which will comprise the next section. These two sections
combined will seek to address two similar questions:

\begin{enumerate}
\def\labelenumi{\arabic{enumi}.}
\tightlist
\item
  Untempered analysis: Does Bach choose to use certain intervals on a
  whole more frequently as a function of key and key group, as dictated
  by historical precedence? In the case of this chapter's analysis, does
  Bach favor the usage of perfect intervals (fifths and fourths) more in
  Pythagorean keys, and the usage of imperfect intervals (thirds and
  sixths) more in meantone keys?
\item
  Tempered analysis: On a more detailed level concerning distributions
  within individual intervals, does Bach choose to use more or less of a
  certain tempered variety of a particular interval based on key, and
  are these differences consistent and predictable? For example, pure
  fifths are more common in Pythagorean keys than meantone keys sheerly
  from the virtue of temperament, but does Bach exaggerate this
  difference through conscious compositional choice? That is to say,
  would Bach choose to use more pure fifths in Pythagorean keys, and
  more tempered fifths in meantone keys above the expected ideal values
  for that particular key?
\end{enumerate}

\textbf{Pythagorean Perfect Intervals vs. Meantone Imperfect Intervals}

The intervals that I will be focusing on in these next two sections are
all intervals from the perfect and imperfect class intervals, namely
fifths, fourths, minor thirds, and sixths. The motivation behind
focusing on perfect and imperfect consonant intervals is that these are
the intervals that are inherently important in the tuning traditions of
Pythagorean and meantone temperament, and directly influenced the
compositions conceived during their respectful eras. Compositions
written during the medieval and early Renaissance time periods, in which
Pythagorean tuning was the prevalent system, were very predominantly
structured around perfect intervals-\/-\/-namely fifths-\/-\/-for both
melodic and harmonic material; the advent of meantone temperament in the
15th century shifted this compositional focus to the imperfect
intervals, structuring harmony upon thirds and sixths instead of fifths
and fourths.

The system of well-temperament, which, unlike its Pythagorean and
meantone predecessors, is an unequal tunings system, and as we have
observed from the analysis from the previous chapter, preserves features
of both meantone temperament-\/-\/-close to pure imperfect intervals at
the expense of tempered perfect intervals-\/-\/-in the diatonic keys,
and Pythagorean tuning-\/-\/-pure perfect intervals at the expense of
tempered imperfect intervals-\/-\/-in the chromatic keys. This section
seeks to test whether or not this division of Pythagorean and meantone,
and their respective interval predilections, is also reflected from a
compositional standpoint in regards to interval choice in the body of
WTC I minor fugues. Just like in previous sections, horizontal and
vertical domains will be explored separately, and conclusions will be
discussed, along with their musical implications, at the conclusion of
the section.

    \subsubsection{Horizontal Perfect Intervals (Fifths and
Fourths)}\label{horizontal-perfect-intervals-fifths-and-fourths}

In this subsection, I will start by examining horizontal perfect
intervals, with emphasis on fifths. The following graph charts the
frequency of all horizontal fifths as a function of fugue, normalized by
duration of the piece for fair comparison. As consistent with the
previous graphs, the Pythagorean keys are marked in blue, and meantone
in red.




\begin{figure}[H]
    \begin{center}
    \adjustimage{max size={.7\linewidth}{0.9\paperheight}}{analysis_2_files/analysis_2_28_0.png}
    \caption{Normalized Occurrences of Nontempered Fifths in Each Minor Fugue. }
    \end{center}
\end{figure}
    
    Observations:

\begin{enumerate}
\def\labelenumi{\arabic{enumi}.}
\tightlist
\item
  With the exception of the boundary keys of f minor and f-sharp minor,
  all of the Pythagorean keys are higher normalized frequency values
  than the meantone keys, indicating that melodic fifths are
  considerably more prevalent in the Pythagorean pieces than in the
  meantone ones. Pythagorean group mean and SD: 4.48, 1.30; meantone
  group mean and SD: 2.96, 0.62.
\item
  The peak of the frequency of fifths falls on the d-sharp minor fugue
  and tapers off steadily in both directions. This peak coincides
  directly with the most extreme Pythagorean keys-\/-\/-d-sharp minor
  and b-flat minor-\/-\/-the two keys that contain the highest
  percentage of pure fifths as oppose to the tempered variety.
\item
  The shape of the distribution is roughly bimodal, with the main peak
  at d-sharp minor, and another minor peak at a minor. While this may
  seem strange at first, as intuitively, we might expect to see just a
  unimodal curve, this actually reflects the breaking points in the
  progression from pure to tempered fifths as we cycle through the
  circle of fifths. In Werckmeister III, of the 12 fifths available
  through the chromatic scale, 8 fifths are pure, and 4 are tempered:
  three fall on the tonic fifths of the diatonic keys of c, g and d (a
  and e contain pure tonic fifths), and the fourth falls on the tonic
  fifth of b. This break in the continuity of tempered fifths between
  the keys of d, g, c, and the key of b coincides with the dip in the
  distribution of fifths across the keys.
\end{enumerate}

To further expound upon items 2 and 3 above, below are the graphs of the
percentages of pure fifths for each key; the graph on the left charts
the values from the actual fugues, and the graph on the right charts the
averaged value across all transposed fugues, a representation of the
``ideal'' value for each particular key. For example, the value of 98.52
for the d-sharp minor bar on the left graph means that of all the
horizontal fifths present in the d-sharp minor fugue, 98.52\% of them
are pure. The value of 95.08 for the same d-sharp minor bar on the right
graph means that after transposing each minor fugue from Book I to the
key of d-sharp minor and normalizing for length, the percentage of pure
fifths of all of these transposed fugues combined is 95.08\%. Both
values are very high (as expected), but as we can see in this case, the
actual fugue's value is higher than the ``ideal'' value derived from the
averaged transposed fugues. We will be analyzing these tempered graphs
in more detail in the next section, but have provided it here for
reference.



\begin{figure}[H]
    \begin{center}
    \adjustimage{max size={0.9\linewidth}{0.9\paperheight}}{analysis_2_files/analysis_2_31_0.png}
    \caption{Normalized Occurrences of Nontempered Fifths in Each Minor Fugue. }
    \end{center}
\end{figure}
    
    From observing these purity percentage graphs alongside the fifths
graph, we can see that there are clear correlations between the shapes
of the graphs, namely:

\begin{enumerate}
\def\labelenumi{\arabic{enumi}.}
\tightlist
\item
  There is a clear division between the Pythagorean and meantone keys;
  in the percentage of pure fifths graphs, Pythagorean keys contain an
  overwhelmingly larger percentage of pure fifths than the meantone
  keys. In the graph of the fifths, these Pythagorean keys on a whole
  show greater frequency values than the meantone keys.
\item
  The general shape of the distributions, and peak values of the two
  percentage of purity graphs and afore untempered fifths frequency
  graph lie at the same locations, with the key of d-sharp being the
  maximum peak for all graphs, and the second peak minor peak occurring
  at a minor. The similarity of shape amongst these three graphs, as
  well as their peaks coinciding at the same places demonstrates
  definitively that the frequency of fifths usage indeed scales directly
  with the percentage of purity of a given key.
\end{enumerate}

The following two graphs present the frequency information for fourths,
and then fifths and fourths combined. While the graphs containing the
fourths are not as dramatic as the graph of fifths in terms of the
separation between Pythagorean and meantone keys, nor is the slope as
smooth as the graph of the fifths, the aforementioned correlative traits
between key and interval frequency is still retained. A key observation
from the graph of combined fifths and fourths is that the two exemplar
Pythagorean keys, d-sharp and b-flat, the two keys that maximize the
purity of perfect intervals, are the two keys that contain the highest
frequency of fifths and fourths amongst all the twelve minor keys by a
considerable margin. Similar to the graph of fifths alone, the frequency
values fall off steady and incrementally from the epicenter of d-sharp
and b-flat, with g-sharp minor being the only exception, as it dips
below the values of c-sharp, f-sharp, and f minor.


    \begin{center}
    \adjustimage{max size={0.9\linewidth}{0.9\paperheight}}{analysis_2_files/analysis_2_33_0.png}
    \end{center}
    
    The intermediate conclusion from examining the frequency graphs for
horizontal perfect intervals in this section is that there is a strong
correlation between interval purity and frequency of usage within a key.
For the Pythagorean keys, which maximize the purity of the perfect
intervals, fifths and fourths are more frequently used in a melodic
setting than in the meantone keys. Furthermore, the rate of frequency
corresponds directly with purity in almost all the keys, with the
frequency of interval usage scaling consistently with the percentage of
that interval's purity in a particular key.

    \subsubsection{Vertical Perfect
Intervals}\label{vertical-perfect-intervals}

Before we depart from perfect intervals and proceed to examining the
frequency graphs of imperfect intervals, this section will examine the
usage of perfect intervals (namely the fifth) on the vertical domain.
The trends and differences between values in these following vertical
domain graphs are considerably less dramatic than the horizontal graphs,
nonetheless, the important features and conclusions garnered by the
horizontal graphs have been retained and corroborated. It is important
to note that the vertical domain distributions, although appearing more
uniform in values across different keys, are still not random when it
comes to key area and frequency correlations, and that the separation
between Pythagorean and meantone keys is still readily observable.

While this dissertation will not directly test the reasons behind why
these differences are far more dramatic on the horizontal domain, there
are two immediate plausibilities as to why. The first is
musical/compositional: it is entirely plausible that these effects are
simply not as strong on the vertical domain, due to the predominantly
linear nature of Baroque composition, which places the horizontal domain
more in the composer's direct control, and from which arises the
perception of the vertical domain as a companion structure. Furthermore,
the contrapuntal style and tonal framework, with the proper guidelines
on consonance and dissonance resolution, as well as the functional
nature of harmony and the intervallic building blocks of harmonic thirds
and fifths, place more restraints upon the vertical domain than the
horizontal. For this reason, the horizontal domain, within the specific
Baroque style and composition that Bach operated under, inherently
allows for more variation in interval choice and control-\/-\/-at least
in regards to thirds and fifths-\/-\/-and consequently can result in a
broader distribution than the more constrained vertical domain in this
specific instance. Continuing this vein of logic, it is also a
possibility that the weaker observable effects in the vertical domain
could be due to the already ubiquitous presence of functional fifths and
thirds, rendering a mere count of interval instances less powerful
without an additional function for parsing out salient motivic intervals
from the general functional ones.

As an example, the D-sharp minor fugue from Book I is heavily centered
upon the fifth as motivic material, not only on the horizontal domain as
the opening interval of the subject, but also on the vertical domain as
a way to bolster the serious and solemn mood of the piece. While both
the choice of the fifth as horizontal subject material, and the fifth as
important harmonic thematic material are readily observable to the ear,
there is a far greater reflection in the raw observable number of fifths
in the horizontal domain than the vertical domain in our frequency
graphs.

The second reason behind the weaker observable effect on the vertical
domain could very well be the result of the current computational
framework, and the vertical domain's greater susceptibility to the
framework's built in constraints and limitations. The specifics as to
why this is will be discussed as the analysis of this section unfolds.

Having established these plausible reasons behind the weaker fluctuation
of values in the vertical domain, I will proceed to present the data
below, where we shall observe the results corroborating and supporting
the conclusions drawn from the horizontal domain graphs through
containing similar correlative effects between frequency of interval and
key, albeit at the smaller scale.

    The following two graphs present the percentage of fifths, then both
fifths and fourths, used-\/-\/-out of the total intervals found in a
piece, the same procedure as the horizontal graphs above-\/-\/-for
instances of vertical intervals. In this graph, all occurring intervals
are accounted for, given equal weighting regardless of which vocal
combinations they belong to, and simple and complex intervals are not
distinguished apart from one another (e.g. Perfect 12ths are recorded as
Perfect 5ths).

Considering the data in such a raw format is not without its drawbacks;
immediately, it should be said that not being able to distinguish
between complex and simple instances of an interval, as well as not
having any way to weight for the configuration of vocal part separation
or account for thickness in texture limits the musical information drawn
from such a survey to only the most general of conclusions. However, it
is important to be reminded that at this point the purpose of this
intermediate section of analysis is still primarily to establish the
existence of overarching trends to be statistically significant without
the danger of subconsciously cherry picking data. Nonetheless, this
dissertation is, at its heart, a musical discourse, and after adequate
statistical significance is established in the first two chapters, the
last chapter of analysis will focus more intently upon exploring the
relationship between temperament and musical elements, and how these
forces of temperament directly impact and shape underlying musical
structures.

Returning to the vertical charts at hand though, the data presented in
this raw fashion offers the most unbiased characterization of the
vertical domain intervals, as it is susceptible to far fewer artifacts
and built-in, hidden constraints and variables, such as the baseline
vocal thickness of the fugue, dictated by the number of voices present.
Thus, while the differences between values may not be as drastic as
those seen on the horizontal domain-\/-\/-to which indeed, they very
much are not-\/-\/-any mirroring of results found in the horizontal
domain, or consistent correlation between frequency and key still should
be considered as important and indicative of a connection between
interval usage and key area. With these clarifications in place, we
shall present the following two graphs of fifths and fourths on the
vertical domain:


    \begin{center}
    \adjustimage{max size={0.9\linewidth}{0.9\paperheight}}{analysis_2_files/analysis_2_37_0.png}
    \end{center}
    
    Immediately, it is observable that in both graphs, the values for the
Pythagorean (blue) keys are all greater than the values for the meantone
(red) keys, with the exception of the g minor fugue, which contains a
greater amount of fifths than two of the Pythagorean keys (f-sharp and
g-sharp) when considering only fifths, and exceeds the values of two
different keys (f and d-sharp) when both fifths and fourths are
accounted for.

Proceeding through both the graphs separately and in order though,
starting with the graph of fifths only: while, expectedly, not as
dramatic in difference when compared to the horizontal graph, the
frequency values of fifths in the Pythagorean keys are still nonetheless
clearly higher than the meantone counterparts, with values ranging from
11.31-12.6 (mean = 11.95, SD = 0.41, median = 11.89), compared to the
meantone keys, which values range from 6.75-11.78 (mean = 9.5, SD =
1.86, median = 10.35). For the graph of both fifths and fourths, the we
observe the same result of the Pythagorean keys containing higher mean
and median values, with their range being 21.63-24.1 (mean = 22.89, SD =
.95, median = 22.96), and the meantone values ranging from 13.38-22.27
(mean = 19.05, SD = 3.56, median = 21.19).

From these values, it is evident that perfect intervals are indeed
employed more on the vertical domain as well in the Pythagorean keys
compared to meantone. In the third chapter of analysis in this thesis, I
will give more weight and attention to the horizontal domain, analyzing
in depth how horizontal fifths and fourths are deeply integrated as
fugal and motivic material in the Pythagorean fugues, giving musical
context and purpose behind their greater presence in these keys.
However, the sections in this chapter, such as this one, that do focus
upon the vertical domain remain important reminders that these
correlations between intervals and keys-\/-\/-and the musical effects
that arise as a result-\/-\/-are not limited to the horizontal domain
alone, but rather permeate throughout the fabric of the entire piece,
informing both vertical and horizontal dimensions, as separate entities,
as well as an interactive force.

    \subsubsection{Horizontal Imperfect Intervals (Sixths and
Thirds)}\label{horizontal-imperfect-intervals-sixths-and-thirds}

Having established the clear connection between the the Pythagorean keys
and increased usage of fifths in both the horizontal and vertical
domains, the natural next step, following the logic of the relationship
between historical tuning and compositional style, should be to look at
imperfect intervals, and investigate if a converse relationship holds to
be true. In other words, if we speculated that-\/-\/-and subsequently
demonstrated-\/-\/-there to be increased usage of fifths in the
Pythagorean keys, which is historically in line with Pythagorean tuning
and the fifth-centric harmony, should we expect to see the converse to
be true, that meantone keys contain a greater degree of sixths and
thirds?

The following graph shows the frequency of usage of horizontal minor
sixths for all the minor fugues:



    \begin{center}
    \adjustimage{max size={.7\linewidth}{0.9\paperheight}}{analysis_2_files/analysis_2_41_0.png}
    \end{center}
    
    If the correlations for horizontal fifths were clear between key group
and frequency, they are even more dramatic here in the instance of minor
sixths. With the exception of the a minor fugue, all the values of the
meantone keys are far above those of the Pythagorean keys, yielding a
clear conclusion that sixths-\/-\/-at least of the minor
variety-\/-\/-are far more prevalent in meantone keys than they are in
the Pythagorean ones. Examining the values, the Pythagorean keys have a
frequency range from 0-.24, with a mean of .27, standard deviation of
.2, and median of .23. In comparison, the meantone keys have a range of
.52-6.89, with a mean of 2.89, standard deviation of 1.96, and a median
of 2.33; both mean and median values which are dramatically higher than
the Pythagorean mean and median. If we examine the same type of chart,
but this time looking at Major sixths, and then all sixths
together-\/-\/-


    \begin{center}
    \adjustimage{max size={0.9\linewidth}{0.9\paperheight}}{analysis_2_files/analysis_2_43_0.png}
    \end{center}
    
    Again in both charts, it is readily observable that sixths are employed
far more frequently in the meantone keys than the Pythagorean ones. The
two keys that stand out as exceptions to the rule though are g-sharp and
a minor, both which contain values that are anomalous compared to the
other fugues from their key group. While these two anomalous values may
look suspect, it is still important to remember that these numerical
values are representations of raw frequencies, and do not take in
account any further musically motivated weighting functions, such as
taking in consideration whether or not the interval occurs at a strong
metric position, or if it is an integral portion of the subject or
countersubject material, or comprises a recurring episodic motif.

With this in mind, in the case of the g-sharp minor fugue, we can
observe that most of what is responsible for is abnormally high spike is
the relatively large quantity of Major sixths (the frequency of minor
sixths for g-sharp minor are still higher than the other Pythagorean
keys, but still remain within a relatively close range to the other
values). However, at a closer look, a large number of these Major sixths
occur in the subject material, but specifically the leap that serves to
join two portions of the subject together. Furthermore, the subject of
the g-sharp minor fugue subject is a bit of an anomalous subject to
begin with because it is comprised of two portions, the first which is
in the tonic key, and the second that suggests a tonal center in the
dominant key. For this reason, the connecting leap of a Major sixth that
bridges these two segments of the subject has relatively little motivic
or harmonic presence, and as a result, does not hold nearly the same
weight as the colorful tritone that is featured prior to the leap, or
the decisive fifth that ends the subject statement.


    \begin{center}
    \adjustimage{max size={0.9\linewidth}{0.9\paperheight}}{analysis_2_files/analysis_2_45_0.png}
    \end{center}
    


    \begin{center}
    \adjustimage{max size={.6\linewidth}{0.9\paperheight}}{analysis_2_files/analysis_2_47_0.png}
    \end{center}
    
    As demonstrated in this momentary, closer look into the composition of
the g-sharp minor fugue, we can see how deeper score analysis for
musical context and function is still essential in being able to provide
purpose and meaning to these numbers generated by computational methods.
It is an apt demonstration-\/-\/-indeed, one of many-\/-\/-of the
symbiotic relationship between theoretical and computational frameworks,
in which these distributions are useful and important in determining
overall statistical significance, but are dependent upon greater
theoretical analysis to not only verify and explain their results, but
more importantly to provide the greater musical meaning and purpose
behind the numbers.

Ultimately though, in the case of sixths, the correlations between key
group and interval frequency are clear, and when looking at the minor
sixths in isolation, and the major and minor sixths combined, these two
anomalies (g-sharp minor and a minor) still are not drastically
outstanding; the peak value of g\# minor still being well below 5 out of
6 of the meantone keys, and vice versa for a minor being greater than 5
out of 6 of the Pythagorean keys. Still, these-\/-\/-along with
other-\/-\/-anomalous features are worth further investigation and
revisitation, either through fine tuning of the computation framework,
or deeper investigation via theoretical analysis, as demonstrated above.

Before moving on the vertical domain, let us examine the graph for
horizontal thirds in the following three charts-\/-\/-


    \begin{center}
    \adjustimage{max size={0.9\linewidth}{0.9\paperheight}}{analysis_2_files/analysis_2_49_0.png}
    \end{center}
    


    \begin{center}
    \adjustimage{max size={.6\linewidth}{0.9\paperheight}}{analysis_2_files/analysis_2_51_0.png}
    \end{center}
    
    Although the differences between the key groups are not as strong as
observed with the sixths, it is still evident that, in both cases for
major and minor thirds, the meantone keys utilize these intervals
clearly and consistently more frequently than the Pythagorean keys. For
both the types of thirds combined, as recorded in the third graph above,
all of the values of the Pythagorean keys lie below those of the
meantone keys, with the range of the Pythagorean keys spanning from
3.08-6.2 (mean = 4.57, SD = 1.11, median = 5.2), as opposed to the
meantone range of 6.45 - 21.14 (mean = 12.19, SD = 6.15, median =
9.19)-\/-\/-all in all a sizable difference.

In regards to the greater differences in sixths between key groups than
thirds, it is very plausible that this is largely due to the third being
an interval that traditionally serves a functional role on the
horizontal domain, as most melodic lines are largely comprised of
stepwise (seconds) and skip (thirds) motion. In contrast, the sixth
(like the fifth) is far less a common interval on the horizontal domain,
and when employed, has a greater chance of having a motivically
significant presence outside of mere function. Indeed, when referring to
the numerical values, the baseline/average value for thirds is much
higher than the average value for sixths, corroborating this functional
feature of composition.

\subsubsection{Vertical Imperfect Intervals (Sixths and
Thirds)}\label{vertical-imperfect-intervals-sixths-and-thirds}

We will complete this section of non-tempered intervals by looking at
the vertical instances of thirds and sixths. The effects in this
category are by far the weakest out of all the other ones, perhaps to
the point of being arguably inconclusive for some of the graphs.
However, the fact that the vertical imperfect intervals would yield the
weakest results out of all the domains analyzed may not be surprising
when considering that the general harmonic building blocks for much of
tonal tradition-\/-\/-and certainly very much so in Bach's
era-\/-\/-consist of imperfect intervals (thirds and sixths). This
tertiary nature of harmony results in a large number of obligatory
thirds and sixths in each piece, making it difficult for a survey of raw
intervals, without the fine tuning and constraint of any additional
musical parameters, to characterize with the same amount of accuracy as
the other intervals composer intent beyond functionality.

Yet, with all this said, the composite results from this subsection are
still consistent with the trends from the previous subsections, and are
far from being able to be dismissed as inconclusive on a whole. While
the differences between key groups for the vertical thirds and sixths
are far more difficult to be immediately visually discerned, the overall
mean and median values still show that, even amidst the noise introduced
by functionality, the meantone keys still possess a greater propensity
towards imperfect intervals than the Pythagorean keys. The following are
all the graphs of imperfect intervals-\/-\/-thirds and sixths, presented
respectively, then collectively together-\/-\/-for the vertical domain:


    \begin{center}
    \adjustimage{max size={0.9\linewidth}{0.9\paperheight}}{analysis_2_files/analysis_2_53_0.png}
    \end{center}
    

    \begin{center}
    \adjustimage{max size={0.9\linewidth}{0.9\paperheight}}{analysis_2_files/analysis_2_54_0.png}
    \end{center}
    


    \begin{center}
    \adjustimage{max size={.5\linewidth}{0.9\paperheight}}{analysis_2_files/analysis_2_56_0.png}
    \end{center}
    
    The means and standard deviations for the above charts are as listed:

Minor sixths: Pythagorean: Mean = 7.54, SD = .38 Meantone: Mean = 8.75,
SD = 1.72

All sixths: Pythagorean: Mean = 18.34, SD = 1.63 Meantone: Mean = 19.82,
SD = 2.65

Minor thirds: \emph{Pythagorean: Mean = 15.63, SD = 1.40}
\emph{Meantone: Mean = 15.34, SD = 2.17}

All thirds: Pythagorean: Mean = 26.54, SD = 1.98 Meantone: Mean = 26.75,
SD = 1.46

All imperfect intervals: Pythagorean: Mean = 44.88, SD = 2.12 Meantone:
Mean = 46.57, SD = 2.17

Although the mean values of both groups are considerably closer than
those contained in the other subsections, all of the mean values for the
meantone group-\/-\/-with the exception of the minor thirds, marked in
italics above-\/-\/-lie above the mean values for the Pythagorean group.
Furthermore, the combination of all imperfect intervals at the end
yields a higher mean value for the meantone keys than the Pythagorean
ones, which is also reflected in the ranges for both the groups
(meantone keys having an overall higher range from 44.09-50.77, compared
against the Pythagorean group's range of 41.96-48.23), as well as the
higher median value in the meantone group (meantone median = 46.5,
Pythagorean median = 45.46).

These results, while less dramatic and telling perhaps than those from
the other subsections, are still consistent within the various graphs of
vertical imperfect intervals, as well consistent with the conclusion
drawn from graphs of the horizontal domain. Ultimately, this allows us
to confidently conclude that the greater usage of thirds and sixths are
reflected in the vertical domain as well as the horizontal, and
musically, we should expect the two key groups to differ in the
harmonies that they employ.

\subsubsection{Conclusions for Non-Tempered
Intervals}\label{conclusions-for-non-tempered-intervals}

From the four studies conducted in this section of horizontal and
vertical non-tempered intervals, we are able to conclude that there are
clear correlations between key and interval choice that follow
compositional traditions dictated by contemporary historical tunings of
the time. Keys that belong to the Pythagorean group, that is, keys that,
in the well-tempered system, resemble the Pythagorean tuning tradition
of retaining pure perfect intervals, with resulting wider major thirds
and narrow minor thirds, favor the usage of fifths and fourths both
melodically and harmonically significantly more than their meantone
counterparts. On the other hand, meantone keys, which resemble the
meantone tradition of close to pure thirds and sixths, at the expense of
tempered fifths and fourths, contain within their compositions a far
greater number of these imperfect intervals compared to the Pythagorean
keys.

All in all, this yields a strong conclusion that temperament had a
direct influence on Bach's choice of interval when constructing a piece.
While the studies in this section are still too general to be able to
accurately conclude significant additional musical information from
these correlations, such as how these intervals are utilized in a
greater fugal and compositional context, the bottom line still remains
that the correlation very much exists. It will be the domain of the
final analysis chapter to connect these results with other underlying
musical elements and structures, but this more general survey here is
crucial in first determining impartiality, that these correlations
really are present in the music, and not a result of cherry picking due
to inherent, subconscious bias.

    \subsection{Tempered Intervals}\label{tempered-intervals}

The last segment of this section will explore the relationship between
key and tempered versions of perfect and imperfect intervals,
essentially the next logical question to be asked after establishing a
connection between temperament and untempered intervals. While the
previous section determined that fifths and fourths were used more
frequently in Pythagorean keys, and sixths and thirds more in meantone
keys, this did not say anything about the tempering of these intervals
employed. We can surmise that Bach used fifths and fourths more in
Pythagorean keys to utilize their purity to create more sonority in the
sound, but is this principle further bolstered and reflected in Bach's
choice of which specific fifths he used-\/-\/-in other words, did Bach
not only tend to use more fifths in keys in which fifths are purer, but
specifically lean towards choosing the ones that were pure? Affirming a
correlation between key and tempered versions of intervals would show an
interaction between temperament and composition on an even deeper level,
showing that Bach was not only sensitive to choosing certain intervals
for certain keys, but further sensitive to selecting specific versions
of those intervals within a key based on temperament to maximize the
effect.

The key question that this subsection seeks to address is whether or not
a composition's key would cause Bach to use a specific variety of
tempered (or non-tempered/pure) interval more frequently than other
versions of that interval, had the key been different. While this notion
is quite simple to understand, in order to test this we cannot just
merely look at the frequency of a specific tempered version of an
interval in isolation within a piece to formulate a conclusion. For
example, a survey of looking at the tempering of all the fifths in the
d-sharp minor fugue would yield that roughly 98.5\% of these fifths are
pure (as opposed to narrow); this is objectively a large amount, but
before we can say that Bach is deliberately favoring the usage of pure
fifths over narrow, we must remember that d-sharp minor is one of the
most extreme of Pythagorean minor keys, a key that, by construct,
contains an overwhelmingly large proportion of pure fifths. Therefore,
to determine whether or not the value of pure fifths in the d-sharp
minor fugue is really due to compositional elements, rather than just
the makeup of the key itself, we need to first determine what the
expected value for pure fifths in that particular key is, and make sure
that our value exceeds that. Only then, can a claim that a connection
between key and compositional strategy on the level of selecting
specific tempered versions of intervals be valid.

For the test to determine the expected value of a certain tempered
variety of an interval for a given key, I will again employ the function
of transposition, specifically the method of taking the average value of
all fugues transposed to the particular key (for more information on
this method, please refer to the KDE section in the previous chapter).
This averaged value, derived from transpositions of 12 fugues each
written originally in different keys, is our most accurate
representation of the expected value of the frequency that a specific
type of interval should be employed for a particular key, unbiased by
possible differences in compositional approach informed by key. Just
like in previous sections, all fugues are normalized before they are
transposed and summed to ensure that duration artifacts are avoided.
Finally, it is important to mention that all percentage values in this
subsection measure percentages of one specific interval (e.g., 98.5\% of
pure fifths in d-sharp minor is the percentage of pure fifths out of all
fifths in that piece, not out of all intervals). Because these
percentages are taken from only the interval itself isolated from the
rest of the population, as long as there are a reasonable number of that
specific interval used in the piece, we have no reason to believe that
the actual frequency of the interval itself within the larger body of
the piece would exert any significant effect on the frequency of a
specific tempered version of that interval.

\subsubsection{Horizontal and Vertical Pure
Fifths}\label{horizontal-and-vertical-pure-fifths}

We will open this subsection with looking at frequencies of horizontal
pure fifth usage across all 12 minor fugues (this graph was already seen
in the previous section). The following two graphs are a side by side
comparison of the percentages of pure fifths (out of the composite
fifths), with the actual fugues on the left graph, and the expected
values for pure fifths given key on the right. Below is a line graph
representation of the same data that superimposes the two graphs onto
the same axis for better visual comparison. The values of the actual
fugues are marked in black; the values for the averaged transposed
fugues are marked in grey.


    \begin{center}
    \adjustimage{max size={0.9\linewidth}{0.9\paperheight}}{analysis_2_files/analysis_2_59_0.png}
    \end{center}
    


    \begin{center}
    \adjustimage{max size={.7\linewidth}{0.9\paperheight}}{analysis_2_files/analysis_2_61_0.png}
    \end{center}
    
    Looking at the line graph that superimposes the data from the actual
fugues with their corresponding estimated values (averaged transposed
fugues), we can observe that for the Pythagorean keys on the left (f to
f-sharp), the black line and markers lie mainly above the grey, with the
exception of g-sharp minor, indicating that for 5 out of the 6
Pythagorean keys, the percentage of pure fifths contained within the
fugue lies above the expected value of pure fifths for that particular
key. However, when we proceed to the meantone keys at the right end of
the graph, we see that this trend is reversed; the grey line and
markers-\/-\/-besides the boundary key of b minor-\/-\/-now assume the
higher position in the graph, exceeding the values of the black markers
for all of the meantone keys. This indicates that not only do
Pythagorean keys utilize pure intervals at a rate higher than expected,
the opposite holds true for the meantone keys, which employ tempered
intervals more often than expected for their respective keys.

Let us look to the same graphs done for the vertical domain:


    \begin{center}
    \adjustimage{max size={0.9\linewidth}{0.9\paperheight}}{analysis_2_files/analysis_2_63_0.png}
    \end{center}
    


    \begin{center}
    \adjustimage{max size={.7\linewidth}{0.9\paperheight}}{analysis_2_files/analysis_2_65_0.png}
    \end{center}
    
    Although the values in the vertical domain between the two graph are
closer than the horizontal domain (much like the previous section
involving untempered intervals), the same trend is repeated here: the
actual values (black line and marker) lie above the averaged values for
all the Pythagorean keys except for the boundary key of f-minor, with
the reverse holding mostly true for the meantone keys, except for the
key of g minor, and the boundary key of b minor.

These results suggest that, along with the predilection towards using
certain intervals depending on key type-\/-\/-in this case, perfect
intervals for Pythagorean keys that scales with expected
purity-\/-\/-this predilection towards purity is further extended to the
selection of pure intervals over tempered ones in these keys, and vice
versa in the meantone keys.

In the case for perfect intervals, the conclusions are quite clear, and
the patterns systematic, and are as follows:

\begin{enumerate}
\def\labelenumi{\arabic{enumi}.}
\tightlist
\item
  The usage of pure intervals are maximized in Pythagorean keys
\item
  The purer the key in terms of these intervals, the more the more
  frequently they are utilized
\item
  This positive correlation between key and interval frequency exists
  both in terms of the tempered and untempered variety of the intervals.
  In other words, for keys in which more perfect intervals are pure, not
  only does the frequency of usage of the intervals themselves increase,
  but also the frequency of the pure versions of those intervals.
\end{enumerate}

    \subsubsection{Vertical Pythagorean Minor
Thirds}\label{vertical-pythagorean-minor-thirds}

The last portion of this section involving tempered intervals will
examine intervals of the imperfect class; specifically, the Pythagorean
minor third. The reason behind choosing this specific tempered interval
to focus upon lies in the interval's idiosyncrasy and non-purity. To
address first the former: the narrow Pythagorean minor third (22 cents
narrow from just) was essentially a sound from a bygone era, as the
meantone era that directly preceded well-temperament focused on
retaining the purity of imperfect intervals. Furthermore, its extreme
tempered nature gives it a very different sound from the pure third, and
its own uncommon, unique color. There was a reason why this interval was
assigned to the distant keys of d-sharp minor and b-flat, while the
purer thirds were retained for the more common diatonic keys.
Conservatively, the remote placement of these extreme intervals can be
indicative of a preference towards purity, as a relegation to uncommon
keys would mean that these tempered intervals were less called upon. On
the other hand, if one were to look at composition as an innovative
pursuit, and different intervals as colors and their tempered varieties
as shades, extremely tempered intervals such as the Pythagorean third
would be an opportunity for creative force as well as innovation,
especially in a case dealing with territory that was essentially
unexplored.

In the previous subsection dealing with fifths, we were able to conclude
that Bach uses more pure fifths in the Pythagorean keys, keys that are
characterized by the purity of their perfect intervals and Pythagorean
minor thirds. In this section, we would like to see whether or not Bach
does the same with his treatment of Pythagorean thirds, and uses them to
a greater degree in these particular keys, especially the exemplar keys
of b-flat and d-sharp. Because the Pythagorean third is not a pure
interval, we have stronger reason to believe that any increase of usage
is a result of a deliberate compositional decision to amplify the
built-in effects and idiosyncrasies of the key, rather than just a mere
preference towards purity.

Lastly, I will be omitting the horizontal domain from the survey for
this section, as the sparse presence of horizontal thirds in the
Pythagorean keys, coupled with the dramatic difference between their
frequency in the two key groups will likely introduce skew to the data.
Because these thirds have a more balanced presence for vertical
intervals however, this domain should not be subjected to these possible
hidden biases.

Below are the graphs for vertical Pythagorean minor thirds, organized in
the same way that the graphs for the pure fifths were previously. The
bar graphs and first line graph represent compound intervals as their
simple counterparts; the final line graph only looks at simple
intervals, omitting compound ones.


    \begin{center}
    \adjustimage{max size={0.9\linewidth}{0.9\paperheight}}{analysis_2_files/analysis_2_68_0.png}
    \end{center}
    


    \begin{center}
    \adjustimage{max size={.7\linewidth}{0.9\paperheight}}{analysis_2_files/analysis_2_70_0.png}
    \end{center}
    


    \begin{center}
    \adjustimage{max size={.7\linewidth}{0.9\paperheight}}{analysis_2_files/analysis_2_72_0.png}
    \end{center}
    
    Although the effects from the graphs of the Pythagorean thirds may not
be as strong as that of the fifths in the previous section, a reversed
trend can be observed for the thirds here in terms of purity, but one
that remains consistent to the characteristics of the system of
Pythagorean tuning. In this instance, it is the extreme, tempered
Pythagorean third that is more frequently employed beyond the expected
value for the Pythagorean keys, and less so in the meantone keys, as
observed by the placement of the black and grey markers in the line
graphs.

Furthermore, it can be observed that this difference is strongest in the
Pythagorean group keys, and much weaker when it comes to the meantone
keys. Additionally, within the Pythagorean keys, this difference is most
pronounced for the exemplar keys of b-flat and d-sharp minor, especially
when considering only simple intervals. This stronger effect at these
two particular keys continue to be in line with the tuning tradition, as
while all six of the Pythagorean group keys contain pure fifths, and
utilize the Pythagorean minor third at particular points in their scalar
makeup, it is these two keys whose scales most strongly resemble
medieval Pythagorean tuning, with the highest degree of purity in the
fifths, along with the crucial placement of pure fifths and Pythagorean
minor thirds at the tonic and dominant scale degrees (I will get further
into the relationship between placement of tempered intervals and scale
degrees, and the musical effects and implications that this exerts on
both key groups in the next and final section of this chapter). It would
therefore make the most sense that it would be these two keys in which
Bach has the most musical reason to lean more heavily upon the
Pythagorean minor third to amplify, and perhaps recall, certain
idiosyncratic and idiomatic effects of that particular tuning system,
namely the sonority of the imperfect intervals, and the depressive and
coloristic nature of the very narrow minor third. While this
dissertation precludes the preludes from much of the analysis, this is
most certainly so in the case of the preludes of b-flat and d-sharp
minor, as they are the only preludes from Book I to utilize a pervasive,
chordal texture as harmonic accompaniment that is motivically structured
around the repetitive knell of vertical thirds.

In some ways, this is an even more interesting result than the result
observed for the fifths, as results for the fifths follows a principle
for favoring pure intervals, and also has the frequencies for untempered
and tempered intervals scaling directly for pure fifths, and overall
number of perfect intervals. The relationship between the tempered and
untempered frequencies are a little more complex for the Pythagorean
thirds though, as they are firstly extreme tempered intervals, and the
frequency of the imperfect class intervals that thirds belong to is
generally associated with meantone class keys instead of Pythagorean.

To first explore the element of purity: while the Pythagorean third is
also the idiomatic interval for the Pythagorean keys, it is the version
of the third that is furthest from pure within the well-tempered tuning
system, and therefore its increase of usage in this case cannot be
explained by a principle of adhering to purer intervals, but instead
would be directly attributed to other traits of the tempered interval
itself. In terms of the latter point, if we refer back to the frequency
graphs for the untempered versions of imperfect intervals, we observed
that imperfect intervals on both horizontal and vertical domains are far
more frequently employed in meantone keys that Pythagorean for all
intervals except for one: vertical instances of minor thirds. Although
the differences between the two key groups for vertical imperfect
intervals was consistently slight, it was the Pythagorean class keys
that saw a more frequent usage of vertical minor thirds than the
meantone class keys, a reversal to the overwhelming overall trend of
positive correlation between meantone keys and frequent imperfect
intervals otherwise.

The uncharacteristic reversal of frequency of vertical thirds
corresponding with Pythagorean keys instead of meantone, coupled with an
increase of usage for the specific Pythagorean third in the Pythagorean
group, especially at b-flat and d-sharp minor, suggests there to be
deliberate employment of this idiosyncratic interval that served a
musical effect outside of mere retention of interval purity.

    \subsection{Section Conclusions}\label{section-conclusions}

\subsubsection{Subsection results:}\label{subsection-results}

From each of the subsections in this first segment of this chapter,
there have emerged systematic, consistent, and marked differences
between the compositional treatment of the Pythagorean and meantone key
groups, both in terms of interval predilection (both tempered and
untempered), as well as other musical elements. The following is a table
that summarizes the differences between meantone and Pythagorean keys as
concluded by the first part of this chapter:
\begin{table}[H]
\begin{singlespace}
\small
\centering
\begin{tabularx}{\textwidth}{|>{\RaggedRight}X|>{\RaggedRight}X|}
\hline
\textbf{Pythagorean} & \textbf{Meantone} \\
\hline
\vspace{-1em}
\begin{itemize}[leftmargin=*]
    \item Longer average note durations
    \item Thicker vocal texture (more number of voices)
    \item Preference towards perfect intervals (fifths, fourths)
    \item Preference towards pure fifths
    \item Preference towards Pythagorean thirds
\end{itemize}
&
\vspace{-1em}
\begin{itemize}[leftmargin=*]
    \item Shorter average note durations
    \item Thinner vocal texture (fewer number of voices)
    \item Preference towards imperfect intervals (sixths, thirds)
    \item Preference towards tempered fifths
    \item No preference towards Pythagorean thirds
\end{itemize}
\\

\hline
\end{tabularx}
\small
\end{singlespace}
\caption{Pythagorean/Meantone Comparison Chart}
\end{table}
    In summary, the first portion of this chapter demonstrates the two key
groups, Pythagorean and Meantone, to have the following musical and
acoustical qualities:

\textbf{Pythagorean keys} to be slower moving, prone to thicker, more
chordal textures, favoring the perfect class intervals that are in line
with the historical tuning in which their temperament structure
resembles (Pythagorean tuning). Within these perfect intervals, pure
intervals are favored, as well as Pythagorean minor thirds, serving to
further adhere to the Pythagorean tradition. Sonically and acoustically,
all of these above traits (dense usage of perfect intervals, selecting
for just/pure fifths and Pythagorean thirds, longer note durations,
thicker textures resulting in more voices present and closed position
chords (simple intervals)) result in maximizing acoustic resonance.

\textbf{Meantone keys} to be faster moving, prone to a lighter, more
sparse vocal texture, with predilection to the usage of the imperfect
class intervals (thirds and sixths) that were traditionally employed in
meantone temperament.

Framing things in terms of musical affect and key characteristics: the
Pythagorean minor keys are characterized by sonority, solemnity, and a
slow-moving/staid demeanor, while the meantone minor keys are generally
faster, upbeat, and more lively.

\subsubsection{Conclusions and Summary:}\label{conclusions-and-summary}

\begin{enumerate}
\def\labelenumi{\arabic{enumi}.}
\tightlist
\item
  Keys and key groups tend towards their historic intervals dictated by
  tuning traditions of the time: Pythagorean keys to perfect intervals,
  and meantone keys to imperfect intervals.
\item
  Correlations between key group and interval choice are present not
  only with overall (untempered) intervals, but also exist on a deeper
  level involving specific tempered versions of intervals.
\item
  Key group correlations extend to other musical elements, namely
  duration and vocal thickness (number of voices), which work together
  with both tempered and untempered interval types towards a clear and
  predictable character delineation between the key groups.
\end{enumerate}

Essentially, this first section of this chapter further demonstrates
that the built-in differences between the Pythagorean and Meantone key
groups that were explored in chapter 4 go beyond mere effects that are
embedded in the temperament, and extend to differences in compositional
approach to keys that we can pinpoint in specific, traceable, and
quantifiable ways. The KDE plots of the latter portion of the first
chapter established differences between the two key groups beyond mere
traits of the tuning system through showing that keys retained certain
aspects of their original key group even after transposition, but were
too general to offer more information as to what musical and
compositional elements were responsible for this effect. The first
portion of this chapter is the first level of the exploration into the
compositional elements responsible for key differences, and a prelude to
an increasingly more musically focused narrative on how these
compositional elements interact with temperament and key.

It is from this section that we can start to glean systematic
connections between compositional choices and the structure of
well-temperament, and how the phenomenon of key characteristics begins
to emerge from these various elements intersecting and working together.
The next subsection on scale degrees will further unpack how the
organization of well-temperament within scales gives direct rise to more
defined, fine-grained differences by way of scale construction not only
between the larger Pythagorean and meantone key groups, but within these
key groups themselves, and how these unique scale structures and
tempered interval placement further informs musical decisions on scalar
interval choices, deepening the perception of key characteristics.

    \section{Scale Degree Graphs}\label{scale-degree-graphs}

The first portion of this chapter saw a continuation of emergence of two
types of minor keys from the well-tempered system: Pythagorean and
meantone, which recalled and resembled their respective titular
historical tuning systems not only in terms of acoustical scale
construction based on the pure virtue of temperament, but additionally
mirrored the predilections of their parent systems in regards to
interval choice-\/-\/-tempered and untempered-\/-\/-within composition.
Furthermore, we were able to demonstrate that compositional differences
between the two key groups extends beyond interval choice, involving
other musical elements of note duration and vocal texture. Tracing the
trajectory of analysis that began with the broad-scope, general
temperament identity/fingerprint plots in the first section of chapter
4, to successively more localized and specific compositional aspects of
the musical terrain mapped out in the latter portion of this chapter, we
were able to begin to observe the emergence of key characteristics as
informed by temperament-\/-\/-in more musically dispassionate language,
a systematic connection between temperament and the treatment of musical
and compositional elements given key and key group.

In this final section to conclude the chapter, I will be focusing in
detail on temperament and its direct relationship with scale
construction, specifically how differing placements of tempered
intervals, directly-\/-\/-and uniquely-\/-\/-due to the unequal system
of well-tempering varies across key and key group, and how this may
inform deeper implied musical structures such as cadential and
modulation patterns within key both in quantifiable and theoretical
ways. This section in many ways serves as a logical conclusion of the
first two initial sections, as well as a bridge between these two
statistically heavy initial chapters, and a musically centric final
chapter to come, foreshadowing a final leg of analysis that focus
primarily upon how temperament shapes fugal elements and structure
through controlling the compositional building blocks on a motivic
level.

Before delving into analysis though, I want to first step back a bit and
address in detail just how temperament determines scale construction,
and how understanding this interaction is important to understanding how
temperament can have such a significant role in the formation of key
differences and characteristics.

\subsection{Temperament's Important Role in Scale
Construction}\label{temperaments-important-role-in-scale-construction}

Early on in chapter 2, we saw that it was the unequal sizes of fifths
that enabled the system of well-temperament to allow for the usage of
all chromatic keys, whereas the previous systems of Pythagorean and
meantone, that allowed for only pure fifths or thirds, resulted in the
allocation of the comma to one unusable wolf interval, rendering a
significant portion of the keys inaccessible. Because of the different
sizes of fifths (8 pure and 4 tempered to be exact), this also resulted
in greater variation of sizes of the other intervals as well (see
well-temperament interval chart in chapter 4), and a uneven chromatic
scale.

From considering these two points of well-temperament, we can observe
that the results from the new system were largely twofold: from an
acoustical, scientific standpoint, the introduction of unequal fifths
offered a practical solution to the "wolf" interval, but from a more
musical and compositional viewpoint, the unequal system increased the
forces of available types of intervals, expanding a composer's arsenal
of harmonic and melodic colors due to this intervallic variation.
Furthermore, because this organization of pure and tempered fifths was
systemically based on the circle of fifths, this resulted in a smooth
coloristic change across modulation, and divided up the spectrum of keys
between two large key groups, with diatonic keys resembling the patterns
of meantone tuning, and chromatic keys taking on the traits of
Pythagorean tuning.

This division of keys between Pythagorean and meantone, and its
relationship to temperament in regards to pure acoustical construct, as
well as deliberate compositional effort, has been at the heart of all
the analysis in this dissertation thus far. The first section in chapter
4 opened with a discussion on how the anatomy of the Pythagorean keys
differed intrinsically from meantone, with Pythagorean keys favoring
pure fifths at the expense of tempered thirds, and vice versa for
meantone keys. It is in this section though, that I will finally analyze
and unpack in detail how the construction of tempered intervals within
the scale contributes to this anatomical difference between Pythagorean
and meantone keys, and how this difference in scale construction plays
directly into the musical characteristics attached to these keys and key
groups. Finally, it is also in this section that I will introduce one
final interval into our analytical arsenal: the semitone, motivated by
its integral role in scale construction (especially in minor mode), as
well as its consistently stronger correlative effects between
temperament and composition that we have been observing on the
horizontal domain in the analysis thus far.

\subsection{Temperament and Scale Degree
Analysis}\label{temperament-and-scale-degree-analysis}

It is important to understand that the acoustical variation between keys
and key groups within the well-tempered system is directly resultant of
the unequal division of the chromatic scale, which leads to an uneven
chromatic scale, and a unique configuration of tempered intervals for
each key. However, what we have yet to unpack is the importance of the
exact placement and location of where these tempered interval lie within
the structure of the scale of each key (i.e., the scale degree that
these intervals fall on), for it is this element that largely controls
their frequency of usage and presence within a key. As a result, to have
full comprehension of how temperament impacts key, it is important to
develop a deeper understanding of how it impacts scale construction at
the level of scale degree.

Just as in the analysis of chapter 4, merely understanding the
acoustical construction of scales given well-temperament and key is only
the primary step; the important following step is to ascertain how scale
construction may have impacted how Bach treats different keys with
different compositional approaches.

The main points to explore in this section are, in
chronological-\/-\/-and logical-\/-\/-order:

\begin{enumerate}
\def\labelenumi{\arabic{enumi}.}
\tightlist
\item
  How do the two key groups, Pythagorean and meantone, differ from one
  another based on their well-tempered scale construction, i.e.: how do
  the distribution of tempered intervals differ between key groups?
\item
  What are the theoretical musical implications of this difference in
  tempered scale construction between Pythagorean and meantone keys (and
  on a finer-grained scale, individual keys within these categories),
  and how do these traits relate to the other musical elements and
  trends observed about the two key groups in earlier sections?
\item
  On a quantifiable level, can we observe correlations between frequency
  of interval usage given scale degree, similar to correlations that we
  have observed between interval frequency and general key in previous
  sections? If so, what are the musical implications of such
  correlations?
\end{enumerate}

Although all intervals have their function and importance in harmonic
and melodic construct, the intervals that we will be discussing in this
section are: fifths, thirds, and semitones for reasons relating to their
musical salience and their measurability, as well as our belief in their
heightened role in temperamental structures, as informed by previous
analysis sections (fifths and thirds because of their idiomatic role in
the Pythagorean and meantone keys, and semitones because of its strong
relationship to the horizontal domain). In terms of the importance of
scale degree analysis and establishing why focusing on these specific
intervals, and their respective placement within the scale, is musically
valuable, we must look towards what musical elements that these certain
intervals-\/-\/-at certain placements of the scale-\/-\/-govern. The
following chart details the types of intervals, scale degrees, and
musical elements that they influence:
\begin{table}[H]
\begin{singlespace}
\small
\centering
\begin{tabularx}{5.5in}{|>{\RaggedRight}X|>{\RaggedRight}X|>{\RaggedRight}X|}
\hline
\textbf{Interval} & \textbf{Scale Degrees} & \textbf{Musical Elements \newline Influenced} \\
\hline

\vspace{-1em}
\begin{itemize}[leftmargin=0cm]
    \item[] \textbf{Fifths: \newline~}
    \item[] \textbf{Minor thirds:}
    \item[] \textbf{Semitones: \newline~}
\end{itemize}
                  &

\vspace{-1em}
\begin{itemize}[leftmargin=*]
\item Tonic, Dominant, \newline Subdominant
\item Tonic
\item Leading tone, mediant, submediant
\end{itemize}
                  &
\vspace{-1em}
\begin{itemize}[leftmargin=*]
\item Tonal sonority, stability, \newline and modulation points
\item Modal color
\item Cadence types, linear power, chromaticism
\end{itemize}
\\

\hline
\end{tabularx}
\small
\end{singlespace}
\caption{Scale Degree Influences Chart}
\end{table}
    To elaborate on the information in the chart: the fifths are the main
intervals that govern acoustic sonority, with the pure versions of these
fifths providing more stability and stasis than the tempered version. In
terms of scale degrees, this would mean that keys in which the most
prevalent scale positions of tonic, dominant, and subdominant contain
pure fifths would not only possess the greatest overall acoustical
sonority (something that we can verify based on the analysis from
previous sections), but could also imply that the keys with pure fifths
at these tonically centric positions could have a greater pull towards
their tonic centers.

While the fifth and perfect intervals supports a key's sense of tonality
and level of sonority/resonance, it is the third that provides modal
information in a tonal system (i.e., an open fifth, while providing
information about a key's tonal center, communicates no information
about major or minor mode; it is the third that determines this); this
is especially true of the tonic third, the definitive third of its
assigned key. This property of the third and the imperfect class of
intervals in which it belongs to, as well as it being a more
acoustically complex interval than the fifth, essentially assigns the
power of tone color to the office of thirds and imperfect intervals. The
more pure the third, the less acoustical motion there is, yielding a
sound that is calmer and more settled-\/-\/-more "colorless" under
certain interpretations. As the third tempers-\/-\/-in the case of the
minor third, narrows-\/-\/-and gains more acoustical unrest (e.g.
beating/friction resulting from the desynchronization of frequency
envelopes, as well as heightens the perception to collapse inwards), the
more depressive, colorful, and directional they become, with the
narrowest Pythagorean third marking the edge of this spectrum.

Lastly, the semitone is a powerfully functional and affective interval
that exerts the most direct power on linear direction, voice leading,
and cadential resolution. Functionally speaking, because it is the
narrowest of the intervals in the 12-tone western chromatic scale, it is
the interval most responsible for voice leading and direction on the
horizontal domain. Furthermore, this directional effect can be
heightened through intonation, making this interval a prime candidate
for temperament directed analysis. Aside from temperament, a prime
example of intonation supporting voice leading is commonly seen in
non-discretized instruments (e.g. voice, or stringed instruments), in
which musically sensitive players will instinctively sharp the leading
tone at cadential points for stronger tonic resolution. Beyond function
though, the tempering of leading tone semitones (ti-do) can also have
more musical grounded implications, such as increasing the perception of
tension and resolution, as can non-leading tone semitones, which are
often employed as emotionally heightened motivic material (e.g. sigh
motifs, lament bass). Additionally, in minor mode, modal semitones at
the mediant and submediant scale positions (me-re, le-sol) are also
especially important, as these semitones, in conjunction with the minor
thirds at these positions, bolster the directionality of these thirds,
further accentuating modal color. All of these aforementioned effects
are altered, enhanced, or surpressed by the width of these semitones in
these tonal and modal scale positions, and it the goal of this section's
analysis to determine whether or not these theoretical elements, and
their proposed effects are reflected in a quantifiable and measurable
way within the compositions.

In summary, perfect intervals, and their placement at certain scale
degrees (namely the tonic and fifth) control acoustical sonority, as
well as tonal stability, minor thirds at the tonic position control
modal color, and semitones at the leading tone-to-tonic, mediant, and
submediant position control voice leading and linear direction, and
further facilitate color and overall chromaticism in minor mode. Because
we have not much explored the usage of the semitone in the prior
analysis sections of this dissertation, this section will more or less
focus upon the semitone in terms of quantitative analysis, with the
analysis on thirds and fifths more touched upon in a theoretical manner.
Specifically, the following lists the types of analyses engaged in this
section, with the questions that they respectively seek to answer:

\begin{enumerate}
\def\labelenumi{\arabic{enumi}.}
\tightlist
\item
  Theoretical discussion of fifths and thirds and semitones, addressing
  the question: How does the tempering of thirds, fifths, and semitones
  occurring at specific scale degrees work in conjunction to contribute
  to a key's sense of "minorness", and what are the possible musical
  implications of the differing patterns of tempered intervals at
  specific scale degrees between key groups? Furthermore, how does this
  connect with results found in previous chapters concerning the
  differences between compositional approach of Pythagorean and meantone
  class keys?
\item
  Quantitative analysis of semitones, addressing the question: How does
  the tempering of semitones at specific scale degrees affect Bach's
  frequency of usage of a particular interval, and how does this relate
  to the interval's potential function? For example, in keys that have
  extremely narrow semitones at the leading tone position, does Bach
  feature leading tone motion more in these compositions, as opposed to
  other compositions in keys in which the leading tone semitone is
  wider? In other words, can we observe measurable trends to support the
  idea that Bach is musically sensitive to adjusting compositional
  approach given the specific forces of semitones available to him in a
  given key, similar to the way performers of non-discretized
  instruments adjust intonation to bolster a desired musical effect?
\end{enumerate}

The composite goal of this final section of these first two chapters
will be a progression deeper into the territory of a narrative that is
more musical and theoretical. It will wrap up these first two chapters
of statistically heavier, broader surveys, and create a bridge into the
final chapter that focuses on localized, fugal analysis from a musical
lens. Before delving into the analysis, it is beneficial to note that,
although we are talking in terms of acoustical effects in this section
in regards to motion, there is an undeniably strong and causal
relationship between the strict acoustical realm and the more abstract
musical world, in which sounds, filtered through human perception,
become evocative of and analogous to emotions and human affections.
Somewhat tacitly understood too is the fact that the musical effects to
be explored in this section are complex perceptual constructs that arise
out of a cooperation of an intricate web of compositional devices, and
to assign a musical result exclusively to a singular compositional
component would be gross oversimplification. However, while it is to be
expected that as we shift the focus towards more abstract and complex
musical structures, the musical forces responsible will undoubtedly
increase and become intertwined, this should not preclude us to continue
to draw careful conclusions from quantitative trends that we can observe
between single elements and their underlying musical implications.

    \subsection{Fifths and Thirds}\label{fifths-and-thirds}

    In this first portion, we will be looking at how the construction of
fifths and thirds within the scale under the well-tempered system differ
between keys and key groups. Although we will still be presenting data
in the form of quantitative bar charts, this section is not as primarily
focused on achieving conclusions based on quantitative measurements as
it is illuminating the direct relationship between temperament and scale
structure, deepening the discussion on key characteristics from a more
theoretical point of view. The theoretical understanding on scale
structure and temperament established in this section will also set up
the main quantitative portion on semitones that will follow.

One of the main factors that set well-temperament apart as unique from
its predecessors of meantone and Pythagorean, as well as its successor
tuning system of equal temperament, is precisely its feature of
containing not just one type of fifth and third, but a variety of these
intervals, resulting in a unique organization of these intervals
corresponding with different scale degrees for different key. In this
light, this makes the connection between scale degree analysis and key
characteristics very relevant, as under the system of well-temperament
the construction of the scales are physically different given key.

The scale degrees that we will focus on most in dealing with fifths will
be the tonic, dominant, and subdominant (0, 5, and 7 in the chart), as
these are the scale degrees that these intervals most commonly occur on
in the tonal system. This is not only true theoretically in functional
tonal harmony, given the hierarchical relationship between and
respective importance of the tonic, dominant, and predominant harmonic
functions, but this can also be demonstrated empirically as well. The
following charts the corresponding scale degrees of all horizontally
occurring fifths in all 12 fugues, normalized by length of piece to
avoid duration biases and overrepresentation of keys with lengthier
compositions.



    \begin{center}
    \adjustimage{max size={.7\linewidth}{0.9\paperheight}}{analysis_2_files/analysis_2_83_0.png}
    \end{center}
    
    As theoretically expected, the tonic fifth is the most frequently used
fifth (38\% out of all fifths), followed by dominant, subdominant, and
then the natural seventh, and mediant, the latter both which are also
prominent scale degrees in minor mode. This chart will serve as a
comparative guideline to the individual key graphs that we will be
examining below, not only to bolster our theoretical knowledge of
expected frequency values, but also to illuminate any dramatic potential
anomalies or curiosities that should occur in certain keys.

Having established the importance of these scale degrees from a
theoretical tonal standpoint, as well as an empirical one drawn directly
from the compositions at hand, the following is a chart for the
distribution of horizontal fifths, direction invariant, for all
Pythagorean keys given scale degree:


    \begin{center}
    \adjustimage{max size={0.9\linewidth}{0.9\paperheight}}{analysis_2_files/analysis_2_85_0.png}
    \end{center}
    
    In this chart, the x-axis consists of scale degrees, indicated by
numerical value corresponding with the chromatic scale (0 being tonic, 5
being subdominant, 7 dominant). The y-axis charts duration in the units
of quarter beats; unchanged from previous sections. The value atop the
bar indicates percentage of usage of the particular fifth given scale
degree; all values for a given key summing to 1. Just as in the
temperament plots from chapter 4, the lighter color bars indicate pure
intervals (in this case, pure fifths), and the darker color bars
indicate the tempered intervals (in this case, narrow fifths).
Observations are as follows:

\begin{enumerate}
\def\labelenumi{\arabic{enumi}.}
\tightlist
\item
  All the keys in the Pythagorean group have pure tonic fifths.
\item
  All of the non-boundary keys have pure fifths at the dominant and
  subdominant scale degree positions.
\item
  The boundary keys each contain one of the two pure fifths; f minor
  containing a pure fifth at the subdominant degree, and f\# minor
  containing a pure fifth at the dominant scale degree. For f minor the
  percentage for the pure, subdominant fifth is higher than the tempered
  dominant fifth; the reverse is true for f-sharp minor. This pattern
  (of favoring a specific fifth over the other) is also retained in the
  close harmonic neighbors of these keys: for the keys of b-flat and
  d-sharp, that are harmonically closer to f minor, subdominant fifths
  are more frequently used than dominant fifths; for the keys of c-sharp
  and g-sharp, that are harmonically closer to f-sharp, dominant fifths
  are more frequently used than subdominant fifths.
\item
  Aside from the three prominent scale degrees of tonic, dominant, and
  subdominant, the majority of fifths at other scale degrees are also
  pure, with the exemplar keys of b-flat minor and d-sharp minor having
  the greatest ratio of scale degrees containing pure fifths.
\end{enumerate}

In summary, the most important traits of these Pythagorean keys are that
they all contain pure fifths at their tonic positions, most of them
contain pure fifths at the dominant and subdominant, and most of their
other scale degrees are populated by pure fifths as well. These
observations confirm the findings from chapter 4 and 5, in which we
demonstrated through KDE plots and graphs of fifths that Pythagorean
keys contain an overall high amount of purity in terms of perfect
intervals, but these graphs go one step further to illuminate in greater
detail why the exemplar keys of b-flat and d-sharp have a greater deal
of purity than the other Pythagorean keys. Additionally, as we progress
through this section, we will also begin to make finer grained
distinctions between keys of the same key class, and understand how
these detailed differences between scale layout lends individual keys to
a distinct set of musical attributes and compositional predilections,
set apart from other keys even within their own class.

The following are the same graphs, but this time for meantone keys:


    \begin{center}
    \adjustimage{max size={0.9\linewidth}{0.9\paperheight}}{analysis_2_files/analysis_2_87_0.png}
    \end{center}
    
    Again, the lighter bars in these graphs represent the pure fifths, and
dark bars represent tempered fifths (the color of blue and red do
reflect anything numerically distinct, besides the fact of
distinguishing the meantone group apart from the Pythagorean group).
Right away, we can see an expected reversal in the trend of
pure/tempered intervals, with the meantone group containing many more
tempered than pure fifths. It is an important reminder that the forces
of pure fifths in well-temperament outnumber tempered fifths 2 to 1,
with the system producing 8 pure fifths and 4 tempered ones. With this
in mind, it is expected that we still observe a notable presence of pure
fifths in the meantone keys, but as reflected in the graphs, these tend
to fall on lower priority intervals, with the key-defining tonic
position mostly occupied by a tempered fifth. Observations are as
follows:

\begin{enumerate}
\def\labelenumi{\arabic{enumi}.}
\tightlist
\item
  Four of the 6 meantone keys have a tempered fifth at the tonic
  position-\/-\/-b, d, g, and c.
\item
  Of these four keys, g minor should have the highest expected value of
  tempered fifths, having these fifths at the tonic, dominant, and
  subdominant position. This is corroborated by the graph charting
  percentages of pure fifths in the previous section (and also copied
  immediately below for referential convenience), in which g minor
  occupies the lowest value of expected pure fifths (for the graph of
  actual values, we see that c minor dips slightly below the value of g
  minor, becoming the actual minimum). At any rate, the three
  keys-\/-\/-d, g, and c-\/-\/-all occupy the lowest purity positions on
  the percentage graph, an expected occurrence given that the tempered
  fifths all occupy important tonal scale degrees for each key.
\item
  Even though the two meantone keys-\/-\/-g and c-\/-\/-that contain the
  highest expected and actual values of tempered fifths, in terms of
  number of scale degrees occupied by tempered fifths (3 out of 6 scale
  degrees), they are equivalent to the two keys that contain the lowest
  value of tempered fifths, e and a. This is directly due to the fact
  that, although e minor and a minor also have 3 out of 6 engaged scale
  degrees harboring narrow fifths, these scale degrees fall on weaker
  tonal positions, rather than the strong ones of tonic, dominant, and
  subdominant. Most importantly, e minor and a minor both have pure
  fifths at the tonic position, interestingly making them more similar
  to the Pythagorean keys in regards to perfect intervals.
\item
  The boundary key of b minor contains pure fifths at every scale degree
  except for the important tonic position, and this is significant
  enough that it greatly affects the purity composition of the key, as
  it is significantly lower than that of the Pythagorean keys in the
  graph of averages.
\end{enumerate}


    \begin{center}
    \adjustimage{max size={0.9\linewidth}{0.9\paperheight}}{analysis_2_files/analysis_2_89_0.png}
    \end{center}
    
    Now that we have examined both Pythagorean and meantone group keys, we
can observe in finer grained detail systematic differences between the
two groups from the level of scale construction, differences that
bolster conclusions from other chapters, but also shed additional
understanding on how these differences between key groups have come to
be. Although these conclusions at this point in the section may seem
facile, we are only looking at one interval-\/-\/-one facet-\/-\/-of a
far vaster and more complex picture of key and the intervals given
temperament that define it, one that will more reliably emerge as we add
upon the observations from the two other intervals we will be examining
in this section. For now though, the main observations garnered from
fifths are as follows:

\begin{enumerate}
\def\labelenumi{\arabic{enumi}.}
\tightlist
\item
  The Pythagorean keys differ from meantone keys by main virtue of the
  type of interval that falls upon the tonic scale degree, then dominant
  and subdominant respectively. For Pythagorean keys, tonic position is
  always occupied by a pure fifth, on the other hand for meantone, 4 out
  of 6 keys have tonics that consist of tempered fifths.
\item
  Looking at keys within key groups, we can begin to see why certain
  keys are more extreme versions of the groups' definitive traits, not
  only by virtue of the sheer forces of number of scale degrees that
  certain fifths fall on, but also by the specific scale degrees that
  contain these certain types of fifths.
\end{enumerate}

We can also observe that the peak of the "Pythagorean" model of keys,
when defined in terms of pureness of perfect intervals, lies at the
expected position of b-flat and d-sharp by virtue of the scale
construction of these two keys in regards to well-temperament, allowing
pure fifths to fall on all of the tonally important scale degrees of
tonic, dominant, subdominant, seventh, and mediant. However, the peak of
the "meantone" model keys, if defined under converse terms of minimizing
pureness of perfect intervals, actually lies at g minor and c minor. In
the proceeding section examining thirds, we will see how these minimum
and maximum points of fifths compare with those defined by thirds, and
how these peaks interact with one another.

The following is now the scale degree charts for minor thirds, for both
the Pythagorean and meantone keys:


    \begin{center}
    \adjustimage{max size={0.9\linewidth}{0.9\paperheight}}{analysis_2_files/analysis_2_91_0.png}
    \end{center}
    

    \begin{center}
    \adjustimage{max size={0.9\linewidth}{0.9\paperheight}}{analysis_2_files/analysis_2_92_0.png}
    \end{center}
    
    Immediately, it is evident that there are more forces in play when
dealing with thirds, that is, while there were only two types of fifths,
the well-tempered tuning system produces 4 types of thirds. Along the
spectrum from most tempered to pure, there are 4 thirds at 294 cents
(the narrowest, Pythagorean minor third), 5 thirds at 300 cents, two
thirds at 306 cents, and only one third at 312 cents, only 4 cents away
from a pure, 6:5 ratio fifth of 316 cents. While each type of third
undoubtedly has its own unique sound and set of acoustic properties, the
two thirds that are of most immediate interest to us are the thirds at
the opposite extremes of the spectrum, the Pythagorean minor third, and
the "pure" minor third, for the sake of comparison between Pythagorean
and meantone class keys. Observations for thirds as as follows:

\begin{enumerate}
\def\labelenumi{\arabic{enumi}.}
\tightlist
\item
  As expected, the narrow Pythagorean thirds occupy strong scale
  positions in the Pythagorean keys; the four keys that have this
  Pythagorean third as its tonic third are b-flat, d-sharp, f, and
  curiously c (the only key from the meantone group). Even though all
  these four keys have Pythagorean thirds as their tonic third, the way
  that the profiles of their fifths and thirds interact makes them very
  different: because both the peaks of the Pythagorean thirds and pure
  fifths coincide at they keys of b-flat and d-sharp, these keys are
  most traditionally "Pythagorean". However, although c minor contains
  the Pythagorean third at its tonic position, its fifth profile most
  strongly resembles a meantone minor key. Furthermore, there are fewer
  Pythagorean thirds involved in the construction of c minor than its
  Pythagorean key counterparts, as it only contains Pythagorean thirds
  at two strong tonal positions, the tonic and subdominant. As already
  noted in previous sections, and supported by later analysis in the
  next chapter dealing with fugal elements, we will observe b-flat and
  d-sharp to embody the traits attributed to Pythagorean minor keys,
  while c-minor shares more musical traits to meantone minor keys.
\item
  Because there is only one third at 312 cents (third that is closest to
  pure in Werckmesiter III), only one key-\/-\/-a minor-\/-\/-possesses
  this third as its tonic third. However, its surrounding keys of e
  minor and d minor, although not possessing this third as their tonic
  thirds, both have this third at tonally relevant scale positions; the
  dominant for d minor, and the subdominant for e minor.
\end{enumerate}

Now that we have looked at both fifths and thirds, certain traits that
further separate the Pythagorean keys from meantone keys become evident,
namely, the uniformity of intervals used, and the alignment of fifths
and thirds. In terms of the first point on uniformity, we are able to
observe that Pythagorean keys are much more uniform in the types of
intervals that they employ when it comes to fifths and thirds, due to
the pure fifth and Pythagorean third being more commonly occurring in
the system than tempered fifths and pure thirds. On the contrary,
meantone keys have a greater mixture of different types of intervals in
comparison, although retained in tonally significant positions are still
the tempered fifths and purer thirds that define the group.

In regards to alignment, Pythagorean keys have their fifths and thirds
align at the same place (b-flat minor and d-sharp minor), and
additionally are more uniform in the types of intervals they employ,
partially due to pure fifths and Pythagorean thirds being more common
than tempered fifths and pure thirds within the well-tempered system.
However, the close proximity of c minor, and the quick presence of
tempered fifths makes the drop off from Pythagorean to meantone quite
dramatic on that end of the circle of fifths; the progression the other
way into the direction of g-sharp minor and c-sharp minor is a great
deal smoother, and we will observe later that, in some ways, the keys of
c-sharp and f-sharp minor bear more similarities in terms of their
treatment of fifths than do f minor. Contrasting the alignment of fifths
and thirds in the Pythagorean group, the meantone keys have the peaks of
their tempered fifths and pure thirds misaligned, with the peak of
tempered fifths falling around g minor, and the peak of pure thirds
occurring at a minor. Interestingly, looking at these finer grained
details, we can start to see two subtypes of meantone keys emerging, the
meantone keys of a and e that adhere more to the purity of thirds, and
the meantone keys of c and g that bear more the trait of tempered
fifths.

Before we delve deeper into a discussion on key and key group given the
various alignments of these intervals, and the possible musical
implications of these intersections, we have one last important interval
to examine to complete the picture: the semitone.

    \subsection{Semitones}\label{semitones}

The semitone is an extremely important interval, especially in minor
mode, and a very special interval that has not been raised for
discussion in this dissertation much until now. It will from this point
on be playing an pivotal role for analysis, especially in the last
chapter dealing with fugal elements, by virtue of its motivic prominence
both within and outside of the subject matter, specifically in the
Pythagorean keys. Like the minor thirds, there are four different types
of semitones available in the well-tempered system at 90, 96, 102, and
108 cents. And again, like the minor thirds, the specific interval of
special interest to us is the narrowest of semitones at 90 cents, for
reasons being simultaneously historical, functional, and musical.

Historically, this 90 cent semitone, much like the Pythagorean third and
pure fifth, was a relic from the past, and out of circulation in the
most recent tuning predecessor of meantone temperament for the reason of
scale construction. Both the Pythagorean and meantone system consisted
of two different sizes of semitones, following their distinction as
diatonic or chromatic, with one type being wider than the other.
Diatonic semitones are defined as the semitone between two notes on the
diatonic scale, i.e., between two notes that have different note names,
and theoretically form the interval of a minor second; chromatic
semitones, or augmented unisons, are defined as the space between notes
that share the same name. While both types of semitones are used in
tonal music, diatonic semitones occur far more often, as they are built
into the functional diatonic construction of the scale (this can be
verified in the Expected Average Percentages for Semitones graph below,
as we can see considerably higher frequency values for the seven
diatonic semitones, and none of the frequency values for the five
chromatic semitones exceeding 1\%). Thus, while both the Pythagorean and
meantone system had a wide semitone and a narrow one, it is the diatonic
semitone that largely defines the size of the semitone used in the
system, and as it is, the two systems saw a reversal in the trend of
which size semitone was chosen as diatonic. In the older, Pythagorean
system, the narrower semitone at 90 cents (the Pythagorean limma) was
chosen as the diatonic interval, with the wider semitone at 114 cents
(the Pythagorean apotome) relegated to the chromatic position. The
meantone system reversed this, having the even wider semitone at 117
cents as the diatonic semitone, and a narrow chromatic semitone of 76
cents.

Complicated technical jargon aside, all this aims to say is that a) the
historical Pythagorean tuning system utilized narrower semitones as
opposed to the much wider semitones built into the meantone system, and
b) the 90 cent narrow semitone was a special interval that, like the
narrow Pythagorean third, was reintroduced into the intervallic arsenal
via the well-tempered tuning system, and, combined with the Pythagorean
third and pure fifth, hearkened directly back to a scalar construction
and sound from a more temporally distant era. Furthermore, under the
current system of well-temperament, this extremely narrow semitone is
relatively rare, as only two semitones out of the 12 are tempered to 90
cents.

Before we delve into the charts and analysis, we will be paying
attention to three particular semitones, the ones falling on the leading
tone (\emph{ti-do}), mediant (\emph{me-re}), and submediant
(\emph{le-sol}) position. These are the three semitones that are all
prominent in minor mode, the leading tone for obvious functional
cadential reasons, establishing tonality, and the latter two essentially
defining and establishing the modality as minor, carrying coloristic and
affective import. Indeed, the semitone in this minor context is the
interval right after the minor third that is most attached to the
emotional affect of the mode, as its placement within the scale defines
horizontal direction and the degree as to which a minor third, or minor
sixth, tends to resolve downwards. Furthermore, while minor thirds
create this depressiveness on a more vertical dimension, the
depressiveness for the semitone is most felt on the horizontal domain
(even when presented vertically, it still begs a horizontal resolution),
as it is a highly dissonant interval that seeks immediate linear
direction. In the next chapter, we will be looking specifically at how
these semitones are used motivically in fugue subjects, and how this
significantly impacts the general affect of a particular key and
composition.

I will present the data in this portion in the following fashion:

\begin{enumerate}
\def\labelenumi{\arabic{enumi}.}
\tightlist
\item
  Semitone scale degree graphs (both m2 and A1) for both meantone and
  Pythagorean keys, and briefly compare the two key groups in a
  theoretical manner.
\item
  Isolate the Pythagorean group for a more in depth, theoretical
  discussion on how specific placements of the narrow "Pythagorean"
  semitone may interact and shape their corresponding keys.
\item
  Quantitative analysis of the Pythagorean semitone (m2 only, as we are
  focusing more on function) to see if there are correlations between
  the tempering of semitones at certain scale degrees and frequency of
  usage to further bolster theoretical claims raised above.
\end{enumerate}


    \begin{center}
    \adjustimage{max size={0.9\linewidth}{0.9\paperheight}}{analysis_2_files/analysis_2_95_0.png}
    \end{center}
    

    \begin{center}
    \adjustimage{max size={0.9\linewidth}{0.9\paperheight}}{analysis_2_files/analysis_2_96_0.png}
    \end{center}
    


    \begin{center}
    \adjustimage{max size={.7\linewidth}{0.9\paperheight}}{analysis_2_files/analysis_2_98_0.png}
    \end{center}
    
    Adhering consistently to the same system of graphical representation as
previous sections, the darkness of the bars correspond to the degree of
tempering from pure, with the darkest of the bars representing the 90
cent Pythagorean limma, and the lightest bar representing the widest
semitone of 108 cents.

Immediately, we can observe that the semitones in the Pythagorean group
on a whole list towards the narrower end of the spectrum, while the
semitones for the meantone group tend towards the wider end, creating
yet another point of distinction between these two key groups. Moreover,
the usage of the very narrow Pythagorean limma lies almost exclusively
within Pythagorean group keys, with it being present in every
Pythagorean key, and also functioning in at least one of the three
important scale degrees-\/-\/-mediant, submediant, leading tone-\/-\/-in
every key except for g\# minor (more on this later). In contrast, it is
essentially absent in the meantone keys, not occurring at all in the
keys of e, a, d, and g, and only peripherally in the boundary keys of c
minor and b minor. B minor is the only meantone key that it makes
somewhat of an appearance in for the meantone group, but on more
auxiliary scale degrees; we will unpack this shortly in an upcoming
subsection.

For this section on semitones, our examination of the meantone group
largely stops here, not because it is not of inherent interest to us,
but for the sake that we would wish to look closer at the usage of the
Pythagorean limma in the next section, and specifically how it, and the
level of narrowness of the semitone potentially affects rate of usage at
specific scale degrees. The absence of the Pythagorean limma from the
meantone keys most directly precludes them from this discussion, but on
a more theoretical level, because of the different palettes of semitones
between the key groups, comparing trends across both key groups may be
selecting far too wide of a scope, a generality that can unwittingly
yield a comparison of disparate objects, running the danger of creating
unwanted noise, or artifacts in the resulting distributions. Indeed, we
will observe this difference even more so in the next chapter, as
semitones are combined with other musical elements and treated quite
differently in the Pythagorean and meantone keys.

Turning now our attention to the Pythagorean group in interest of closer
examination of the Pythagorean limma, we can see that all of the
Pythagorean group keys incorporate this extremely narrow semitone in a
significant way. Based on the scale degrees that the limma falls upon,
we can see a rough division of the Pythagorean keys into two categories
(in which not all keys will fall neatly into, or in some cases, not at
all): keys that contain the limma on one, or both modal scale degrees of
the mediant (\emph{me-re}) and submediant (\emph{le sol}), and keys that
contain the limma on the tonal scale degree of the leading tone
(\emph{ti-do}).

The modal group is comprised of the exemplar keys of b-flat, d-sharp,
and f minor, with b-flat minor possessing the limma on both the mediant
and submediant positions, d-sharp minor possessing the limma on the
mediant position, and f minor on the submediant position. In terms of
strength of the effect of the limma, b-flat would obviously be first, as
it contains this interval at both modally important scale positions, and
d-sharp second, having the limma at the more definitive mediant
position, with f minor being the weakest of the three keys, having the
limma at the submediant position. The tonal group is comprised of
c-sharp minor and f-sharp minor, with f-sharp minor further possessing a
second Pythagorean limma at dominant position of \emph{fi-sol}.

Given the role of the semitone as a primary force in voice leading,
linear motion, and tension resolution, which can furthermore be
accentuated by its degree of narrowness, we may expect to see more
attention being drawn to the modal Pythagorean limmas in the modal
group, and more attention drawn to the leading tone limmas in the tonal
group to utilize this directional and harmonic power. Practically
speaking from a musical point of view, this could be compositionally
manifested through choosing motifs or subjects that accentuate these
modal scale degrees for the keys of b-flat and d-sharp minor, and
subjects and motifs that draw more attention to tonic resolution for the
c-sharp and f-sharp fugues. Harmonically speaking, keys from the tonal
group may favor stronger and more frequent cadential motion (and in the
case of f-sharp, may highlight the dominant), while b-flat and d-sharp
may lend themselves to more modal harmonies.

As we will see in the coming chapter on fugal elements, as well as
unpack in greater detail, all of these elements are indeed reflected in
the compositional structures of these fugues. Harmonically, the d-sharp
minor fugue is the only fugue of WTC I that is written extensively in
the modal tradition, although both it and the b-flat minor fugue
incorporate modal harmonies. Motivically, the fugues of f minor and
b-flat minor both utilize semitones motifs that features descending
submediant-dominant motion (f minor in the form of its opening interval,
b-flat minor in the form of a highly unusual and stylized leap of a
ninth in the subject). The c-sharp minor fugue heavily utilizes the
leading tone semitone as motivic material for both its subjects, having
this semitone as the opening interval to the first subject, with the
final subject suggesting cadential motion that features the leading tone
in a prominent way. F-sharp minor's subject highlights the dominant
scale degree in a significant way, with a rising chromatic line up to
the fifth scale degree, ultimately approaching it by a semitone
resolution, implying the fifth as a local tonic, before falling back
down the scale and resolving on the original tonic of f-sharp.

However, this type of detailed fugal subject/motivic discussion and
analysis is the domain of the next chapter; immediately in this chapter,
we can also observe these effects in a measurable, quantitative way when
it comes more to function. Below is a graph charting percentages of
usage for semitones in the leading tone position (\emph{ti-do}) for each
of the Pythagorean class fugues.



    \begin{center}
    \adjustimage{max size={.6\linewidth}{0.9\paperheight}}{analysis_2_files/analysis_2_101_0.png}
    \end{center}
    
    As we can clearly observe, the frequency of usage of the leading tone
semitone increases smoothly and gradually as a function of its
narrowness, with the two keys that possess the Pythagorean limma,
c-sharp and f-sharp minor, containing the most instances of the leading
tone semitone (at roughly 20\% and 23\%). The leading tone semitone is a
tonally important semitone in that it facilitates tonic resolution,
aided by the dissonance and narrowness of this interval, a function
which can be highlighted by moving the leading tone closer to the tonic
(i.e., narrowing the semitone interval) to increase linear motion. In
translation, narrower leading tone semitones possess more functional
power, and in this case, there seems to be a strong, positive
correlation between leading tone power and the frequency in which Bach
employs leading tone motion across Pythagorean class fugues. The
following two charts also represent these leading tone semitones, but
this time looking at it weighted against the entire piece (first graph),
and the second which looks at only ascending leading tone semitones. In
both instances, the strong correlative results are similar to our
original graph.


    \begin{center}
    \adjustimage{max size={0.9\linewidth}{0.9\paperheight}}{analysis_2_files/analysis_2_103_0.png}
    \end{center}
    
    The next two graphs look at mediant (\emph{me-re}) semitones to see if
we can see any similar correlations between frequency of usage and
narrowness of semitone. The left graph looks at all mediant semitones
regardless of direction, and the graph on the right isolates only
descending mediant semitones, for reasons that will be explained
directly below:


    \begin{center}
    \adjustimage{max size={0.9\linewidth}{0.9\paperheight}}{analysis_2_files/analysis_2_105_0.png}
    \end{center}
    
    While there is less variation between sizes of semitones with mediant
semitones as opposed to the leading tone semitones, we can nonetheless
see that the highest value for the mediant semitone still falls on a key
containing the Pythagorean limma, specifically d-sharp minor. When
examining only descending mediant semitones this peak is even more
dramatic. Just like ascending leading tones are more likely to contain
functional information, as that traces the natural direction of
resolution (leading tones resolve upwards), the natural direction of
mediant semitones is downwards, hence we should expect to see an effect
stronger with this extra parameter.

The last two graphs are the submediant semitones (direction invariant),
followed by ascending secondary dominant "leading" tones
(\emph{fi-sol}). Correlative effects for the submediant semitones may
appear weaker than the previous two semitones analyzed, but the peak for
these semitones is still retained by a key containing the Pythagorean
limma on this scale degree (b-flat minor, in this case). Also, note that
f-sharp minor utilizes this submediant semitone considerably less
frequently than all the other Pythagorean keys, and is also the only
Pythagorean key to contain a wider semitone (102 cents) on this
interval. In fact, the 14.86 frequency percentage is an overall very low
number, considering that the submediant semitone is the second most
prominent semitone in minor mode on average; additionally, f-sharp minor
is the only Pythagorean key that uses the leading tone semitone (which
in its key contains the Pythagorean limma) more frequently than its
submediant semitone. This unusual reversal of the usage of these two
semitones in the key of f-sharp minor provides further evidence of
Bach's sensitivity to the chromatic forces within a key's temperamental
scale scheme, and a compositional sensitivity for selecting narrower
semitones for their voice leading purposes.

For the secondary dominant "leading" tone graph, we have included the
meantone boundary key, b minor, in the chart, as it is the only other
key besides f-sharp that possesses the Pythagorean limma at this
specific scale degree. While there are no readily observable correlative
effects between frequency and semitone width in the other types of
intervals (non-Pythagorean limma), the frequency values of the only keys
that possess the limma at this scale degree-\/-\/-f-sharp and
b-\/-\/-are considerably higher than the values in the other Pythagorean
keys, as well as the meantone keys (not shown in the graph).
Furthermore, these values are again uncommonly high in a general sense
for this interval, as on average, this semitone is expected to be the
6th most frequent semitone in minor mode, with a expected frequency
percentage of 6.8. Lastly, the f-sharp minor fugue is the only key in
Book I to utilize this secondary dominant leading tone as salient
subject material.



    \begin{center}
    \adjustimage{max size={.6\linewidth}{0.9\paperheight}}{analysis_2_files/analysis_2_108_0.png}
    \end{center}
    


    \begin{center}
    \adjustimage{max size={.6\linewidth}{0.9\paperheight}}{analysis_2_files/analysis_2_110_0.png}
    \end{center}
    
    On a whole, the quantitative analysis of different scale degree
semitones for the Pythagorean keys show correlations between narrowness
and frequency of usage, with the leading tone and mediant scale degrees
having the strongest correlative effects. These results suggest that the
tempering of semitones was indeed taken into account in compositional
construction, with Bach utilizing the forces of the narrower
semitones-\/-\/-especially the Pythagorean limma-\/-\/-much like a
vocalist or violinist would to accentuate and facilitate voice leading
and linear direction.

Further observations in regards to how the placement of the limma, as
well as overall semitone profile, shapes different individual keys
within the Pythagorean group in specific ways, creating a further degree
of key distinction are as follows:

\begin{itemize}
\tightlist
\item
  As we have seen earlier in this section involving fifths and thirds,
  b-flat minor is exemplary of the Pythagorean minor group, as it
  maximizes the purity of fifths, and possesses Pythagorean minor thirds
  at all tonally important scale degrees. The double Pythagorean limma
  (limma at both the mediant and submediant position; the only instance
  in all minor keys) in b-flat minor completes this picture, offering
  further explanation as to why b-flat minor is the exemplar Pythagorean
  minor key, both musically and idiomatically. Historically, the
  Pythagorean third, Pythagorean limma, and pure fifths are all
  hallmarks of the Pythagorean tuning system, and although b-flat minor
  is not a perfect reconstruction of what a minor key would have
  physically been in that tuning system, it is the closest
  reconstruction that exists within the well-tempered system. Musically,
  the presence of this limma at these modal scale degrees works in
  conjunction to the Pythagorean thirds at these corresponding locations
  to achieve increased linear direction for these thirds to compress
  inwards, further increasing the perception of the thirds' narrowness.
  This provides a very concrete acoustical explanation through
  temperament and scale construction to the extreme minor nature of
  b-flat minor, and the key's association with emotional/musical
  descriptors such as dramatic, extreme, depressive, plaintive.
\item
  G-sharp minor is the only Pythagorean minor key that does not contain
  the limma at prominent scale degrees of subtonic, mediant, or
  submediant locations; instead, it contains the limma at the major
  third position (\emph{mi-fa}), and major sixth (\emph{la-te}). This
  could be a partial explanation as to why g-sharp minor is a bit of an
  anomalous behaving key within the Pythagorean group in terms of
  interval choices that in some instances aligned it closer to the
  meantone group keys.
\item
  Heightened chromaticism is characteristic of many of the Pythagorean
  minor mode fugues, but the distribution of types of semitones could be
  responsible for different approaches to treatment of chromaticism
  within these fugues given key. The highly chromatic minor fugues of
  WTC I, evaluated with focus on the level of chromaticism found in the
  subject and recurring motifs, are b minor, f minor, c\# minor, and f\#
  minor. While these four keys share the trait of having highly
  chromatic subjects, the treatment of chromaticism in the c-sharp and
  f-sharp minor fugues are different than the chromaticism in f and b
  minor. The chromaticism of f minor and c-sharp minor are notably
  smoother, with chromatic lines that favor stepwise motion, limiting
  leaps to smaller intervals. However, the chromaticism of b minor and f
  minor is more erratic; b minor's subject utilizes all twelve chromatic
  tones, featuring a series of highly dissonant, erratic leaps and
  chromatic sigh motifs that do not expressly focus on traditional
  dissonance resolutions, f minor's subject features a tonally awkward
  chromatic perfect fourth leap between \emph{fi} and \emph{ti}, and a
  counter subject with an equally dissonant and awkward leap of a minor
  ninth down from \emph{sol} to \emph{fi}. On the other hand, the
  chromaticism of c-sharp and f-sharp are smoother; f-sharp's sinuous
  subject and counter subject, whilst highly chromatic, feature long and
  smooth lines instead of jagged leaps, c-sharp minor's first subject is
  succinct, compact, and slow moving, with a second subject that
  features flowing sixteenth notes, and an organ textural style that
  utilizes descending tetrachords both as recurring motifs as well as
  long pedal points. These different chromatic styles are mirrored in
  their semitone profiles: b minor and f minor are boundary keys, and
  are the only minor keys that substantially contain semitones from both
  ends of the width spectrum. Musically speaking, the presence of both
  extreme types of semitones can serve to mirror and accentuate the
  jagged chromaticism to create even more contrast and variety in the
  line. Conversely, c-sharp and f-sharp minor are the two most uniform
  keys when it comes to retaining narrow semitones on average (while the
  narrowest Pythagorean limma has more of a presence in b-flat and
  d-sharp, they also have more instances of the wider semitone at 102
  cents). This uniformity of narrowness, in which semitones are largely
  either 90 or 96 cents, complements the smoother chromatic style,
  accentuating the uninterrupted linear motion featured in these fugues
  (more to come of this subject in the next chapter).
\end{itemize}

    \subsection{Section Conclusions}\label{section-conclusions}

The conclusions gathered from this chapter not only continue to support
the conclusions drawn by the previous chapters and sections on the
division between the meantone and Pythagorean key groups and sections,
but also engenders a more detailed understanding of temperament's direct
control over scale construction, and a more nuanced distinction between
key groups, as well as between keys within each group. By focusing on
the intervals of the fifth, third, and semitone, and how specific
tempered versions on these intervals interact with important tonal scale
degrees, we witness an emergence of an even clearer sense of key
distinctions and characteristics that is strongly grounded in
temperament as well as historical musical tradition.

The following is a summary of the conclusions formed about the key
groups and subgroups from this section:

\begin{enumerate}
\def\labelenumi{\arabic{enumi}.}
\tightlist
\item
  Meantone group: Emphasis on tempered fifths and purer thirds on
  tonally significant scale degrees, with lack of narrow thirds, and
  almost no presence of the narrow Pythagorean limma. Musically, these
  purer thirds and mellower semitones give meantone minor a rounder
  sound more akin to the Renaissance interpretation of minor mode as a
  "softer" mode, and generally making these minor keys less extreme in
  their sound. Additionally, the tempered fifths on the vertical domain
  create less of a sense of stasis; this, combined with the wider
  semitones that allows for more even, resolute stepwise motion, lends
  meantone keys to faster, more upbeat tempos. This also influences the
  relationship with their relative major modes, allowing toggling
  between major and minor modes to be a more seamless process, as
  opposed to the usage of major modes as a tool of contrast.

  \begin{enumerate}
  \def\labelenumii{\arabic{enumii}.}
  \tightlist
  \item
    a and e minor - Meantone subgroup with focus on pure thirds: The
    peak of tempered fifths and pure thirds do not align in meantone
    keys, but are instead staggered. These keys contain pure thirds on
    their tonic scale degrees, as well as pure fifths on tonally
    important scale degrees.
  \item
    g and c minor - Meantone subgroup with focus on tempered fifths:
    These keys contain tempered fifths on tonally important scale
    degrees, and narrower thirds on their tonic scale degrees.
  \end{enumerate}
\item
  Pythagorean group: Emphasis on pure fifths and narrow thirds, with the
  presence of the Pythagorean minor third and Pythagorean limma on
  important scale degrees hearkening back to the older tradition of
  Pythagorean tuning. Because of the pure fifths, Pythagorean keys have
  more sonority and gravity, which combined with the presence of the
  Pythagorean minor third and narrow limma gives them much more of a
  dolorous and compressed character as opposed to the meantone group.
  Pythagorean minor keys are more extreme, and in a modern sense, more
  "minor" sounding, and lend themselves to more "minor" characteristics,
  such as affective chromaticism and slower tempos.

  \begin{enumerate}
  \def\labelenumii{\arabic{enumii}.}
  \tightlist
  \item
    b-flat and d-sharp minor - Pythagorean modal subgroup (exemplar):
    This group is characterized by the peak frequency of pure fifths, as
    well as Pythagorean minor thirds on most tonally important scale
    degrees (most importantly, the tonic scale degree), and Pythagorean
    limmas on modally important scale degrees (me and le). This is the
    most extreme "minor" subgroup.
  \item
    c-sharp and f-sharp minor - Pythagorean tonal subgroup: This group
    possesses Pythagorean minor thirds at lesser tonally central
    positions, and the Pythagorean limma at the leading tone position.
    These keys are the most uniformly "minor" keys, for while they do
    not possess the extreme minor intervals at important positions as
    does the modal subgroup, the collections of intervals contained in
    these keys are more consistently and uniformly narrow (in regards to
    both minor third and semitones), even more so than the modal
    subgroup. In regards to the semitones and chromaticism, these two
    keys contain the most uniformly narrow semitones (almost exclusively
    at 90 and 96 cents) supporting smooth chromaticism, contrasted by
    the boundary keys of f and b minor which contain the most polarized
    collection of semitones (incorporating both narrow and wide
    semitones at 90 and 108 cents), lending these keys to a chromatic
    scheme that is more erratic and jagged.
  \end{enumerate}
\end{enumerate}

Moreover, these attributes also allow us to make predictions as to what
to expect on a fugal level in terms of what we should observe in regards
to subject and motivic material given key and scale construction.
Besides the general expectation of anticipating faster moving fugal
themes/subjects in meantone keys, and slower moving material in
Pythagorean keys, within Pythagorean keys we should expect to see
subjects featuring fifths, especially so in the exemplar keys of b-flat
and d-sharp minor. In regards to semitones, we should anticipate seeing
subjects/motivic material drawing attention to the modal degree
semitones in b-flat and d-sharp minor, and the leading tone for c-sharp
and f-sharp minor, as well as an expectation for chromatic material in
c-sharp and f-sharp minor to be smoother, while the treatment of
chromaticism in the boundary keys of f and b minor to be more jagged.

The main takeaway from this section is that temperament creates a wider,
and more nuanced array of minor qualities of keys, not only separating
the minor modes into the two broad categories of meantone and
Pythagorean, but also creating unique and distinct minor subcategories
within groups, with different intervallic attributes and resulting
characteristics. The conclusions and predictions, both quantitative and
theoretical, established in this chapter will be the basis of
exploration and analysis in the next chapter on the fugal level, where
we will see how these projections fare.


    % Add a bibliography block to the postdoc
    
    
    
